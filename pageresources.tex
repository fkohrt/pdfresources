% !Mode:: "TeX:DE:UTF-8:Main"
%\input{regression-test}
\RequirePackage{pdfresources}
%\documentsetup{pdfversion=2.0}
\ExplSyntaxOn
\driver_pdf_compresslevel:n {0}
\driver_pdf_compress_objects:n {0}
\ExplSyntaxOff
\documentclass{article}
\usepackage[english]{babel}
\usepackage[autostyle]{csquotes}
%\RequirePackage[customdriver=hluatex-experimental]{hyperref}
\begin{document}

\title{Sorting page resources}
\maketitle

\section{pdfpagesattr}
\begin{itemize}
\item token register, is added to the root object when the pdf is finished

\item non-additive: A new call overwrites the content of the register.
\end{itemize}

\begin{verbatim}
\pdfpagesattr{/WWWWW (blab)}
<<
/Type /Pages
/Count 1
/Kids [2 0 R]
/WWWWW (blab)
>>
\end{verbatim}

\subsection{Uses}
\begin{description}
\item[zref-pageattr]: get current content
\item[hyperref: Cropbox] \verb+/CropBox[\@pdfpagescrop]+, takes care of existing content
\item[llncsconf: Cropbox]    overwrites existing content
\item[authorarchive: Cropbox] overwrites existing content
\item[pdfx: various box sizes in pdf/x standard]  overwrites existing content
\begin{verbatim}
  \edef\next{\endgroup\pdfpagesattr{%
    /MediaBox[0 0 \pdfx@mwidth\space \pdfx@mheight]^^J
%%    /ArtBox[0 0 \pdfx@mwidth\space \pdfx@mheight]^^J
    /BleedBox[0 0 \pdfx@mwidth\space \pdfx@mheight]^^J
    /CropBox[0 0 \pdfx@mwidth\space \pdfx@mheight]^^J
    /TrimBox[25 20 \pdfx@twidth\space \pdfx@theight]}
  }\next
\end{verbatim}
and then copies to the page attribute if these are empty (why?):
\begin{verbatim}
 \EveryShipout{%
    \expandafter\ifx\expandafter\relax\the\pdfpageattr\relax
     \immediate\pdfpageattr\expandafter{\the\pdfpagesattr}%
    \fi }%
\end{verbatim}
\end{description}

\subsection{Discussion}

The values added here are simple arrays. The data structure is easy: One needs a property, a command to add and one to remove a key. The main open problem is when to move the content of the property to \verb`\pdfpagesattr` and how to force packages to use the property. Can one redefine \verb+\pdfpageattr+ so that \verb+\the\pdfpageattr+ doesn't gives bad errors?

\subsection{Implementation}
new values are stored in a prop, and at every change the register is updated. There is a test for changes so intermediate calls of the primitive can be detected and users are warned. The register is updated a last time at the end of the document.

\section{pdfpageattr}
\begin{itemize}
\item token register, is added to the current page at shipout
\item non-additive: A new call overwrites the content of the register.
\item Entries (emphasized if used/set by packages):
   \begin{enumerate}
    \item Resources (dict, see below),
    \item \emph{MediaBox}, \emph{CropBox}, \emph{Bleedbox}, \emph{Trimbox}, \emph{Artbox} (rectangle array),
    \item \emph{Rotate} (integer),
    \item \emph{Dur} (number),
    \item \emph{Trans} (dict),
    \item \emph{AA} (dict, \enquote{additional actions}),
    \item \emph{StructParents} (integer),
    \item \emph{Tabs} (name, possible values are /R, /C, /S),
    \item BoxColorInfo (dict),
    \item Contents (stream or array, automatic),
    \item Group (dict),
    \item Thumb (stream),
    \item B (array),
    \item Annots (array, automatic),
    \item Metadata (stream),
    \item PieceInfo (dict),
    \item ID (byte string),
    \item PZ (number),
    \item SeparationInfo (dict),
    \item TemplateInstantiated (name),
    \item PresSteps (dict),
    \item UserUnit (number),
    \item VP (dict).
    \end{enumerate}
   \end{itemize}
\begin{verbatim}
\pdfpageattr{/XXXXXXXX (blub)}
<<
/Type /Page
/Contents 3 0 R
/Resources 1 0 R
/MediaBox [0 0 595.276 841.89]
/XXXXXXXX (blub)
/Parent 9 0 R
>>
\end{verbatim}

\subsection{Implementation}
new values are stored in a prop, and at every change the register is updated. There is a test for changes so intermediate calls of the primitive can be detected and users are warned.

\subsection{Uses}
\begin{description}
  \item[aastex62.cls: Rotate] overwrites existing content
  \item[beamer/multimedia: AA] \verb+ /AA << \mm@pdfpageadditionalaction\space >>+, uses hyperref code to add/remove entries.
  \item[fancytoolstips: AA] \verb+/AA << /O << /S /JavaScript /JS (    \fancytooltips@pdfpageattrJS) >>  >>+, seems not to care about other packages.

  \item[gentombow: Bleedbox,Trimbox] seems not to care about other packages.
  \item[lscape: Rotate] seems not to care about other packages.
  \item[hyperref: Trans] \verb+/Trans << /S /\@pdfpagetransition\space >>+, has code to remove an existing /Trans value (if the dict is not an indirect object), takes care of existing content of the register.

  \item[hyperref: Dur] \verb+/Dur \@pdfpageduration\space+, see Trans.
  \item[hyperref: Hid] (obsolete page object)
  \item[jmlrbook.cls: MediaBox,Bleedbox,Trimbox] seems not to care about other packages.
  \item[pdflscape: Rotate] has code to remove an existing Rotate entry, takes care of existing content of the register.
  \item[zref-pageattr:] Retrieve and store content.
  \item[pdfslide (old): Trans]  seems not to care about other packages.
  \item[pdfx] See above at pdfpagesattr.
  \item[tagpdf: StructParents, Tabs] (in tree code), avoids duplicates and takes care of existing content.
  \item[aeb-pro.def: Trans, AA]
\end{description}

\subsection{Discussion}

Most values are simple and can be handled similar to pdfpagesattr. The content of Trans can be a rather complicated dict, but only one transition per page makes sense, so it is ok if one package \enquote{wins}. It is theoretically possible that the AA entry should combine content from two sources (e.g. two javascript commands) but imho it should be then the task of the packages/the users to coordinate this.

It must be decided when to copy the property into pdfpageattr. Atbeginshipout?

\section{pdfpageresources}
\begin{itemize}
\item token register, is added to the Resources dictionary of all \emph{following} pages.
\item non-additive: A new call overwrites the content of the register (but not automatic entries like Font or XObject).
\item Entries:
 \begin{enumerate}
 \item \emph{ExtGState} (dict)
 \item \emph{ColorSpace} (dict), e.g. with colorspace added page wise
 \item \emph{Pattern} (dict)
 \item \emph{Shading} (dict)
 \item \emph{Properties} (dict) (maps names to objects for BDC-dictionaries, done page wise in ocgx)
 \item Font (dict, automatic)
 \item XObject (dict, automatic)
 \item ProcSet (array, automatic) (obsolete)
 \end{enumerate}

\end{itemize}
\begin{verbatim}
\pdfpageresources{/BBBBB (blub)}
11 0 obj
<<
/Type /Page
/Contents 12 0 R
/Resources 10 0 R
/MediaBox [0 0 595.276 841.89]
/Parent 9 0 R
>>
endobj

10 0 obj
<<
/BBBBB (blub)
/Font << /F26 4 0 R /F14 5 0 R /F8 6 0 R /F31 13 0 R /F29 7 0 R >>
/ProcSet [ /PDF /Text ]
>>
endobj
\end{verbatim}

\subsection{Uses}
\begin{description}
  \item[ocg] old sty, only pdftex. Add something to Properties, takes care of the rest of the resources but not of the current content of Properties.
  \item[colorspace] Add entries to the Colorspace dict and also /ExtGState. Seems to expect that color are defined in the preamble or the first page. Unclear if it still clashes with pgf, see https://tex.stackexchange.com/a/405166/2388. Additions to ColorSpace are page wise!
  \item[hyperref] adds /OCView and /OCPrint to Properties, takes care of other content of pdfpageresources, but not of the content of Properties. (experimental driver uses central interface).
  \item[pdfbase] adds content to Properties, takes care of other content of pdfpageresources, but imho not of the content of Properties. Will probably have a problem if the value of Properties is an indirect object.
  \item[transparent] adds content to ExtGState \verb+/ExtGState<</TRP1<</ca 1/CA 1>>>>+. Tries to be compatible with pgf. (patch added to pdfresources, but need to be addressed properly in the sty).
  \item[ocg-p] similar to ocg
  \item[pdfpagediff] \verb+\edef\next{\pdfpageresources={/Properties \layersnames}}+
  \item[sesamanuel.cls] \verb+/ExtGState \@extgs\space 0 R +, saves rest of resources.
  \item[spotcolor] (old) \verb+/ColorSpace<<#1>>+, saves rest of resources.
  \item[storebox] copies pageresources to xform resources.
  \item[zwpagelayout] \verb+/ExtGState \@extgs\space 0 R+, saves rest of resources.
  \item[pgf] Reserves objects for (at least) ExtGState, ColorSpace and Pattern, Shading seems to be used directly.
\end{description}

\subsection{Discussion}
Every of the five dictionaries ExtGState, ColorSpace, Pattern, Shading, Properties can receive values from more than one package and must combine them. So every of this entries needs its own properties and commands to add to the property. Packages can avoid duplicate keys by suitable name spaces.

At least two should be handled pagewise.
It must be decided when to write the properties to pdfpageresources.

\section{pdfcatalog}
\begin{itemize}
\item The catalog dictionary (the root of the pdf) is filled by e.g. \verb+\pdfcatalog+.
\item Multiple appearances of  \verb+\pdfcatalog+ are concatenated (is this true for all drivers??).

\begin{verbatim}
\pdfcatalog{/XXX (abc)}
\pdfcatalog{/YYY (cde)}
\pdfcatalog{/XXX (cde)}
9 0 obj
<<
/Type /Catalog
/Pages 5 0 R
/XXX (abc)/YYY (cde)/XXX (cde)
>>
endobj
\end{verbatim}

\item Entries:
\begin{enumerate}
  \item Type (name, required, automatic)
  \item Pages (dict, required, automatic)
  \item Version (name, e.g. /1.4), takes precendance over the heaver \emph{if later}
  \item \emph{Collection} (dict, pdf 1.7, (for file attachments))
  \item \emph{OCProperties} (dict, pdf 1.5, required if a document contains optional content)
        contains OCGs (array of references), D (dict pointing to a start config), Configs (array of dicts).
  \item \emph{AcroForm} (dict)
  \item \emph{Metadata} (stream)
  \item \emph{PageMode} (name: UseNone, UseOutlines, UseThumbs, FullScreen, UseOC (PDF 1.5),
   UseAttachments (PDF 1.6)
  \item \emph{ViewerPreferences} (dict)
  \item \emph{OutputIntents} (array of dict)
  An array of output intent dictionaries describing the color characteristics of output devices on which the document might be rendered
  \item \emph{MarkInfo} (dict) relevant for tagging
  \item \emph{PageLabels} (number tree /indirect object)
  \item Names (dict)
  \item Dests (dict, must be an indirect reference)

  \item PageLayout (name: one of SinglePage, OneColumn,
  TwoColumnLeft, TwoColumnRight, TwoPageLeft (PDF 1.5), TwoPageRight (PDF 1.5)
  \item PageMode (name: UseNone, UseOutlines, UseThumbs, FullScreen, UseOC (PDF 1.5),
   UseAttachments (PDF 1.6)

  \item Outlines (dict, must be an indirect reference)
  \item Threads (array, must be an indirect reference)
  \item OpenAction (array (dest) or dictionary (action))
  \item AA (dict, additional-actions)
  \item URI (dict)
  \item StructTreeRoot (dict) relevant for tagging

   \item Lang (string, e.g. (de-DE))
  \item SpiderInfo (dict)
  \item PieceInfo (dict)

  \item Perms (dict, pdf 1.5, permissions)
  \item Legal (dict, pdf 1.5)
  \item Requirements (array, pdf 1.7)

  \item NeedsRendering (boolean, pdf 1.7)
  \item Extensions (dict, pdf 2.0)
  \item DSS (dict, pdf 2.0)
  \item AF (array of dictionaries, pdf 2.0, associated files, important for accessibility)
  \item DPartRoot, (dict, pdf 2.0)
\end{enumerate}
\end{itemize}



\subsection{Uses}
\begin{description}
  \item[embedfile] /Collection (dict)
  \item[ocg] /OCProperties
  \item[cooltooltips] \verb+/AcroForm \the\pdflastobj\space 0 R + (a dict)
  \item[pdfnotiz] /AcroForm (in sync with cooltooltips)
  \item[dccpaper-base] /Metadata (XML-stream)
  \item[hyperref] /OCProperties, /PageMode, /URI, openaction goto page, /ViewerPreferences<<lots of content>>, /Lang, /AcroForm (not in sync with cooltooltips), /PageLabels
  \item[hyperxmp] /Metadata
  \item[jmlrbook.cls] /OutputIntents, /Metadata
  \item[pdfbase] /AcroForm
  \item[ocgbase] /OCProperties,
  \item[ocg-p] /OCProperties
  \item[pdfpagediff] /OCProperties
  \item[pdfx] /ViewerPreferences, /OutputIntents, /Metadata
  \item[tagpdf-tree-code] /MarkInfo, /StructTreeRoot
  \item[vpe] (old, by Heiko, implements source specials with annotations) /AcroForm
  \item[xmpincl] (old, doesn't work with dvi, but nevertheless loaded by pdfx, should probably be dropped ...) /Metadata
  \item[zwpagelayout] /OutputIntents
  \item[insdljs (acrotex)] /AA, /Type/Requirement/S/EnableJavaScripts  (directly in the catalog?? Shouldn't that be inside Requirements?), /OpenAction,
  \item[aebpro.def (aeb-pro)] /ViewerPreferences,  /MarkInfo, /PageMode, /OpenAction
\end{description}

\subsection{Discussion}
Some values in the catalog need special treatment. So we will have a central property for the catalog, and a number of special properties. An entry like Lang can be added to the catalog property, but e.g ViewerPreferences needs a special property, and the general prop should only contain the indirect reference to the dict obj. Probably a number of tests are needed to prevent users to write into the wrong prop.

\begin{itemize}
 \item Collection: unclear
 \item OCProperties: (unclear) Possible that more than one package wants to add something. Then it must be done in the sub dicts OCGs and Configs and would need management.
 \item AcroForm: (difficult, with sub sub dictionaries) Entries: Fields (array of references to fields, looks as if it needs to be filled by more than one package), NeedAppearances (boolean, only one can win, deprecated in 2.0), SigFlags (only one can win), CO (array, probably only one package should add here \ldots), DR (resources dictionary, seems to have a rich structure on its own \ldots, but only hyperref fills it), DA (string), Q (Integer), XFA (stream or array, deprecated in 2.0).
     Currently the packages conflicts.

 \item Metadata: (unclear) Only one can win here, so we need to sort out, how to create and fill the stream.
 \item PageMode, Uri, Lang: (direct write)
 \item StructTreeRoot, MarkInfo: this belongs to tagging. Two packages/code trying to add a structure to a pdf would give chaos, so there must be a unique access to this keys.
 \item ViewerPreferences: needs its own property that can be filled by packages. At the end it should then be added by us to the catalog.

 \item OutputIntents: needs it own property, as more than one intent can be added. The \enquote{ID} of an intent is the value of OutputCondition, so there should be a way to test it.
\item
\end{itemize}

\section{pdfinfo}
\begin{itemize}
\item The info dictionary (in the xref-dict) is filled by e.g. \verb+\pdfinfo+.
\item Multiple appearances are concatenated.
\item Standard entries are Title, Author, Subject, Keywords, Creator, Producer (all text strings), CreationDate (date), ModDate (date), Trapped (name: either /True, /False, or /Unknown).
\item More entries are allowed, their value should be text strings.
\item If the value is empty, the entries should be removed.
\item A number of entries are added by the engine automatically but can be overwritten with own data.
CreationDate and ModDate can be suppressed with \verb+\pdfinfoomitdate+. The fullbanner with \verb+\pdfsuppressptexinfo+.
\begin{verbatim}
12 0 obj
<<
/Producer (pdfTeX-1.40.20)
/Creator (TeX)
/CreationDate (D:20190510185641+02'00')
/ModDate (D:20190510185641+02'00')
/Trapped /False
/PTEX.Fullbanner (This is pdfTeX, Version 3.14159265-2.6-1.40.20 (TeX Live 2019/W32TeX) kpathsea version 6.3.1)
>>
endobj
\end{verbatim}
\end{itemize}

\subsection{Use}
\begin{description}
  \item[ltnews.cls]
  \item[beamerthemeTorinoTh]
  \item[bgteubner.cls]
  \item[exam-n.cls]
  \item[factura.cls]
  \item[hcart.cls]
  \item[hrreport.cls]
  \item[hcslides.cls]
  \item[hustthesis.cls]
  \item[hyperref]
  \item[jmlrbook.cls]
  \item[l3build/regression-test.tex]
  \item[lion-msc.cls]
  \item[pdfprivacy]
  \item[pdfx]
  \item[skb]
  \item[skrapport.cls]
  \item[legislation.cls]
  \item[webquiz-doc.code.tex]
  \item[zwpagelayout]

\end{description}


\end{document}

