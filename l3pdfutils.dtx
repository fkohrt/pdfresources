% \iffalse meta-comment
%
%% File: l3pdfutils.dtx
%
% Copyright (C) 2018-2020 The LaTeX3 Project
%
% It may be distributed and/or modified under the conditions of the
% LaTeX Project Public License (LPPL), either version 1.3c of this
% license or (at your option) any later version.  The latest version
% of this license is in the file
%
%    http://www.latex-project.org/lppl.txt
%
% This file is part of the "(experimental) pdfmanagement bundle" (The Work in LPPL)
% and all files in that bundle must be distributed together.
%
% -----------------------------------------------------------------------
%
% The development version of the bundle can be found at
%
%    https://github.com/latex3/pdfresources
%
% for those people who are interested.
%
%<*driver>
\documentclass[full]{l3doc}
\begin{document}
  \DocInput{\jobname.dtx}
\end{document}
%</driver>
% \fi
%
% \title{^^A
%   The \pkg{l3pdfutils} package\\ pdf-utilities   ^^A
% }
%
% \author{^^A
%  The \LaTeX3 Project\thanks
%    {^^A
%      E-mail:
%        \href{mailto:latex-team@latex-project.org}
%          {latex-team@latex-project.org}^^A
%    }^^A
% }
%
% \date{Released XXXX-XX-XX}
%
% \maketitle
% \begin{documentation}
%
% \section{\pkg{l3pdfutils} documentation}
% This module contains a number of commands to create xforms.
%  \begin{function}[added = 2019-08-05]
%   {
%     \pdf_xform_new:nnn
%   }
%   \begin{syntax}
%     \cs{pdf_xform_new:nnn} \Arg{name} \Arg{attributes} \Arg{content}
%   \end{syntax}
%    This command create a new form XObject that can be used as appearance or
%    directly later.
%    If the \meta{content} contains BDC-marks it should \emph{not} be given as a
%    previously typeset box, but directly so that the names of the
%    BDC-marks can be added to the resources of the xform. The xform will automatically
%    include the resources of the current page.
%    The content will be typeset in a hbox. With pdflatex and luatex
%    the surrounding color is \emph{not} stored in the XObject
%    but should be if wanted added e.g. with |\color_select:n{.}|. This keeps
%    the option of color injection open.
%   \end{function}
%   \begin{function}[added = 2019-08-05]
%     {
%      \pdf_xform_use:n
%     }
%   \begin{syntax}
%     \cs{pdf_xform_use:n} \Arg{name}
%   \end{syntax}
%    This command uses (typesets) a previously created form XObject.
%    If the surrounding color is different, it is injected in the XObject with the
%    engines pdftex or luatex.
%   \end{function}
%   \begin{function}[EXP,added = 2019-08-05]
%     {
%       \pdf_xform_ref:n
%     }
%   \begin{syntax}
%     \cs{pdf_xform_ref:n} \Arg{name}
%   \end{syntax}
%   Inserts the appropriate information to reference the xform \meta{name}
%   in for example appearance dictionaries.
%   \end{function}
%   \begin{function}[EXP,added = 2019-08-05]
%     {
%       \pdf_xform_wd:n, \pdf_xform_ht:n, \pdf_xform_dp:n
%     }
%   \begin{syntax}
%     \cs{pdf_xform_wd:n} \Arg{name}
%   \end{syntax}
%    These command give back the sizes of the XObject. The values are stored in
%    tl-variables with the unit pt and not in dimensions!
%   \end{function}
%   \begin{function}[EXP,pTF,added = 2020-04-29]
%     {
%       \pdf_xform_if_exist:n
%     }
%   \begin{syntax}
%     \cs{pdf_xform_if_exist_p:n} \Arg{name}
%     \cs{pdf_xform_if_exist:NTF} \meta{name} \Arg{true code} \Arg{false code}
%   \end{syntax}
%    These command tests if an xform with name \Arg{name} has been already defined.
%   \end{function}%
% \end{documentation}
%
% \begin{implementation}
%
% \section{\pkg{l3pdfutils} implementation}
%
%    \begin{macrocode}
%<*package>
%<@@=pdf>
\ProvidesExplPackage {l3pdfutils} {2020-12-04} {0.2}
  {command to create xforms}
%    \end{macrocode}
% \subsection{Form XObject (pdfxform)}
% \begin{NOTE}{UF}
%  - As in dvi mode the xform is immediate, this is done for pdftex/luatex too.
%    If needed a delayed version can be added later.
%  - the argument for attributes is needed to add e.g. /StructParents
%  - it is not clear if an argument for additional resources is needed, probably they
%    should / need to be added automatically.
%  - code for adding ExtGState etc to the local resource is missing, will be
%    added when the object name is clear.
%  - should the size be stored in dim or tl?
%  - dvips implementation is missing for ideas: pdfbase, atfi-dvips.def,
%  \end{NOTE}
%
%  \subsubsection{Form XObject / management}
%    \begin{macrocode}
%<*package>
\cs_new_protected:Npn  \pdf_xform_new:nnn #1 #2 #3
  {
    \@@_backend_xform_new:nnnn { #1 } { #2 } {  } { #3 }
  }

\cs_new_protected:Npn \pdf_xform_use:n #1
  {
    \@@_backend_xform_use:n { #1 }
  }
% expansion?
\cs_new:Npn \pdf_xform_ref:n #1
  {
    \@@_backend_xform_ref:n { #1 }
  }

\cs_generate_variant:Nn \pdf_xform_ref:n {o}

\cs_new:Npn \pdf_xform_wd:n #1
  {
    \tl_use:c { c_@@_backend_xform_wd_ \tl_to_str:n { #1 } _tl }
  }

\cs_new:Npn \pdf_xform_ht:n #1
  {
    \tl_use:c { c_@@_backend_xform_ht_ \tl_to_str:n { #1 } _tl }
  }

\cs_new:Npn \pdf_xform_dp:n #1
  {
    \tl_use:c { c_@@_backend_xform_dp_ \tl_to_str:n { #1 } _tl }
  }
%</package>
%    \end{macrocode}
%
% \end{implementation}
%
% \PrintIndex
