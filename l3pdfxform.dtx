% \iffalse meta-comment
%
%% File: l3pdfxform.dtx
%
% Copyright (C) 2018-2021 The LaTeX Project
%
% It may be distributed and/or modified under the conditions of the
% LaTeX Project Public License (LPPL), either version 1.3c of this
% license or (at your option) any later version.  The latest version
% of this license is in the file
%
%    http://www.latex-project.org/lppl.txt
%
% This file is part of the "LaTeX PDF management testphase bundle" (The Work in LPPL)
% and all files in that bundle must be distributed together.
%
% -----------------------------------------------------------------------
%
% The development version of the bundle can be found at
%
%    https://github.com/latex3/pdfresources
%
% for those people who are interested.
%
%<*driver>
\RequirePackage{pdfmanagement-testphase}
\DeclareDocumentMetadata{pdfstandard=A-2b}
\makeatletter
\declare@file@substitution{doc.sty}{doc-v3beta.sty}
\makeatother
\documentclass[full]{l3doc}
\usepackage{array,booktabs,hyperxmp}
\hypersetup{pdfauthor=The LaTeX Project,pdftitle=l3pdfxform (LaTeX PDF management testphase bundle)}

\begin{document}
  \DocInput{\jobname.dtx}
\end{document}
%</driver>
% \fi
%
% \title{^^A
%   The \pkg{l3pdfxform} module\\ Commands for form XObjects   ^^A
%   \\ \LaTeX{} PDF management testphase bundle
% }
%
% \author{^^A
%  The \LaTeX{} Project\thanks
%    {^^A
%      E-mail:
%        \href{mailto:latex-team@latex-project.org}
%          {latex-team@latex-project.org}^^A
%    }^^A
% }
%
% \date{Version 0.95g, released 2021-07-21}
%
% \maketitle
% \begin{documentation}
%
% \section{\pkg{l3pdfxform} documentation}
% This module contains a number of commands to create \enquote{form XObjects},
% for which pdf\TeX{} introduced the short name \enquote{xforms}.
%
% The status is still beta.
% The dvips route is now better supported but there are open issus
% if landscape pages are involved or if the objects are larger than the page.
% With dvips the creation of a form involves typesetting and so can affect spacing.
%
% Probably some of the function will also need to be enhanced by hooks.
%
%  \begin{function}[added = 2019-08-05,updated=2021-02-14]
%   {
%     \pdfxform_new:nnn
%   }
%   \begin{syntax}
%     \cs{pdfxform_new:nnn} \Arg{name} \Arg{attributes} \Arg{content}
%   \end{syntax}
%    This command create a new form XObject that can be used as appearance or
%    directly later.
%    If the \meta{content} contains BDC-marks it should \emph{not} be given as a
%    previously typeset box, but directly so that the names of the
%    BDC-marks can be added to the resources of the xform. The xform will automatically
%    include the resources of the current page.
%    The content will be typeset in a hbox. With pdflatex and luatex
%    the surrounding color is \emph{not} stored in the form XObject
%    but should be if wanted added e.g. with |\color_select:n{.}|. This keeps
%    the option of color injection open.
%   \end{function}
%   \begin{function}[added = 2019-08-05]
%     {
%      \pdfxform_use:n
%     }
%   \begin{syntax}
%     \cs{pdfxform_use:n} \Arg{name}
%   \end{syntax}
%    This command uses (typesets) a previously created form XObject.
%    If the surrounding color is different, it is injected in the form XObject with the
%    engines pdftex or luatex.
%   \end{function}
%   \begin{function}[EXP,added = 2019-08-05]
%     {
%       \pdfxform_ref:n
%     }
%   \begin{syntax}
%     \cs{pdfxform_ref:n} \Arg{name}
%   \end{syntax}
%   Inserts the appropriate information to reference the xform \meta{name}
%   in for example appearance dictionaries.
%   \end{function}
%   \begin{function}[EXP,added = 2019-08-05]
%     {
%       \pdfxform_wd:n, \pdfxform_ht:n, \pdfxform_dp:n
%     }
%   \begin{syntax}
%     \cs{pdfxform_wd:n} \Arg{name}
%   \end{syntax}
%    These command give back the sizes of the form XObject. The values are stored in
%    tl-variables with the unit pt and not in dimensions!
%   \end{function}
%   \begin{function}[EXP,pTF,added = 2020-04-29]
%     {
%       \pdfxform_if_exist:n
%     }
%   \begin{syntax}
%     \cs{pdfxform_if_exist_p:n} \Arg{name}
%     \cs{pdfxform_if_exist:nTF} \meta{name} \Arg{true code} \Arg{false code}
%   \end{syntax}
%    These command tests if an xform with name \Arg{name} has been already defined.
%   \end{function}%
% \end{documentation}
%
% \begin{implementation}
%
% \section{\pkg{l3pdfxform} implementation}
%
%    \begin{macrocode}
%<@@=pdf>
%<*header>
\ProvidesExplPackage{l3pdfxform}{2021-07-21}{0.95g}
  {command to create xforms (beta)---LaTeX PDF management testphase bundle}
%</header>
%    \end{macrocode}
% \subsection{Form XObject (pdfxform)}
% \begin{NOTE}{UF}
%  - As in dvi mode the xform is immediate, this is done for pdftex/luatex too.
%    If needed a delayed version can be added later.
%  - the argument for attributes is needed to add e.g. /StructParents
%  - it is not clear if an argument for additional resources is needed, probably they
%    should / need to be added automatically.
%  - code for adding ExtGState etc to the local resource is missing, will be
%    added when the object name is clear.
%  - should the size be stored in dim or tl?
%  - dvips implementation is missing for ideas: pdfbase, atfi-dvips.def,
%  \end{NOTE}
% \begin{macro}{\pdfxform_new:nnn}
%    \begin{macrocode}
%<*package>
\cs_new_protected:Npn  \pdfxform_new:nnn #1 #2 #3
  {
    \@@_backend_xform_new:nnnn { #1 } { #2 } {  } { #3 }
  }
%    \end{macrocode}
% \end{macro}
% \begin{macro}{\pdfxform_use:n}
%    \begin{macrocode}
\cs_new_protected:Npn \pdfxform_use:n #1
  {
    \@@_backend_xform_use:n { #1 }
  }
%    \end{macrocode}
% \end{macro}
% \begin{macro}{\pdfxform_ref:n,\pdfxform_ref:o}
%    \begin{macrocode}
% expansion?
\cs_new:Npn \pdfxform_ref:n #1
  {
    \@@_backend_xform_ref:n { #1 }
  }

\cs_generate_variant:Nn \pdfxform_ref:n {o}
%    \end{macrocode}
% \end{macro}
% \begin{macro}{\pdfxform_wd:n,\pdfxform_ht:n,\pdfxform_dp:n,}
%    \begin{macrocode}
\cs_new:Npn \pdfxform_wd:n #1
  {
    \tl_use:c { c_@@_backend_xform_wd_ \tl_to_str:n { #1 } _tl }
  }

\cs_new:Npn \pdfxform_ht:n #1
  {
    \tl_use:c { c_@@_backend_xform_ht_ \tl_to_str:n { #1 } _tl }
  }

\cs_new:Npn \pdfxform_dp:n #1
  {
    \tl_use:c { c_@@_backend_xform_dp_ \tl_to_str:n { #1 } _tl }
  }
%    \end{macrocode}
% \end{macro}
% \begin{macro}[pTF]{\pdfxform_if_exist:n}
% The command is defined in the backend (to avoid timing problems).
% \end{macro}
%    \begin{macrocode}
%</package>
%    \end{macrocode}
%
% \end{implementation}
%
% \PrintIndex
