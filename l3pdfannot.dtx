% \iffalse meta-comment
%
%% File: l3pdfannot.dtx
%
% Copyright (C) 2021 The LaTeX Project
%
% It may be distributed and/or modified under the conditions of the
% LaTeX Project Public License (LPPL), either version 1.3c of this
% license or (at your option) any later version.  The latest version
% of this license is in the file
%
%    http://www.latex-project.org/lppl.txt
%
% This file is part of the "PDFmanagement bundle (testphase)" (The Work in LPPL)
% and all files in that bundle must be distributed together.
%
% -----------------------------------------------------------------------
%
% The development version of the bundle can be found at
%
%    https://github.com/latex3/pdfresources
%
% for those people who are interested.
%
%<*driver>
\RequirePackage{pdfmanagement-testphase}
\DeclareDocumentMetadata{pdfstandard=A-2b}
\makeatletter
\declare@file@substitution{doc.sty}{doc-v3beta.sty}
\makeatother
\documentclass[full]{l3doc}
\usepackage{array,booktabs,hyperxmp}
\hypersetup{pdfauthor=The LaTeX Project,pdftitle=l3pdfannot (PDFmanagement bundle (testphase))}
\begin{document}
  \DocInput{\jobname.dtx}
\end{document}
%</driver>
% \fi
% \providecommand\hook[1]{\texttt{#1}}
% \title{^^A
%   The \pkg{l3pdfannot} package\\ Commands for PDF annotations   ^^A
%   \\ PDFmanagement bundle (testphase)
% }
%
% \author{^^A
%  The \LaTeX{} Project\thanks
%    {^^A
%      E-mail:
%        \href{mailto:latex-team@latex-project.org}
%          {latex-team@latex-project.org}^^A
%    }^^A
% }
%
% \date{Version 0.95a, released 2021-02-21}
%
% \maketitle
% \begin{documentation}
%
% \section{\pkg{l3pdfannot} documentation}
% This module contains a number of commands to create PDF annotations.
% The commands are \emph{not} always simple wrappers around primitive commands.
% To allow external packages to configure links and other annotations,
% some of the commands have hooks and use shared attribute dictionaries.
% For these commands the hooks and dictionaries are selected depending on the \meta{type}
% of the annotation. Currently the module only supports some general commands
% and link annotations. Commands for other annotations like widgets will be added
% later.
%
% \subsection{General annotation commands}
%
% \begin{function}[added = 2019-09-05, updated = 2020-04-14]
%   { \pdfannot_box:nnnn }
%   \begin{syntax}
%     \cs{pdfannot_box:nnnn} \Arg{width} \Arg{height} \Arg{depth} \Arg{annot spec}
%   \end{syntax}
%   This creates an \texttt{/Type/Annot} object with the given dimensions.
%   It doesn't use hooks or dictionaries.
% \end{function}
% \begin{function}[added = 2020-03-30]
%   { \pdfannot_box:nnnnn }
%   \begin{syntax}
%     \cs{pdfannot_box:nnnnn} \Arg{type} \Arg{width} \Arg{height} \Arg{depth} \Arg{annot spec}
%   \end{syntax}
%   This creates an \texttt{/Type/Annot} object with the given dimensions.
%   \meta{type} should be currently one of |link/URI|, |link/GoToR|,
%  |link/Launch|, |link/GoTo| or |link/Named| or
%  |widget|, it will then insert the attribute dictionary of this type
%   additionally to the manually given \meta{annot spec}.
%   The attribute dictionaries can be filled with commands described below.
%   Hooks are not used.
% \end{function}
% \begin{function}[added = 2019-09-05]
%   { \pdfannot_box_ref_last: }
%   \begin{syntax}
%     \cs{pdfannot_box_ref_last:}
%   \end{syntax}
%   This retrieves the object reference of the last box annotation created.
% \end{function}
% \subsection{Link annotations}
% Link annotations are special cases of annotations. In the PDF they are identified
% by an |/Subtype/Link| entry in the dictionary.
% Link annotations are quite important as many documents contain links,
% both internal and external. They need a set of special commands for two reasons:
%
% At first the content of links are not only boxes. Links can contain line
% and page breaks (this is normally implemented by the primitive command by
% creating a set of annotations).
%
% At second link annotations are objects that need some
% \enquote{management} as more than one
% package wants to configure their look and behaviour.
% For example \pkg{hyperref}, \pkg{ocgx2} and the code for tagged PDF (currently
% in \pkg{tagpdf}) all want to add keys and values to the dictionaries of
% link annotation and code around links.
% So commands to create link annotations should offer suitable hooks.
% There are three standard places in a link where such hooks are needed:
% At the begin (for example for a structure command or color),
% in the \emph{attr spec} dictionary of the link (for example for the border), and
% at the end of the link (to close a structure or the color group).
% For the begin and end hooks of the LaTeX hook management are predefined and used.
% To add and remove values from the \emph{attr spec} dictionary special
% commands described below are provided. The link commands switch to horizontal mode
% as the commands of pdftex and luatex can't be used in vertical mode.
%
% \begin{variable}{\c_pdfannot_link_types_seq}
% There are currently five link types, \texttt{URI}, \texttt{GoToR},
%  \texttt{Launch}, \texttt{GoTo} or \texttt{Named}, and there are store in this
%  constant.
% \end{variable}
% \begin{variable}
%  {
%   pdfannot/link/TYPE/before,
%   pdfannot/link/TYPE/begin,
%   pdfannot/link/TYPE/end,
%   pdfannot/link/TYPE/after
%  }
%  These are the hooks used by the following commands. TYPE can be one of
%  \texttt{URI}, \texttt{GoToR},
%  \texttt{Launch}, \texttt{GoTo} or \texttt{Named}
% \end{variable}
% \begin{variable}
%  {
%    link/TYPE
%  }
%  These is the name of the dictionary used by the following commands. TYPE can be one of
%  \texttt{URI}, \texttt{GoToR},
%  \texttt{Launch}, \texttt{GoTo} or \texttt{Named}. The dictionary can be changed
%  by the commands \cs{pdfannot_dict_put:nnn} and friends described below.
% \end{variable}
% \begin{function}[added = 2020-03-12, updated = 2020-12-06]{ \pdfannot_link:nnn }
%   \begin{syntax}
%     \cs{pdfannot_link:nnn} \Arg{type} \Arg{user action spec} \Arg{link text}
%   \end{syntax}
%  This creates a link around the \meta{link text} with the specified
%  \meta{user action spec}\footnote{The wording follows the pdftex documentation}.
%  \texttt{/Subtype/Link} is added automatically.
%  \meta{type} should be one of \texttt{URI}, \texttt{GoToR},
%  \texttt{Launch}, \texttt{GoTo} or \texttt{Named}. The |GoTo| variant does
%  \emph{not} complain if the destination name is not known like
%  \cs{pdfannot_link_goto_begin:nw}.
%  The attributes stored in the local dictionary
%  \texttt{link/}\meta{type} are inserted as
%  \emph{attr spec} and the code in the begin and end hook
%  \texttt{pdfannot/link/\meta{type}/before}
%  and \texttt{pdfannot/link/\meta{type}/after}
%  is executed  before and after the link (outside the link command)
%  while \texttt{pdfannot/link/\meta{type}/begin}
%  and \texttt{pdfannot/link/\meta{type}/end} are directly around the link
%  text. None of the hooks introduce a group.
%  \meta{type} should normally be identical to the value of the |/S| key
%  in the action dictionary.
%  As example
%  \begin{verbatim}
%    \pdfannot_dict_put:nnn
%      {link/URI} { C } {[1~0~0]} %red border
%    \pdfannot_link:nnn { URI }
%     {
%       /A
%         <<
%           /Type/Action
%           /S/URI
%           /URI(https://www.latex-project.org)
%         >>
%     }
%     { link text }
%    \end{verbatim}
%
%
% \end{function}
% \begin{function}[updated = 2020-12-06]{ \pdfannot_link_begin:nnw, \pdfannot_link_end:n }
%   \begin{syntax}
%     \cs{pdfannot_link_begin:nnw} \Arg{type} \Arg{user action spec} \meta{content}
%     \cs{pdfannot_link_end:n} \Arg{type}
%   \end{syntax}
%  This creates a link around the \meta{content} with the specified
%  \meta{user action spec} (e.g. an /A dictionary with an URI) or
%  \meta{destination} (a name as defined with the
%  first argument of \cs{pdf_destination:nn}).
%  \texttt{/Subtype/Link} is added automatically.
%  In contrast to \cs{pdfannot_link:nnn} this function
%  does not absorb the argument when finding the \meta{content}, and so can
%  be used in circumstances where the \meta{content} may not be a simple
%  argument. But beside this, it works similar and use the same hooks.
%  As example
%  \begin{verbatim}
%    \pdfannot_link_begin:nnw { URI }
%     {
%       /A<<
%         /Type/Action
%         /S/URI
%         /URI(https://www.latex-project.org)
%       >>
%     }
%     link text
%    \pdfannot_link_end:n { URI }
%    \end{verbatim}
% \end{function}
% \begin{function}[updated = 2020-12-06]{ \pdfannot_link_goto_begin:nw, \pdfannot_link_goto_end: }
%   \begin{syntax}
%     \cs{pdfannot_link_goto_begin:nw} \Arg{destination} \meta{content}
%     \cs{pdfannot_link_goto_end:}
%   \end{syntax}
% This is a special, shorter version for links to internal destinations. It always
% uses the hooks and dictionary of the |GoTo| link type. \meta{destination} is a
% destination name. In difference to |\pdfannot_link:nnn { GoTo }| it will complain if
% \meta{destination} is an unknown destination and give the message
%
% |name{ZZZZ} has been referenced but does not exist, replaced by a fixed one|
%
% \end{function}
% \begin{function}[added = 2021-02-14]{  \pdfannot_link_ref_last: }
%   This retrieves the object reference a link created previously with the commands
%   above. This doesn't work currently with xelatex but a feature request has
%   been made. see https://tug.org/pipermail/dvipdfmx/2020-December/000134.html
% \end{function}
% \begin{function}[added = 2021-02-14]{  \pdfannot_ref_last: }
%   This retrieves the object reference a previously annotation
%   created either with a link or a general box command. When the last was a link
%   it won't work with xelatex.
%   see https://tug.org/pipermail/dvipdfmx/2020-December/000134.html
% \end{function}
% \begin{NOTE}{UF}
% only annot link or also annot?
% \end{NOTE}
% \begin{function}[added = 2020-03-12]{  \pdfannot_link_margin:n }
%  \begin{syntax}
%    \cs{pdfannot_link_margin:n} \Arg{dimen}
%  \end{syntax}
%   This sets the dimension of the link margin.
% \end{function}
% \begin{function}[added = 2020-12-04]{ \pdfannot_dict_put:nnn }
%  \begin{syntax}
%    \cs{pdfannot_dict_put:nnn} \Arg{dictionary name} \Arg{key} \Arg{value}
%  \end{syntax}
%   This adds (locally) a key-value to the internal annot dictionaries used
%   by the link commands above.
%   \meta{dictionary name} should be currently one of \texttt{link/URI},
%   \texttt{link/URI},\texttt{link/GoToR}, \texttt{link/Launch},
%   \texttt{link/GoTo}, \texttt{link/Named}.
% \end{function}
% \begin{function}[added = 2020-12-04]{ \pdfannot_dict_remove:nn }
%  \begin{syntax}
%    \cs{pdfannot_dict_remove:nn} \Arg{dictionary name} \Arg{key}
%  \end{syntax}
%   This removes a key-value from the internal annot dictionary
%   \meta{dictionary name} should be currently one of
%   \texttt{link/URI}, \texttt{link/GoToR}, \texttt{link/Launch},
%   \texttt{link/GoTo}, \texttt{link/Named}.
% \end{function}
% \begin{function}[added = 2020-12-04]{ \pdfannot_dict_show:n }
%  \begin{syntax}
%    \cs{pdfannot_dict_show:n} \Arg{dictionary name}
%  \end{syntax}
%   This shows the content of the internal annot dictionary.
%  \meta{dictionary name} should be currently one of \texttt{link/URI},
%   \texttt{link/URI}, \texttt{link/GoToR}, \texttt{link/Launch},
%   \texttt{link/GoTo}, \texttt{link/Named}.
% \end{function}
%
% \begin{variable}[added = 2020-12-28]{\l_pdfannot_F_bitset}
%  This is a bitset variable, with the named index names suitable for the
%  /F flag in an annotation.
%  It can be used for example like this:
%  \begin{verbatim}
%  \pdfannot_dict_put:nnn {link/URI} {F}
%     { \bitset_to_arabic:N \l_pdfannot_F_bitset }
%  \bitset_set_true:Nn \l_pdfannot_F_bitset {Print}
%  \end{verbatim}
%  The known keys for the bitset are |Invisible|, |Hidden|,
%  |Print|, |NoZoom|, |NoRotate|, |NoView|, |ReadOnly|, |Locked|,  |ToggleNoView|,
%  |LockedContents| which correspond to the names used in the PDF references.
% \end{variable}
% \end{documentation}
%
% \begin{implementation}
%
% \section{\pkg{l3pdfannot} implementation}
%
%    \begin{macrocode}
%<@@=pdfannot>
%<*header>
\ProvidesExplPackage {l3pdfannot} {2021-02-22} {0.95a}
  {PDF-annotations}
\RequirePackage{l3pdfdict}
%</header>
%    \end{macrocode}
% Annotations have a /F flag, we provide a public
% bitset for it.
%    \begin{macrocode}
%<*package>
\RequirePackage{l3bitset}
\bitset_new:Nn \l_pdfannot_F_bitset
  {
    Invisible      = 1,
    Hidden         = 2,
    Print          = 3,
    NoZoom         = 4,
    NoRotate       = 5,
    NoView         = 6,
    ReadOnly       = 7,
    Locked         = 8,
    ToggleNoView   = 9,
    LockedContents = 10
  }
%    \end{macrocode}
% \begin{NOTE}{UF}
% The code/naming tries to unify general annotations and the special type of
% link under a common name.
% regarding naming and relation of annotation commands see
% https://github.com/FrankMittelbach/AccessiblePDF/issues/73
% \subsection{Annotations / backend}
% The backend commands are in l3backend:
% \cs{__pdf_backend_annotation:nnnn} and \cs{__pdf_backend_annotation_last:}
% \cs{__pdf_backend_link_begin_user:nnw}, etc
% \end{NOTE}
%
% \subsection{ General Annotations }
% \begin{variable}
%  { \g_@@_use_lastlink_bool }
% The pdf engines have two different primitive commands to refer to the last created
% annotation: one for links, one for boxed annotation. We use a boolean to decide
% which one should be used, so that only one user command is needed.
%    \begin{macrocode}
\bool_new:N \g_@@_use_lastlink_bool
%    \end{macrocode}
% \end{variable}
% \begin{NOTE}{UF}
%  type or not type? Syntax for type?
%  should there be a version without type?
% \end{NOTE}

%    \begin{macrocode}
\cs_new_protected:Npn \pdfannot_box:nnnn #1 #2 #3 #4
  {
    \__pdf_backend_annotation:nnnn {#1}{#2}{#3}{#4}
    \bool_gset_false:N\g_@@_use_lastlink_bool
  }

\cs_new:Npn \pdfannot_box_ref_last:
  {
    \__pdf_backend_annotation_last:
  }

\cs_new_protected:Npn \pdfannot_box:nnnnn #1 #2 #3 #4 #5
  {
    \exp_args:Nx
    \__pdf_backend_annotation:nnnn {#2}{#3}{#4}
      {
        \pdfdict_if_exist:nT { l_@@/#1 }
          {
            \pdfdict_use:n { l_@@/#1}
          }
        #5
      }
    \bool_gset_false:N\g_@@_use_lastlink_bool
  }
%    \end{macrocode}
% \subsection{Annotations, subtype Widget}\label{pdf:annot:widget}
% Currently no code is provided here.
% The local dictionary \texttt{l_@@/Widget} is a skeleton
% dictionary for this subtype. It currently contains as only entry
% the subtype setting (the /Type is added by the backend).
%    \begin{macrocode}
 \pdfdict_new:n { l_@@/widget }
 \pdfdict_put:nnn { l_@@/widget }{ Subtype }{ /Widget }
%    \end{macrocode}
%
% \subsection{Annotations, subtype Link}\label{sec:links}
% The code assumes that there will be different link types
% (currently URI, GoToR, Launch, GoTo, Named, hyperref uses the names
% url,file,run,link,menu) and that links of the same type share
% the \emph{attr spec} and also the same begin/end
% code. The list of link types need to stay restricted and well documented so that
% all packages know which types they have to handle. It is stored in a constant
% seq.
% \begin{NOTE}{UF}
% Perhaps a |cite| type will be useful at some time. -- Thinking more about it,
% a |cite| type is not sensible. hyperref supports it, but it doesn't fit in.
% Commands like cite, gls, acro, footnote and so on should locally change
% linkcolor and linkbordercolor.
% Probably we will need some commands to add an attribute to all link types
% at once.
% hyperref commands for the various type:
% url  |\hyper@linkurl|,
% file |\hyper@linkfile|,
% run  |\@hyper@launch run|,
% link |\hyper@link|, |\find@pdflink|
% menu |\Acrobatmenu|
% \end{NOTE}
% \begin{variable}[added = 2020-03-12]{ \c_pdfannot_link_types_seq }
% This constant sequence contains the list of currently supported link types
% for which hooks and dictionaries exist.
% \end{variable}
%
% \begin{variable}
%   {
%     link/TYPE,
%     pdfannot/link/TYPE/before,
%     pdfannot/link/TYPE/begin,
%     pdfannot/link/TYPE/end,
%     pdfannot/link/TYPE/after
%   }
% These setup the dictionary and the hook pairs.
%    \begin{macrocode}
\seq_const_from_clist:Nn \c_pdfannot_link_types_seq { URI , GoToR , Launch , GoTo, Named }
\seq_map_inline:Nn \c_pdfannot_link_types_seq
  {
    \pdfdict_new:n { l_@@/link/#1 }
    \hook_new_pair:nn
      {pdfannot/link/#1/before}
      {pdfannot/link/#1/after}
    \hook_new_pair:nn
      {pdfannot/link/#1/begin}
      {pdfannot/link/#1/end}
   }
%    \end{macrocode}
% \end{variable}
% \subsubsection{Annotations, subtype Link /management}
%
% \begin{macro}{\pdfannot_link:nnn,\pdfannot_link:nxn}
%    \begin{macrocode}
\cs_new_protected:Nn \pdfannot_link:nnn %#1 type (URI, GoTo etc),
                                        %#2 action spec, #3 link text
  {
    \hook_use:n { pdfannot/link/#1/before}
    \mode_leave_vertical:
    \exp_args:Nxx %xetex needs expansion
    \__pdf_backend_link_begin_user:nnw
      {
         \pdfdict_if_exist:nT { l_@@/link/#1 }
          {
            \pdfdict_use:n { l_@@/link/#1}
          }
      }
      {
        /Subtype/Link
        #2 %exp_not?
      }
    \bool_gset_true:N \g_@@_use_lastlink_bool
    \hook_use:n { pdfannot/link/#1/begin}
    #3
    \hook_use:n { pdfannot/link/#1/end}
    \__pdf_backend_link_end:
    \bool_gset_true:N \g_@@_use_lastlink_bool
    \hook_use:n { pdfannot/link/#1/after}
  }
\cs_generate_variant:Nn \pdfannot_link:nnn {nxn}
%    \end{macrocode}
% \end{macro}
% \begin{macro}{
%   \pdfannot_link_begin:nnw,
%   \pdfannot_link_begin:nxw,
%   \pdfannot_link_end:n }
%    \begin{macrocode}
\cs_new_protected:Npn \pdfannot_link_begin:nnw #1 #2 %#1 type, #2 action spec
  {
    \hook_use:n { pdfannot/link/#1/before}
    \mode_leave_vertical:
    \exp_args:Nxx %xetex needs expansion
      \__pdf_backend_link_begin_user:nnw
        {
           \pdfdict_if_exist:nT { l_@@/link/#1 }
            {
              \pdfdict_use:n { l_@@/link/#1}
            }
        }
        { #2 }
      \bool_gset_true:N \g_@@_use_lastlink_bool
      \hook_use:n { pdfannot/link/#1/begin}
  }

\cs_new_protected:Nn \pdfannot_link_end:n %#1 type, e.g. url
  {
    \hook_use:n { pdfannot/link/#1/end}
    \__pdf_backend_link_end:
    \bool_gset_true:N \g_@@_use_lastlink_bool
    \hook_use:n { pdfannot/link/#1/after}
  }
\cs_generate_variant:Nn \pdfannot_link_begin:nnw {nxw}
%    \end{macrocode}
% \end{macro}
% \begin{macro}{\pdfannot_link_goto_begin:nw, \pdfannot_link_goto_end:}
%     \begin{macrocode}
\cs_new_protected:Npn \pdfannot_link_goto_begin:nw #1 %#1 destination
  {
    \hook_use:n { pdfannot/link/GoTo/before}
    \mode_leave_vertical:
    \exp_args:Nxx %xetex needs expansion
    \__pdf_backend_link_begin_goto:nnw
      {
        \pdfdict_use:n { l_@@/link/GoTo}
      }
      { #1 }
     \bool_gset_true:N \g_@@_use_lastlink_bool
     \hook_use:n { pdfannot/link/GoTo/begin}
  }

\cs_new_protected:Nn \pdfannot_link_goto_end:
  {
    \hook_use:n { pdfannot/link/GoTo/end}
    \__pdf_backend_link_end:
    \bool_gset_true:N \g_@@_use_lastlink_bool
     \hook_use:n { pdfannot/link/GoTo/after}
  }
%    \end{macrocode}
% \end{macro}
% \begin{macro}[EXP]{\pdfannot_link_ref_last:, \pdfannot_ref_last:}
%    \begin{macrocode}
\cs_new:Nn \pdfannot_link_ref_last: { \__pdf_backend_link_last: }
%    \end{macrocode}
%    \begin{macrocode}
\cs_new:Npn \pdfannot_ref_last:
  {
    \bool_if:NTF \g_@@_use_lastlink_bool
      {
        \__pdf_backend_link_last:
      }
      {
        \__pdf_backend_annotation_last:
      }
  }
%    \end{macrocode}
% \end{macro}
% \begin{macro}{ \pdfannot_link_margin:n}
%    \begin{macrocode}
\cs_new_protected:Npn \pdfannot_link_margin:n #1
  {
    \__pdf_backend_link_margin:n { #1 }
  }
%    \end{macrocode}
% \end{macro}
% \begin{macro}
%   {
%     \pdfannot_dict_put:nnn,
%     \pdfannot_dict_put:nnx,
%     \pdfannot_dict_remove:nn,
%     \pdfannot_dict_show:n
%   }
%    \begin{macrocode}
\cs_new_protected:Npn \pdfannot_dict_put:nnn #1 #2 #3
  {
    \pdfdict_put:nnn { l_@@/#1 } { #2 }{ #3 }
  }
\cs_generate_variant:Nn \pdfannot_dict_put:nnn {nnx}
%    \end{macrocode}
%    \begin{macrocode}
\cs_new_protected:Npn \pdfannot_dict_remove:nn #1 #2
  {
    \pdfdict_remove:nn { l_@@/#1 } { #2 }
  }
%    \end{macrocode}
%    \begin{macrocode}
\cs_new_protected:Npn \pdfannot_dict_show:n #1
  {
    \pdfdict_show:n { l_@@/#1 }
  }
%</package>
%    \end{macrocode}
% \end{macro}
% \end{implementation}
%
% \PrintIndex
