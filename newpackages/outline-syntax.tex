dvipdfmx /Count is created automatically, so mainly the actions and options must be handled.
\bookmarks writes out the needed \special directly. 

 \special{pdf:%
        out 
        [] OR [-] %depending is open or not
        \BKM@abslevel %current level. Must be equal or one off to previous level
        <<
         /Title(\BKM@title)%
        
         /C[\BKM@color] %color, optional
        
         /F \BKM@FLAGS  % flags, optional

         \BKM@action    % action. 
            -> {/Dest[@page\BKM@page/\BKM@view]}
            -> /D[\BKM@page/\BKM@view]
            -> /D(\BKM@dest)
            -> /A<</S/URI /URI(\BKM@uri) >>
            -> /A<<\BKM@rawaction>>
            -> /A<</S/Named/N/\BKM@named>>%
            -> /A<</S/GoTo /D(\BKM@dest)>>%
            -> /A<</S/GoToR/F(\BKM@gotor)\BKM@action >>%            
        >>%
      }%

dvips writes out at the end a header file, jobname.out.ps, which is referenced with \special{header=#1}%
  count must done by the document, so every \bookmark creates a BKM@<id> command which stores {parent,level,count}.
  Childs update the count of their parent. Beside this every \bookmark writes the info with the destination (and more) to the aux-file as
  a command \BKM@entry. When the auxfile is read at end document these \BKM@entry writes the header file.
  
pdflatex,luatex: similar to dvips, only that \BKM@entry uses \pdfoutline directly.    


Catalog 
   /Outlines 56 0 R
             56 0 obj
             <<
            /Type /Outlines
            /First 24 0 R
            /Last 26 0 R
            /Count 3
            >>
            endobj
               24 0 obj
               <<
               /Title 25 0 R
               /A 23 0 R      --> << /S /GoTo /D (section.1) >>
               /Parent 56 0 R
               /Next 33 0 R
               /First 27 0 R
               /Last 30 0 R
               /Count -2
               >>
               endobj
                  27 0 obj
                 <<
                 /Title 28 0 R
                 /A 26 0 R
                 /Parent 24 0 R
                 /Next 30 0 R
                 >>
                 endobj
               
               33 0 obj
               <<
               /Title 34 0 R
               /A 32 0 R
               /Parent 56 0 R
               /Prev 24 0 R
               /Next 36 0 R
               >>
               endobj
               
               
Action types 

/Type /Action /S XXXX /Next XXX 

GoTo  : /D  (goto action is the same as /Dest)

GoToR : /F, /D, /NewWindow (goto remote)

GoToE : /F, /D, /NewWindow, /T (lots of subvalues) (goto embedded), pdf 1.6

GoToDp: (2.0) /Dp (goto part)

Launch:  /F, /NewWindows, (/Win etc deprecated in pdf 2.0)

Thread: /F, /D, /B

URI: /URI, /IsMap


Sound: /Sound, /Volume,/Synchronous,/Repeat,/Mix

Movie: deprecated in 2.0

Hide: /T, /H
 Set an annotation’s Hidden flag. 

Named /N (e.g. /N /NextPage)

SubmitForm
(PDF 1.2) Send data to a uniform resource locator.
“Submit-Form Actions” on page 703
ResetForm
(PDF 1.2) Set fields to their default values.
“Reset-Form Actions” on page 707
ImportData
(PDF 1.2) Import field values from a file.
“Import-Data Actions” on page 708
JavaScript
(PDF 1.3) Execute a JavaScript script.
“JavaScript Actions” on page 709
SetOCGState /State , /PreserveRB

Rendition

Trans
(PDF 1.5) Updates the display of a document, using a transition dictionary.
“Transition Actions” on page 670
GoTo3DView               
