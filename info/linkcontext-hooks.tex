% !Mode:: "TeX:UTF-8:Main"
\DocumentMetadata{}
\documentclass{l3doc}
\usepackage{tcolorbox}
\tcbuselibrary{listings}

\MakeShortVerb|
\hypersetup{pdfborder=0 0 1}

\IfFormatAtLeastTF{2021-11-15}
 {}
 {\providecommand\ActivateGenericHook{\ProvideHook}}

\ActivateGenericHook{hyp/link/acro} %cheating
\ActivateGenericHook{hyp/link/link}

\title{Handling conceptual links  with generic hooks}
\author{Ulrike Fischer}
\begin{document}

\maketitle

The basic command \pkg{hyperref} uses to create an internal document link (a \enquote{GoTo} link in PDF speech)
has a quite unknown first argument which can be used to differentiate the links.
Only two values for this argument are currently in use: |cite| used by various citation commands and |link| used
for everything else. The actual effect of the argument is quite restricted in older \pkg{hyperref} versions:
It only calls |\csname @#1color\endcsname| at some point, so it simply
selects a color.

In the new \pkg{hyperref} driver which is used when the pdfmanagement is active this argument has been connected
with a generic hook, |hyp/link/|\meta{name}. This hook is executed at the begin of the link command, outside
the actual link but inside a group.

Packages or code which create links which are used for a specific concept,
like e.g. citations, or acronyms, or glossary terms,
or links to an author list can make use of this to make the links customizable by the user.
They only need to create the link with |\hyper@linkstart| and |\hyper@linkend| with a suitable
first argument (both commands are official interfaces of \pkg{hyperref} for links) and to activate the
relevant hook (for |cite| the hook is already declared, \pkg{hyperref} uses it to reimplement
the |citecolor| keys. So it can be used directly).
For other types (including the default type |link|) the hooks must be activated first.
This can be done with |\ActivateGenericHook|.


The hook can e.g. change colors with |\hypersetup| or add some text before the link.
The color of a internal link is stored in |hyp/color/link|,
using |\colorlet| to change it is possible too, but this method is less clean: it relies a bit
on some knowledge about the internal processing, and doesn't work for border colors which are
set differently to text colors.





\hypertarget{some-target}{}

\begin{tcblisting}{title= Example with the predefined hook for the keyword \enquote{cite}}
\makeatletter
\hyper@linkstart{cite}{some-target}1234\hyper@linkend

\AddToHook{hyp/link/cite}
 {Cite:~\hypersetup{linkcolor=blue,linkbordercolor=yellow}}

\hyper@linkstart{cite}{some-target}1234\hyper@linkend

\end{tcblisting}



\begin{tcblisting}{title= Activation of a new keyword}

\ActivateGenericHook{hyp/link/acro}
\AddToHookNext{hyp/link/acro}{Acro:~\colorlet{hyp/color/link}{blue}\sffamily}

\makeatletter
\hyper@linkstart{acro}{some-target}acronym\hyper@linkend\ more text
\makeatother

\end{tcblisting}


\begin{tcblisting}{title= Activation of the existing keyword \enquote{link}}

\AddToHook{hyp/link/link}{Link:~\colorlet{hyp/color/link}{orange}\sffamily}
\ActivateGenericHook{hyp/link/link}

\makeatletter
\hyper@linkstart{link}{some-target}see 1\hyper@linkend
\makeatother

\hyperlink{some-target}{see 2}
\end{tcblisting}


\begin{tcblisting}{title= keyword without hook activation}

\hypersetup{linkcolor=brown,linkbordercolor=red}
\AddToHook{hyp/link/unknown}{Blub??}
\makeatletter
\hyper@linkstart{unknown}{some-target}acronym\hyper@linkend\ more text
\makeatother

\end{tcblisting}


\end{document} 