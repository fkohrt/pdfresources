%%%%%%%%%%%%%%%%%%
Internal links
%%%%%%%%%%%%%%%%%

\hyper@linkstart :
   defined in driver,
   used in hyperref.sty!
   two arguments!
   first : color (will ignore it ...)
   second: destination name
   \hyper@linkstart{link}{\Hy@footnote@currentHref}
   \hyper@linkstart{cite}{cite.#1}%
   \hyper@linkstart{link}{\Hy@tocdestname} (toc)

 internal command
  \find@pdflink{#1}{#2}
   defined in driver,
   used only in driver
  ->  \pdf_link_begin_goto:nnw { link } { #2 }
      \expandafter\Hy@colorlink\csname @#1color\endcsname

\hyper@link
   three arguments, basically \hyper@linkstart/\hyper@linkend with
   content as third argument.
   Used in \hyper@link@ - >
    \hyper@@link[cite]
    \hyper@@link[link]
    #1 either link or cite

%%%%%%%%%%%%%%%%%%%%%

\hyper@anchor, \hyper@anchorstart, \hyper@anchorend :
  defined in driver,
  used in hyperref.sty
  one argument: destination name
  
  \hyper@anchor assumes that the outside caller correctly sets \anchor@spot (and the outside callers assume that \hyper@anchor set \anchor@spot.) 
  
  Uses
  \hyper@anchorstart{Doc-Start}\hyper@anchorend
  \hyper@anchorstart{\@currentHref}\hyper@anchorend in \hyper@refstepcounter and various environments
  inside the definition of \hyper@@anchor: \hyper@@anchor (defined in hyperref) is used in various environments where text is involved and typically in combination with nesting. (Very) old version of hyperref.dtx suggests that anchorcolor which does nothing now was involved here.
  \def\hypertarget#1#2{%
  \ifHy@nesting
    \hyper@@anchor{#1}{#2}%
  \else
    \hyper@@anchor{#1}{\relax}#2%
  \fi
}

  Internalcommand
   \new@pdflink #1
   defined in driver
   used only in driver
   argument is destination name, "Fit" argument is got through \@pdfview


%%%%
nesting

The documentation writes about "nesting":
Allows links to be nested; no drivers currently support this.

All places in the code where \ifHy@nesting is used are about anchors, not links.
\ifHy@nesting is false in all drivers, so actually nowhere the true code is used.

It is unclear if nesting can be used, and what use it would have. 