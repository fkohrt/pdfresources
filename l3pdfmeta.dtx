% \iffalse meta-comment
%
%% File: l3pdfmeta.dtx
%
% Copyright (C) 2018-2021 The LaTeX Project
%
% It may be distributed and/or modified under the conditions of the
% LaTeX Project Public License (LPPL), either version 1.3c of this
% license or (at your option) any later version.  The latest version
% of this license is in the file
%
%    http://www.latex-project.org/lppl.txt
%
% This file is part of the "PDFmanagement bundle (testphase)" (The Work in LPPL)
% and all files in that bundle must be distributed together.
%
% -----------------------------------------------------------------------
%
% The development version of the bundle can be found at
%
%    https://github.com/latex3/pdfresources
%
% for those people who are interested.
%
%<*driver>
\RequirePackage{pdfmanagement-testphase}
\DeclareDocumentMetadata{pdfstandard=A-2b}
\makeatletter
\declare@file@substitution{doc.sty}{doc-v3beta.sty}
\makeatother
\documentclass[full]{l3doc}
\usepackage{array,booktabs,hyperxmp}
\hypersetup{pdfauthor=The LaTeX Project,pdftitle=l3pdfmeta (PDFmanagement bundle (testphase))}

\begin{document}
  \DocInput{\jobname.dtx}
\end{document}
%</driver>
% \fi
%
% \title{^^A
%   The \pkg{l3pdfmeta} package\\ PDF standards  ^^A
%   \\
%   PDFmanagement bundle (testphase)
% }
%
% \author{^^A
%  The \LaTeX{} Project\thanks
%    {^^A
%      E-mail:
%        \href{mailto:latex-team@latex-project.org}
%          {latex-team@latex-project.org}^^A
%    }^^A
% }
%
% \date{Version 0.95a, released 2021-02-21}
%
% \maketitle
% \begin{documentation}
%
% \section{\pkg{l3pdfmeta} documentation}
% This module sets up some tools and commands needed
% for PDF standards in general.
% The goal is to collect the requirements and to provide code to check and fulfill them.
%
% In future is will probably also contain
% code to setup XMP-metadata.
% Until then XMP-metadata can be added by one of two mutual incompatible packages:
% \pkg{hyperxmp} and \pkg{pdfx}. Both
% packages aren't yet compatible with the new PDF management,
% but for \pkg{hyperxmp} some patches are provided, so the basic functions works.
%
% \begin{NOTE}{UF}
% This package should not replace both packages.
% Regarding XMP-metadata its goal
% is to create a skeleton metadata stream, add some core default values
% and to define interfaces that allows other packages
% to add data to this metadata and so to extend them.
% The problems to solve here are
% \begin{itemize}
% \item which tree structure is sensible
% \item how to escape if needed the input (or which tools are needed to allow
% the users to correctly escape their input)
% \item how interface to input data should look
% \end{itemize}
% \end{NOTE}
%
% \subsection{Verifying requirements of PDF standards}
%
% Standards like pdf/A set requirements on a PDF: Some things have be in the PDF,
% e.g. the catalog has to contain a /Lang entry and an colorprofile and
% an /OutputIntent, some other things are forbidden or restricted, e.g.
% the action dictionary of an annotation should not contain Javascript.
%
% The \pkg{l3pdfmeta} packages collects a number of relevant requirements,
% tries to enforce the ones which can be enforced and offers some tools
% for package authors to test if an action is allowed in the standard or not.
%
% This is work in progress and more tests will be added. But it should be noted
% that it will probably never be possible to prevent all forbidden actions
% or enforce all required ones or even to simply check all of them.
% The commands here don't replace a check with an external validator.
%
% Verifying against a PDF-standard involves two different task:
%
% \begin{itemize}
%   \item Check if you are allowed to ignore the requirement.
%   \item Decide which action to take if the answer to the first question is NO.
% \end{itemize}
%
% The following conditionals address the first task. Because of the second task
% a return value |FALSE| means that the standard requires you to do some
% special action. |TRUE| means that you can ignore this
% requirement.\footnote{One could also make the logic the
% other way round---there are arguments for both---but I had to decide.}
%
% In most cases it only matters if a requirement is in the standard,
% for example |Catalog_no_OCProperties| means \enquote{don't use |/OCProperties|
% in the catalog}. For a small number of requirements it is also needed to
% test a user value against a standard value. For example, |named_actions|
% restricts the allowed named actions in an annotation of subtype |/Named|,
% in this case it is needed to check not only if the requirement is
% in the standard but also if the user value is in the allowed list.
%
% \begin{function}[EXP,pTF]{\pdfmeta_standard_verify:n}
% \begin{syntax}
% \cs{pdfmeta_standard_verify:n}\Arg{requirement}
% \end{syntax}
%
% This checks if \meta{requirement} is listed in the standard.
% |FALSE| as result means that the requirement is in the standard and
% that probably some special action is required---which one depends
% on the requirement, see the descriptions below.
% |TRUE| means that the requirement is not there and so no special
% action is needed.
% This check can be used for simple requirements where neither
% a user nor a standard value is of importance.
% \end{function}
%
% \begin{function}[TF]{\pdfmeta_standard_verify:nn}
% \begin{syntax}
% \cs{pdfmeta_standard_verify:nn}\Arg{requirement}\Arg{value}
% \end{syntax}
%
% This checks if \meta{requirement} is listed in the standard,
% if yes it tries to find a predefined test handler for
% the requirement and passes \meta{value} and the value recorded
% in the standard to it. The handler returns |FALSE| if some special
% action is needed (e.g. if \meta{value} violates the rule)
% and |TRUE| if no special action is needed. If no handler exists
% this commands works like \cs{pdfmeta_standard_verify:n}.
% \end{function}
%
% In some cases one needs to query the value in the standard,
% e.g. to correct a wrong minimal PDF version you need to know
% which version is required by |min_pdf_version|.
%  For this two commands to access the value are provided:
%
% \begin{function}[EXP]{\pdfmeta_standard_item:n}
% \begin{syntax}
% \cs{pdfmeta_standard_item:n}\Arg{requirement}
% \end{syntax}
% This retrieves the value of \meta{requirement} and leaves it in the input.
% If the requirement isn't in the standard the result is empty,
% that means that requirements not in the standard and
% requirement without values can not be distinguished here.
% \end{function}
%
%
% \begin{function}{\pdfmeta_standard_get:nN}
% \begin{syntax}
% \cs{pdfmeta_standard_get:nN}\Arg{requirement} \meta{tl var}
% \end{syntax}
% This retrieves the value of \meta{requirement} and stores
% it in the \meta{token list variable}.
% If the \meta{requirement} is not found the special
% value |\q_no_value| is used.
% The  \meta{token list variable} is assigned locally.
% \end{function}
%
%
% The following describe the requirements which can be currently tested.
% Requirements with a value should use \cs{pdfmeta_standard_verify:nn}
% or \cs{pdfmeta_standard_verify:nnN} to test a local value against the standard.
% The rule numbers refer to \url{https://docs.verapdf.org/validation/pdfa-part1/}
%
% \subsubsection{Simple tests without handler}
%
% \begin{description}
%
% \item[|outputintent_A|] requires to embed a color profile and
%  reference it in a /Outputintent and that all output intents reference
%  the same colorprofile. The value stores the subtype.
%  {\em This requirement is detected and fulfilled by \pkg{l3pdfmeta} if the
%   provided interface in \cs{DeclareDocumentMetadata} is used, see below}.
%
% \item[|annot_flags|] in annotations the |Print| flag should be true,
%  |Hidden|, |Invisible|, |NoView| should be false.
%  {\em This requirement is detected  and set by \pkg{l3pdfmeta} for annotations
%  created with the \pkg{l3pdfannot}.
%  A new check is only needed if the flags are changed
%  or if links are created by other means.}
%
% \item[|no_encryption|] don't encrypt
% \item[|no_external_content|] no |/F|, |/FFilter|, or |/FDecodeParms|
%  in stream dictionaries
% \item[|no_embed_content|]    no |/EF| key in filespec, no |/Type/EmbeddedFiles|.
%  \emph{This will be checked in future by \pkg{l3pdffiles}
%  for the files it embeds.}
%  The restrictment is set for only PDF/A-1b.
%  PDF/A-2b and PDF/A3-b lifted this restriction: PDF/A-2b allows
%  to embed other PDF documents conforming to either PDF/A-1 or PDF/A-2,
%  and PDF/A-3 allows any embedded files. I don't see a way to test the
%  PDF/A-2b requirement so currently it will simply allow everything. Perhaps
%  a test for at least the PDF-format will be added in future.
% \item[|Catalog_no_OCProperties|] don't add |/OCProperties| to the catalog
% {\em l3pdfmeta removes this entry at the end of the document}
% \item[|annot_widget_no_AA|] (rule 6.6.2-1)
%  no AA dictionary in widget annotation,
%  this will e.g. be checked by the new hyperref driver.
% \item[|annot_widget_no_A_AA|] (rule 6.9-2)  no A and AA dictionary in widget.
% \item[|form_no_AA|] (6.9-3)  no /AA dictionary in form field
% \end{description}
%
% \subsubsection{Tests with values and special handlers}
%
% \begin{description}
%
% \item[|min_pdf_version|]  stores the minimal PDF version.
%  It should be checked against the current PDF version (\cs{pdf_version:}).
%  A failure means that the version should be changed.
%  This check is done by \pkg{l3pdfmeta} when the version is set with
%  \cs{DeclareDocumentMetadata} so more checks are only needed if the version is changed later.
%
% \item[|named_actions|]    this requirement restricts the list of
% allowed named actions to |NextPage|, |PrevPage|, |FirstPage|, |LastPage|.
% The check should supply the named action without slash
% (e.g. |View| (failure) or |NextPage| (pass)).
%
% \item[|annot_action_A|] (rule 6.6.1-1) this requirement restricts
%  the allowed subtypes of the
% |/A| dictionary of an action. The check should supply the user
%  subtype without slash e.g. as |GoTo| (pass) or |Movie| (failure).
% \end{description}
%
% \subsection{Colorprofiles and OutputIntent}
%
% The pdf/A standards require that a color profile is embedded and
% referenced in the catalog in the |/OutputIntent| array.
%
% The problem is that the pdf/A standards also require, that if the PDF has more then
% one entry in the |/OutputIntent| array (which is allowed), their |/DestOutputProfile|
% should all reference the same color profile\footnote{see rule 6.2.2-2 at
% \url{https://docs.verapdf.org/validation/pdfa-part1/}}.
%
% Enforcing this fully is impossible if entries are added manually by users or
% packages with |\pdfmanagement_add:nnn {Catalog}{OutputIntents}{|\meta{object reference}|}|
% as it is difficult to inspect and remove entries from the |/OutputIntent| array.
%
% So we provide a dedicated interface to avoid the need of manual
% user settings and allow the code to handle the requirements of the standard.
% The interface doesn't handle yet all finer points for PDF/X standards, e.g.
% named profiles, it is meant as a starting point to get at least PDF/A validation
% here.
%
% \begin{NOTE}{UF}
% The interface has to handle the following points:
% \begin{itemize}
% \item  We have to assume that some documents wants to add more
%  than one OutputIntent with varying subtypes.
% \item While currently only |/GTS_PDFA1| and |/GTS_PDFX| seem to
% be relevant, we have to assume that the list of subtypes is open.
% \item But we can imho assume that every subtype is there at most once.
% \item The referenced color profile can be used also other means, e.g. an /ICCBased
% color space. We must avoid that it is embedded twice in this case.
% This will need coordination with l3color. It should probably provide the
% code to embed the profile.
% \item While we can predeclare some standard icc-profiles, an interface to
% setup more is needed. This is currently not handled, as it needs
% coordination with a setup in l3color too.
% \item The implementation doesn't really handle yet all finer points for pdf/X
% see \url{tn0002_color_in_pdfa-1_2008-03-141.pdf}
% \end{itemize}
% \end{NOTE}
% The interface looks like this
%
% \begin{verbatim}
%  \DeclareDocumentMetadata
%    {
%      %other options for example pdfstandard
%       colorprofiles=
%        {
%          A = sRGB.icc, %or a or longer GTS_PDFA1 = sRGB.icc
%          X = FOGRA39L_coated.icc, % or x or longer GTS_PDFX
%          ISO_PDFE1 = whatever.icc
%        }
%
%    }
% \end{verbatim}
%
% |sRGB.icc| and |FOGRA39L_coated.icc| (from the \pkg{colorprofiles} package
% are predefined and will work directly\footnote{The \texttt{dvips} route
% will require that \texttt{ps2pdf} is called with \texttt{-dNOSAFER},
% and that the color profiles are in the current folder as \texttt{ps2pdf} doesn't
% use \texttt{kpathsea} to find them.}. |whatever.icc| will need special setup in
% the document preamble to declare the values for the
% |OutputIntent| dictionary, but the interface hasn't be added yet. This will be
% decided later.
%
%
% If an A-standard is detected or set which requires
% that all |/DestOutputProfile| reference the same
% color profile, the setting is changed to the equivalent of
%
% \begin{verbatim}
%  \DeclareDocumentMetadata
%    {
%      %other options
%       pdfstandard=A-2b,
%       colorprofiles=
%        {
%          A = sRGB.icc, %or longer GTS_PDFA1 = sRGB.icc
%          X = sRGB.icc,
%          ISO_PDFE1 = sRGB.icc
%        }
%
%    }
% \end{verbatim}
%
% The pdf/A standards will use |A=sRGB.icc| by default, so this doesn't
% need to be declared explicitly.
%
%
% \end{documentation}
%
% \begin{implementation}
%
% \section{\pkg{l3pdfmeta} implementation}
%
%    \begin{macrocode}
%<@@=pdfmeta>
%<*header>
\ProvidesExplPackage {l3pdfmeta} {2021-02-22} {0.95a}
  {PDF-Standards---PDFmanagement bundle (testphase)}
%</header>
%    \end{macrocode}
% Message for unknown standards
%    \begin{macrocode}
%<*package>
\msg_new:nnn  {pdf }{unknown-standard}{The~standard~'#1'~is~unknown~and~has~been~ignored}
%    \end{macrocode}
% \begin{variable}{\l_@@_tmpa_tl,\l_@@_tmpb_tl,\l_@@_tmpa_str}
%    \begin{macrocode}
\tl_new:N\l_@@_tmpa_tl
\tl_new:N\l_@@_tmpb_tl
\str_new:N \l_@@_tmpa_str
%    \end{macrocode}
% \end{variable}
% \subsection{Standards (work in progress)}
% \subsubsection{Tools and tests}
% This internal property will contain for now the settings for the document.
% \begin{variable}{\g_@@_standard_prop}
%    \begin{macrocode}
\prop_new:N \g_@@_standard_prop
%    \end{macrocode}
% \end{variable}
% \subsubsection{Functions to check a requirement}
% At first two commands to get the standard value if needed:
% \begin{macro}[EXP]{\pdfmeta_standard_item:n}
%    \begin{macrocode}
\cs_new:Npn \pdfmeta_standard_item:n #1
 {
   \prop_item:Nn \g_@@_standard_prop {#1}
 }
%    \end{macrocode}
% \end{macro}
% \begin{macro}{\pdfmeta_standard_get:nN}
%    \begin{macrocode}
\cs_new_protected:Npn \pdfmeta_standard_get:nN #1 #2
 {
   \prop_get:NnN \g_@@_standard_prop {#1} #2
 }
%    \end{macrocode}
% \end{macro}
% Now two functions to check the requirement. A simple and one value/handler based.
% \begin{macro}[pTF]{\pdfmeta_standard_verify:n}
%  This is a simple test is the requirement is in the prop.
%    \begin{macrocode}
\prg_new_conditional:Npnn \pdfmeta_standard_verify:n #1 {T,F,TF}
  {
     \prop_if_in:NnTF \g_@@_standard_prop {#1}
       {
         \prg_return_false:
       }
       {
         \prg_return_true:
       }
  }
%    \end{macrocode}
% \end{macro}
% \begin{macro}[TF]{\pdfmeta_standard_verify:nn}
% This allows to test against a user value. It calls a test handler if this
% exists and passes the user and the standard value to it. The test
% handler should return true or false.
%    \begin{macrocode}
\prg_new_protected_conditional:Npnn \pdfmeta_standard_verify:nn #1 #2  {T,F,TF}
  {
    \prop_if_in:NnTF \g_@@_standard_prop {#1}
      {
        \cs_if_exist:cTF {@@_standard_verify_handler_#1:nn}
          {
            \exp_args:Nnnx
            \use:c
              {@@_standard_verify_handler_#1:nn}
              { #2 }
              { \prop_item:Nn \g_@@_standard_prop {#1} }
          }
          {
            \prg_return_false:
          }
      }
      {
        \prg_return_true:
      }
   }
%    \end{macrocode}
% \end{macro}
%
% Now we setup a number of handlers.
%
% The first actually ignores the user values and tests against the
% current pdf version. If this is smaller than the minimum we report a failure.
% |#1| is the user value, |#2| the reference value from the standard.
% \begin{macro}{\@@_standard_verify_handler_min_pdf_version:nn}
%    \begin{macrocode}
%
\cs_new_protected:Npn \@@_standard_verify_handler_min_pdf_version:nn #1 #2
 {
   \pdf_version_compare:NnTF <
     { #2 }
     {\prg_return_false:}
     {\prg_return_true:}
 }
%    \end{macrocode}
% \end{macro}
% The next checks if the user value is in the list and returns a failure if not.
% \begin{macro}{\@@_standard_verify_handler_named_actions:nn}
%    \begin{macrocode}

\cs_new_protected:Npn \@@_standard_verify_handler_named_actions:nn #1 #2
 {
   \tl_if_in:nnTF { #2 }{ #1 }
     {\prg_return_true:}
     {\prg_return_false:}
 }
%    \end{macrocode}
% \end{macro}
% The next checks if the user value is in the list and returns a failure if not.
% \begin{macro}{\@@_standard_verify_handler_annot_action_A:nn}
%    \begin{macrocode}
\cs_new_protected:Npn \@@_standard_verify_handler_annot_action_A:nn #1 #2
 {
   \tl_if_in:nnTF { #2 }{ #1 }
     {\prg_return_true:}
     {\prg_return_false:}
 }
%    \end{macrocode}
% \end{macro}
% This check is probably not needed, but for completeness
% \begin{macro}{\@@_standard_verify_handler_outputintent_subtype:nn}
%    \begin{macrocode}
\cs_new_protected:Npn \@@_standard_verify_handler_outputintent_subtype:nn #1 #2
 {
   \tl_if_eq:nnTF { #2 }{ #1 }
     {\prg_return_true:}
     {\prg_return_false:}
 }
%    \end{macrocode}
% \end{macro}
% \subsubsection{Enforcing requirements}
%  A number of requirements can sensibly be enforced by us.
%  \paragraph{Annot flags}
% pdf/A require a number of settings here, we store them in a command which
% can be added to the property of the standard:
%    \begin{macrocode}
\cs_new_protected:Npn \@@_verify_pdfa_annot_flags:
  {
    \bitset_set_true:Nn  \l_pdfannot_F_bitset {Print}
    \bitset_set_false:Nn \l_pdfannot_F_bitset {Hidden}
    \bitset_set_false:Nn \l_pdfannot_F_bitset {Invisible}
    \bitset_set_false:Nn \l_pdfannot_F_bitset {NoView}
    \pdfannot_dict_put:nnn {link/URI}{F}{ \bitset_to_arabic:N \l_pdfannot_F_bitset }
    \pdfannot_dict_put:nnn {link/GoTo}{F}{ \bitset_to_arabic:N \l_pdfannot_F_bitset }
    \pdfannot_dict_put:nnn {link/GoToR}{F}{ \bitset_to_arabic:N \l_pdfannot_F_bitset }
    \pdfannot_dict_put:nnn {link/Launch}{F}{ \bitset_to_arabic:N \l_pdfannot_F_bitset }
    \pdfannot_dict_put:nnn {link/Named}{F}{ \bitset_to_arabic:N \l_pdfannot_F_bitset }
  }
%    \end{macrocode}
% At begin document this should be checked:
%    \begin{macrocode}
\hook_gput_code:nnn {begindocument} {pdf}
  {
    \pdfmeta_standard_verify:nF { annot_flags }
     { \@@_verify_pdfa_annot_flags: }
  }
%    \end{macrocode}
%
%  \subsubsection{pdf/A}
%  We use global properties so that follow up standards can be
%  copied and then adjusted.
%  Some note about requirements for more standard can
%  be found in info/pdfstandard.tex.
%  \begin{variable}{
%   \g_@@_standard_pdf/A-1B_prop ,
%   \g_@@_standard_pdf/A-2B_prop ,
%   \g_@@_standard_pdf/A-3B_prop
%   }
%     \begin{macrocode}
\prop_new:c { g_@@_standard_pdf/A-1B_prop }
\prop_set_from_keyval:cn { g_@@_standard_pdf/A-1B_prop }
  {
    ,name             = pdf/A-1B
    ,type             = A
    ,year             = 2005
    ,min_pdf_version  = 1.4        %minimum
    ,no_encryption    =
    ,no_external_content =  % no F, FFilter, or FDecodeParms in stream dicts
    ,no_embed_content = % no EF key in filespec, no /Type/EmbeddedFiles
    ,max_string_size  = 65535
    ,max_array_size   = 8191
    ,max_dict_size    = 4095
    ,max_obj_num      = 8388607
    ,max_nest_qQ      = 28
    ,named_actions    = {NextPage, PrevPage, FirstPage, LastPage}
    ,annot_flags      =
    %booleans. Only the existence of the key matter.
    %If the entry is added it means a requirements is there
    %(in most cases "don't use ...")
    %
    %===============
    % Rule 6.1.13-1 CosDocument, isOptionalContentPresent == false
      ,Catalog_no_OCProperties =
    %===============
    % Rule 6.6.1-1: PDAction, S == "GoTo" || S == "GoToR" || S == "Thread"
    %               || S == "URI" || S == "Named" || S == "SubmitForm"
    % means: no /S/Launch, /S/Sound, /S/Movie, /S/ResetForm, /S/ImportData,
    %        /S/JavaScript, /S/Hide
      ,annot_action_A        = {GoTo,GoToR,Thread,URI,Named,SubmitForm}
    %===============
    % Rule 6.6.2-1: PDAnnot, Subtype != "Widget" || AA_size == 0
    % means: no AA dictionary
      ,annot_widget_no_AA      =
    %===============
    % Rule 6.9-2: PDAnnot, Subtype != "Widget" || (A_size == 0 && AA_size == 0)
    % (looks like a tightening of the previous rule)
      ,annot_widget_no_A_AA    =
    %===============
    % Rule 6.9-1 PDAcroForm, NeedAppearances == null || NeedAppearances == false
    ,form_no_NeedAppearances =
    %===============
    %Rule 6.9-3 PDFormField, AA_size == 0
    ,form_no_AA              =
    %===============
    % to be continued https://docs.verapdf.org/validation/pdfa-part1/
    % - Outputintent/colorprofiles requirements
    % an outputintent should be loaded and is unique.
    ,outputintent_A         = {GTS_PDFA1}
    % - no Alternates key in image dictionaries
    % - no OPI, Ref, Subtype2 with PS key in xobjects
    % - Interpolate  = false in images
    % - no TR, TR2 in ExtGstate
  }

%A-2b ==============
\prop_new:c { g_@@_standard_pdf/A-2B_prop }
\prop_gset_eq:cc
  { g_@@_standard_pdf/A-2B_prop }
  { g_@@_standard_pdf/A-1B_prop }
\prop_gput:cnn
  { g_@@_standard_pdf/A-2B_prop }{name}{pdf/A-2B}
\prop_gput:cnn
  { g_@@_standard_pdf/A-2B_prop }{year}{2011}
% embedding files is allowed (with restrictions)
\prop_gremove:cn
  { g_@@_standard_pdf/A-2B_prop }
  { embed_content}

%A-3b ==============
\prop_new:c { g_@@_standard_pdf/A-3B_prop }
\prop_gset_eq:cc
  { g_@@_standard_pdf/A-3B_prop }
  { g_@@_standard_pdf/A-2B_prop }
\prop_gput:cnn
  { g_@@_standard_pdf/A-3B_prop }{name}{pdf/A-3B}
\prop_gput:cnn
  { g_@@_standard_pdf/A-2B_prop }{year}{2012}
% embedding files is allowed (with restrictions)
\prop_gremove:cn
  { g_@@_standard_pdf/A-3B_prop }
  { embed_content}
%    \end{macrocode}
% \end{variable}
%
% \subsubsection{Colorprofiles and Outputintents}
% The following provides a minimum of interface to add a color profile
% and an outputintent need for PDF/A for now. There will be need to extend it later,
% so we try for enough generality.
%
% Adding a profile and an intent is technically easy:
% \begin{enumerate}
% \item Embed the profile as stream with
% \begin{verbatim}
%  \pdf_object_unnamed_write:nn{fstream} {{/N~4}{XXX.icc}}
% \end{verbatim}
% \item Write a |/OutputIntent| dictionary for this
% \begin{verbatim}
% \pdf_object_unnamed_write:nx {dict}
%  {
%    /Type /OutputIntent
%    /S /GTS_PDFA1  % or GTS_PDFX or ISO_PDFE1 or ...
%    /DestOutputProfile \pdf_object_ref_last: % ref the color profile
%    /OutputConditionIdentifier ...
%    ... %more info
%  }
% \end{verbatim}
% \item Reference the dictionary in the catalog:
% \begin{verbatim}
% \pdfmanagement_add:nnx {Catalog}{OutputIntents}{\pdf_object_ref_last:}
% \end{verbatim}
% \end{enumerate}
% But we need to do a bit more work, to get the interface right.
% The object for the profile should be named, to allow l3color to reuse it
% if needed. And we need container to store the profiles, to handle the
% standard requirements.
%
% \begin{variable}{\g_@@_outputintents_prop}
% This variable will hold the profiles for the subtypes. We assume
% that every subtype has only only color profile.
%    \begin{macrocode}
\prop_new:N \g_@@_outputintents_prop
%    \end{macrocode}
% \end{variable}
% Some keys to fill the property.
%    \begin{macrocode}
\keys_define:nn { document / metadata }
  {
    colorprofiles .code:n =
     {
       \keys_set:nn { document / metadata / colorprofiles }{#1}
     }
  }
\keys_define:nn { document / metadata / colorprofiles }
 {
   ,A .code:n =
      {
        \tl_if_blank:nF {#1}
          {
            \prop_gput:Nnn \g_@@_outputintents_prop
             { GTS_PDFA1  } {#1}
          }
      }
   ,a .code:n =
      {
        \tl_if_blank:nF {#1}
          {
            \prop_gput:Nnn \g_@@_outputintents_prop
              { GTS_PDFA1  } {#1}
          }
      }
   ,X .code:n =
      {
        \tl_if_blank:nF {#1}
          {
             \prop_gput:Nnn \g_@@_outputintents_prop
              { GTS_PDFX  } {#1}
          }
      }
   ,x .code:n =
      {
        \tl_if_blank:nF {#1}
          {
            \prop_gput:Nnn \g_@@_outputintents_prop
              { GTS_PDFX  } {#1}
          }
      }
   ,unknown .code:n =
     {
       \tl_if_blank:nF {#1}
          {
           \exp_args:NNo
            \prop_gput:Nnn \g_@@_outputintents_prop
              { \l_keys_key_str  } {#1}
          }
     }
 }
%    \end{macrocode}
% At first we setup our two default profiles. This is internal as
% the public interface is still undecided.
%    \begin{macrocode}
\pdfdict_new:n   {l_pdfmeta/outputintent}
\pdfdict_put:nnn {l_pdfmeta/outputintent}
  {Type}{/OutputIntent}
\prop_const_from_keyval:cn { c_@@_colorprofile_sRGB.icc}
  {
    ,OutputConditionIdentifier=IEC~sRGB
    ,Info=IEC~61966-2.1~Default~RGB~colour~space~-~sRGB
    ,RegistryName=http://www.iec.ch
    ,N = 3
  }
\prop_const_from_keyval:cn { c_@@_colorprofile_FOGRA39L_coated.icc}
  {
    ,OutputConditionIdentifier=FOGRA39L~Coated
    ,Info={Offset~printing,~according~to~ISO~12647-2:2004/Amd~1,~OFCOM,~ %
           paper~type~1~or~2~=~coated~art,~115~g/m2,~tone~value~increase~
           curves~A~(CMY)~and~B~(K)}
    ,RegistryName=http://www.fogra.org
    ,N = 4
  }
%    \end{macrocode}
% \begin{macro}{\@@_embed_colorprofile:n,\@@_write_outputintent:nn}
% The commands embed the profile, and write the dictionary and add it to
% the catalog. The first command should perhaps be moved to l3color
% as it needs such profiles too. We used named objects so that we can
% check if the profile is already there. This is not full proof if pathes are
% used.
%    \begin{macrocode}
\cs_new_protected:Npn \@@_embed_colorprofile:n #1%#1 file name
  {
    \pdf_object_if_exist:nF {@@_colorprofile_#1}
      {
        \pdf_object_new:nn  {@@_colorprofile_#1}{fstream}
        \pdf_object_write:nx {@@_colorprofile_#1}
         {
           {/N\c_space_tl
             \prop_item:cn{c_@@_colorprofile_#1}{N}
           }
           {#1}
         }
      }
  }

\cs_new_protected:Npn \@@_write_outputintent:nn #1 #2 %#1 file name, #2 subtype
  {
    \group_begin:
     \pdfdict_put:nnx {l_pdfmeta/outputintent}{S}{/\str_convert_pdfname:n{#2}}
     \pdfdict_put:nnx {l_pdfmeta/outputintent}
       {DestOutputProfile}
       {\pdf_object_ref:n{@@_colorprofile_#1}}
     \clist_map_inline:nn { OutputConditionIdentifier, Info, RegistryName }
       {
         \prop_get:cnNT
          { c_@@_colorprofile_#1}
          { ##1 }
          \l_@@_tmpa_tl
          {
            \pdf_string_from_unicode:nVN {utf8/string}\l_@@_tmpa_tl\l_@@_tmpa_str
            \pdfdict_put:nnx
              {l_pdfmeta/outputintent}{##1}{\l_@@_tmpa_str}
          }
       }
     \pdf_object_unnamed_write:nx {dict}{\pdfdict_use:n {l_pdfmeta/outputintent} }
     \pdfmanagement_add:nnx {Catalog}{OutputIntents}{\pdf_object_ref_last:}
    \group_end:
  }
%    \end{macrocode}
% \end{macro}
% Now the verifying code.
% If no requirement is set we simply loop over the property
%    \begin{macrocode}

\AddToHook{begindocument/end}
  {
    \pdfmeta_standard_verify:nTF {outputintent_A}
      {
         \prop_map_inline:Nn \g_@@_outputintents_prop
           {
             \@@_embed_colorprofile:n
               {#2}
             \@@_write_outputintent:nn
               {#2}
               {#1}
           }
      }
%    \end{macrocode}
% If an output intent is required for pdf/A we need to ensure, that the key of
% default subtype has a value, as default we take sRGB.icc.
% Then we loop but take always the same profile.
%    \begin{macrocode}
      {
         \exp_args:NNx
         \prop_if_in:NnF
           \g_@@_outputintents_prop
           { \pdfmeta_standard_item:n { outputintent_A } }
           {
             \exp_args:NNx
             \prop_gput:Nnn
               \g_@@_outputintents_prop
               { \pdfmeta_standard_item:n { outputintent_A } }
               { sRGB.icc }
           }
         \exp_args:NNx
         \prop_get:NnN
           \g_@@_outputintents_prop
           { \pdfmeta_standard_item:n { outputintent_A } }
           \l_@@_tmpb_tl
         \exp_args:NV \@@_embed_colorprofile:n \l_@@_tmpb_tl
         \prop_map_inline:Nn \g_@@_outputintents_prop
           {
             \exp_args:NV
             \@@_write_outputintent:nn
               \l_@@_tmpb_tl
               { #1 }
           }
       }
   }
%    \end{macrocode}
%
%    \begin{macrocode}
%</package>
%    \end{macrocode}
%
% \end{implementation}
%
% \PrintIndex
