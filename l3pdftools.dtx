% \iffalse meta-comment
%
%% File: l3pdftools.dtx
%
% Copyright (C) 2018-2020 The LaTeX3 Project
%
% It may be distributed and/or modified under the conditions of the
% LaTeX Project Public License (LPPL), either version 1.3c of this
% license or (at your option) any later version.  The latest version
% of this license is in the file
%
%    http://www.latex-project.org/lppl.txt
%
% This file is part of the "(experimental) pdfmanagement bundle" (The Work in LPPL)
% and all files in that bundle must be distributed together.
%
% -----------------------------------------------------------------------
%
% The development version of the bundle can be found at
%
%    https://github.com/latex3/pdfresources
%
% for those people who are interested.
%
%<*driver>
\RequirePackage{expl3}
\documentclass[full]{l3doc}
\providecommand\potentialclash{\noindent\llap{\dbend\ }}
\begin{document}
  \DocInput{\jobname.dtx}
\end{document}
%</driver>
% \fi
%
% \title{^^A
%   The \pkg{l3pdftools} package\\ collection of pdf related commands ^^A
% }
%
% \author{^^A
%  The \LaTeX3 Project\thanks
%    {^^A
%      E-mail:
%        \href{mailto:latex-team@latex-project.org}
%          {latex-team@latex-project.org}^^A
%    }^^A
% }
%
% \date{Released XXXX-XX-XX}
%
% \maketitle
% \begin{documentation}
%
% \section{\pkg{l3pdftools} documentation}
%
% This module collects a number of commands of the tool type which haven't found
% another home yet ... Probably this module will disappear again at some time.
%
% \begin{function}[EXP,added=2020-07-04]
%   {\pdf_text_convert:nn}
%  \begin{syntax}
%   \cs{pdf_text_convert:nn}  \Arg{format}  \Arg{content}
%  \end{syntax}
%  This converts \Arg{content} following the rules defined by \Arg{format}.
%  Non-ascii input should be utf8 encoded.
%  Currently the following formats exist:
%  \begin{description}
%     \item[name] This will first expand the content with \cs{text_expand:n} and
%      then escape it in the way needed in a PDF Name with
%      \cs{str_convert_pdfname:e}.
%     \item[name-print] This does the same as the \texttt{name} format, but also
%     adds a slash before, so |\pdf_text_convert:nn{name-print}{abc}| will
%     output \texttt{/abc}.
%   \end{description}
% \end{function}
% \begin{function}[added=2020-07-04]
%   {\pdf_text_convert:nnN}
%  \begin{syntax}
%   \cs{pdf_text_convert:nnN}  \Arg{format}  \Arg{content} \Arg{tlvar}
%  \end{syntax}
%  This converts \Arg{content} following the rules defined by \Arg{format} and stores
%  the result in \Arg{tlvar}. The assignment is done locally.
%  Non-ascii input should be utf8 encoded.
%  Currently the following formats exist:
%  \begin{description}
%     \item[name] see above
%     \item[name-print] see above
%     \item[utf8/string]
%      this converts with \cs{str_set_convert:Nnnn} into utf8/string.
%     \item[utf8/string-print]
%      this converts into utf8/string and add parentheses around.%
%     \item[utf8/URIpdf]
%      this converts with \cs{str_set_convert:Nnnn} into utf8/url and
%      then replace reserved and digits back from the percent encoding. Parentheses
%      are escaped.
%     \item[utf8/url-print]
%      this converts into utf8/URIpdf  and add parentheses around.%
%     \item[utf16/string]
%      this converts with \cs{str_set_convert:Nnnn} into utf16/string.
%     \item[utf16/string-print]
%      this converts into utf16/string and add parentheses around.
%     \item[utf16/hex]
%      this converts into utf16/hex
%     \item[utf16/hex-print]
%     this converts into utf16/hex and add bracket around.
%   \end{description}
% \end{function}
%
% \subsection{BDC operator / Properties resource}
% \begin{NOTE}{UF}
%  we need a switch for the case that the resource should be added to
%  xform resource instead of a page resources, see pdfbase.sty
% - xdvipdfmx: looks fine, the resource is added to the xform resource automatically
% - pdftex should now work okay too
% \end{NOTE}
% Entries to the /Properties dictionary in the page resources can
% be added with dvips only through side-effects: if a BDC-mark is created
% dvips/ghostscript will automatically create the necessary objects and names.
% To get a sensible abstraction the code does the same for the other backends if the
% core management code has been activated. This means that the behaviour
% of the command is different then. The \cs{pdf_bdc_obj:..} should only be used
% if the management is active.
% \begin{function}[updated = 2020-07-03]
%   {
%     \pdf_bdc:nn
%   }
%   \begin{syntax}
%     \cs{pdf_bdc:nn} \Arg{tag} \Arg{dictionary content}
%   \end{syntax}
%    This command adds a BDC marked content operator to the current page stream.
%   \meta{tag} is the tag of this operator (without the leading slash),
%   \meta{dictionary content} is the content of the second argument.
%   If the PDF resource management is active an dictionary object with the
%   content is created and referenced with a name in the BDC operator.
%   Without the resource management the content is used directly. It then
%   depends on the backend how it is handled: with dvips a name is used
%   while the pdfmode engines and dvipdfmx write the content into the stream.
%
% \end{function}
% \begin{function}[added = 2020-07-03]
%   {
%     \pdf_bdc_obj:nn
%   }
%   \begin{syntax}
%     \cs{pdf_bdc_obj:nn} \Arg{tag} \Arg{object name}
%   \end{syntax}
%   This command adds a BDC marked content operator to the current page stream.
%   \meta{tag} is the tag of this operator (without the leading slash),
%   \meta{object name}  is a the name of an dictionary object reserved with
%   \cs{pdf_object_new:nn} and filled with \cs{pdf_object_write:n} with
%   the properties of the BDC. Reusing a predefined object can save space
%   but the command works correctly
%   only if the resources management has been activated and should be used only
%   if this can be ensured.
% \end{function}
% \begin{function}[updated = 2020-07-03]
%   {
%     \pdf_bdc_obj:n
%   }
%   \begin{syntax}
%     \cs{pdf_bdc_obj:n} \Arg{tag}
%   \end{syntax}
%    This command adds a BDC marked content operator to the current page stream.
%   \meta{tag} is the tag of this operator (without the leading slash).
%   As object this commands uses the last anonymous dictionary object created with
%   \cs{pdf_object_now:nn}. It lies in the responsibility of the user that the last
%   object is the wanted one. Like with \cs{pdf_bdc_obj:nn} the command works correctly
%   only if the resources management has been activated and should be used only
%   if this can be ensured.
% \end{function}
% \begin{function}[added = 2019-10-17]
%   {
%     \pdf_bmc:n
%   }
%   \begin{syntax}
%     \cs{pdf_bmc:n} \Arg{tag}
%   \end{syntax}
%    This command created a BMC marked content operator. The argument is the
%    tag without the leading slash. It can be e.g. used for simple artifact
%    markers.
% \end{function}
% \begin{function}[added = 2019-06-30]
%   {
%     \pdf_emc:
%   }
%   \begin{syntax}
%     \cs{pdf_emc:}
%   \end{syntax}
%    This command closes the  BDC marked content operator opened with \cs{pdf_bdc:nn}.
%    It should be on the same page as the bdc-command.
%
%   \begin{verbatim}
%   \pdf_object_new:nn   {objA}{dict}
%   \pdf_object_write:nn {objA}{/Type/Artifact}
%   \pdf_bdc:nn {Span}{objA}
%   text
%   \pdf_emc:
%   \end{verbatim}
% \end{function}
%
%\subsection{references to page objects}
% \begin{NOTE}{UF}
% a reference to the current page would be useful, but difficult to
% abstract in a sensible way. With xetex it is simply @thispage, but the other
% engines are unclear.
% \end{NOTE}
% \begin{function}[EXP, added = 2019-08-18]
%   {\pdf_object_pageref:n}
%   \begin{syntax}
%     \cs{pdf_object_pageref:n} \Arg{number}
%   \end{syntax}
%   This returns the object reference of a page object. The \Arg{number} is
%   the absolute page number. The count starts with one.
% \end{function}
%
% \end{documentation}
%
% \begin{implementation}
%
% \section{\pkg{l3pdftools} implementation}
%
%    \begin{macrocode}
%<*package>
\ProvidesExplPackage {l3pdftools} {2021-01-09} {0.2}
  {various PDF tools}
%    \end{macrocode}
%
%    \begin{macrocode}
%<@@=pdf>
%    \end{macrocode}
% \subsection{Conversions and export functions}
% \begin{macro}{\pdf_text_convert:nn,\pdf_text_convert:nV}
% \begin{macro}{\pdf_text_convert:nnN}
% \begin{macro}{\@@_text_convert_name:n}
% \begin{macro}{\@@_text_convert_name:nN}
% \begin{macro}{\@@_text_convert_name-print:n}
% \begin{macro}{\@@_text_convert_name-print:n}
%    \begin{macrocode}

\cs_new:Npn \pdf_text_convert:nn #1 #2
  {
    \cs_if_exist_use:cF { @@_text_convert_#1:n }
      {
        \__kernel_msg_error:nnn { pdf } { unknown-exp-convert } {#1}
        \use_none:n
      }
    { #2 }
  }
\cs_generate_variant:Nn \pdf_text_convert:nn {nV}

\cs_new:Npn \pdf_text_convert:nnN #1 #2 #3
  {
    \cs_if_exist_use:cF { @@_text_convert_#1:nN }
      {
        \__kernel_msg_error:nnn { pdf } { unknown-convert } {#1}
        \use_none:nn
      }
    { #2 } #3
  }

\cs_generate_variant:Nn \pdf_text_convert:nnN {nVN}
%% tool command escape name
%% commands in the argument are expanded if possible
%% with \text_expand:n
%% name: only name,
%% name_print: with / before, see also string naming

\cs_generate_variant:Nn \str_convert_pdfname:n { e }

\cs_new:Npn \@@_text_convert_name:n #1
  {
    \str_convert_pdfname:e { \text_expand:n { #1 } }
  }

\cs_new_protected:Npn \@@_text_convert_name:nN #1 #2
  {
    \tl_set:Nx #2 {\@@_text_convert_name:n {#1} }
  }

\cs_new:cpn { @@_text_convert_name-print:n } #1
  {
    / \str_convert_pdfname:e { \text_expand:n { #1 } }
  }

\cs_new_protected:cpn { @@_text_convert_name-print:nN } #1 #2
  {
    \tl_set:Nx #2 {\use:c { @@_text_convert_name-print:n } {#1} }
  }

%    \end{macrocode}
% \end{macro}
% \end{macro}
% \end{macro}
% \end{macro}
% \end{macro}
% \end{macro}
%
% The convert command must use a different value the source encoding
% depending on the engines. Until the PR in str-convert is active we add the alias here
% too
%    \begin{macrocode}
\bool_lazy_any:nTF
  {
    \sys_if_engine_luatex_p:
    \sys_if_engine_xetex_p:
  }
  {
    \prop_gput:Nnn \g__str_alias_prop { default } {  }
  }
  {
    \prop_gput:Nnn \g__str_alias_prop { default } { utf8 }
  }
%    \end{macrocode}
% Most converter are simply wrapper around the str-convert commands and so
% use the same name.
% The exception is the one for url's: it reverts most of the percent encodings
% and escapes the parentheses.
% That's why its name is URIpdf instead of url. The current code is probably quite
% slow and will need a replacement.
% \begin{macro}{ @@_text_convert_utf8/string:nN }
% \begin{macro}{ @@_text_convert_utf8/string-print:nN }
% \begin{macro}{ @@_text_convert_utf8/URIpdf:nN }
% \begin{macro}{ @@_text_convert_utf8/URIpdf-print:nN }
% \begin{macro}{ @@_text_convert_utf16/string:nN }
% \begin{macro}{ @@_text_convert_utf16/string-print:nN }
% \begin{macro}{ @@_text_convert_utf16/hex:nN }
% \begin{macro}{ @@_text_convert_utf16/hex-print:nN }
%    \begin{macrocode}
%% TODO Names need a review when it is clear which converters
%% are actually needed
%% string conversions and printing
%% we assume here that the text purify step has been done. The input is
%% a list of (utf8) chars.
%% str convert, not expandable.
%  filespec (attachment view)  tests:
%  utf8:  gr\303\274\303\237e.txt                    %doesn't work, umlaut wrong,
%  utf8 with BOM \357\273\277gr\303\274\303\237e.txt %doesn't work, umlaut wrong, bom visible
%  utf16 with BE: (FEFF)  \376\377\000g\000r\000\374\000\337\000e\000.\000t\000x\000t %works
%                xetex converts to <feff0067007200fc00df0065002e007400780074>
%  utf16 with BE / HEX: <FEFF0067007200FC00DF0065002E007400780074> works

% bookmarks: as pdfoutline uses () currently only utf16 with BE is usable.
% check if one can use HEX too when directly writing the object
% ==========
% uri: utf16BE/string seems not to work, hex neither
%      utf8/string works but not on macos,
%      so a specfic utf8/url variant is needed
% ==========
% "input" is utf8 for pdftex, empty (native) for unicode engine
% commands to output literal strings  (...)

\cs_new_protected:cpn { @@_text_convert_utf8/string:nN }  #1 #2
  {
     \str_set_convert:Nnnn #2
        { #1 }
        { default }
        {utf8/string}
  }

\cs_new_protected:cpn { @@_text_convert_utf8/string-print:nN } #1 #2
  {
    \use:c { @@_text_convert_utf8/string:nN } { #1 } #2
    \str_put_left:Nn #2  {(}
    \str_put_right:Nn #2 {)}
  }
% special url command:
\cs_new_protected:cpx { @@_text_convert_utf8/URIpdf:nN }  #1 #2
  {
     \exp_not:N \str_set_convert:Nnnn #2
       { #1 }
       { default }
       {utf8/url}
     \exp_not:N \str_replace_all:Nnn #2 {\c_percent_str 3A} {:}
     \exp_not:N \str_replace_all:Nnn #2 {\c_percent_str 2F} {/}
     \exp_not:N \str_replace_all:Nnn #2 {\c_percent_str 23} {\c_hash_str}
     \exp_not:N \str_replace_all:Nnn #2 {\c_percent_str 5B} {[}
     \exp_not:N \str_replace_all:Nnn #2 {\c_percent_str 5D} {]}
     \exp_not:N \str_replace_all:Nnn #2 {\c_percent_str 40} {\c_atsign_str}
     \exp_not:N \str_replace_all:Nnn #2 {\c_percent_str 21} {!}
     \exp_not:N \str_replace_all:Nnn #2 {\c_percent_str 24} {\c_dollar_str}
     \exp_not:N \str_replace_all:Nnn #2 {\c_percent_str 26} {\c_ampersand_str}
     \exp_not:N \str_replace_all:Nnn #2 {\c_percent_str 27} {'}
     \exp_not:N \str_replace_all:Nnn #2 {\c_percent_str 2A} {*}
     \exp_not:N \str_replace_all:Nnn #2 {\c_percent_str 2B} {+}
     \exp_not:N \str_replace_all:Nnn #2 {\c_percent_str 2C} {,}
     \exp_not:N \str_replace_all:Nnn #2 {\c_percent_str 3B} {;}
     \exp_not:N \str_replace_all:Nnn #2 {\c_percent_str 3D} {=}
     \exp_not:N \str_replace_all:Nnn #2 {\c_percent_str 30} {0}
     \exp_not:N \str_replace_all:Nnn #2 {\c_percent_str 31} {1}
     \exp_not:N \str_replace_all:Nnn #2 {\c_percent_str 32} {2}
     \exp_not:N \str_replace_all:Nnn #2 {\c_percent_str 33} {3}
     \exp_not:N \str_replace_all:Nnn #2 {\c_percent_str 34} {4}
     \exp_not:N \str_replace_all:Nnn #2 {\c_percent_str 35} {5}
     \exp_not:N \str_replace_all:Nnn #2 {\c_percent_str 36} {6}
     \exp_not:N \str_replace_all:Nnn #2 {\c_percent_str 37} {7}
     \exp_not:N \str_replace_all:Nnn #2 {\c_percent_str 38} {8}
     \exp_not:N \str_replace_all:Nnn #2 {\c_percent_str 39} {9}
     \exp_not:N \str_replace_all:Nnn #2 {\c_percent_str 28} {\c_backslash_str(}
     \exp_not:N \str_replace_all:Nnn #2 {\c_percent_str 29} {\c_backslash_str)}
  }

\cs_new_protected:cpn { @@_text_convert_utf8/URIpdf-print:nN } #1 #2
  {
     \use:c { @@_text_convert_utf8/URIpdf:nN } {#1} #2
     \str_put_left:Nn #2  {(}
     \str_put_right:Nn #2 {)}
  }
% with utf16 with BE marker
\cs_new_protected:cpn { @@_text_convert_utf16/string:nN }  #1 #2
  {
    \str_set_convert:Nnnn #2
      { #1 }
      { default }
      {utf16/string}
  }

\cs_new_protected:cpn { @@_text_convert_utf16/string-print:nN } #1 #2
  {
    \use:c { @@_text_convert_utf16/string:nN } {#1} #2
    \str_put_left:Nn #2  {(}
    \str_put_right:Nn #2 {)}
  }

% commands to output hex strings  (...)

\cs_new_protected:cpn { @@_text_convert_utf16/hex:nN } #1 #2
  {
     \str_set_convert:Nnnn #2
       { #1 }
       { default }
       {utf16/hex}
  }

\cs_new_protected:cpn { @@_text_convert_utf16/hex-print:nN } #1 #2
  {
    \use:c { @@_text_convert_utf16/hex:nN } {#1} #2
    \str_put_left:Nn #2  {<}
    \str_put_right:Nn #2 {>}
  }

%    \end{macrocode}
% \end{macro}
% \end{macro}
% \end{macro}
% \end{macro}
% \end{macro}
% \end{macro}
% \end{macro}
% \end{macro}
%
% \subsubsection{BDC  operator commands}
% \begin{macro}{\pdf_bdc:nn}
% \begin{macro}{\pdf_bdc_obj:nn}
% \begin{macro}{\pdf_bdc_obj:n}
% \begin{macro}{\pdf_bmc:n}
% \begin{macro}{\pdf_emc:}
%    \begin{macrocode}
%<*package>
\cs_new_protected:Npn  \pdf_bdc:nn #1 #2 { \@@_backend_bdc:nn { #1 }{ #2 } }
\cs_new_protected:Npn  \pdf_bdc_obj:nn #1 #2 { \@@_backend_bdc_obj:nn { #1 }{ #2 } }
\cs_new_protected:Npn  \pdf_bdc_obj:n #1     { \@@_backend_bdc_obj:n  { #1 } }
\cs_new_protected:Npn  \pdf_bmc:n #1     { \@@_backend_bmc:n  { #1 } }
\cs_new_protected:Npn  \pdf_emc:         { \@@_backend_emc: }
%</package>
%    \end{macrocode}
% \end{macro}
% \end{macro}
% \end{macro}
% \end{macro}
% \end{macro}
%    \begin{macrocode}
%</package>
%    \end{macrocode}
%
% \subsection{Reference to the page object}
% \begin{macro}{ \pdf_object_pageref:n }
%    \begin{macrocode}
%<*package>
\cs_new:Npn \pdf_object_pageref:n #1 { \@@_backend_pageobject_ref:n { #1 }}
%</package>
%    \end{macrocode}
% \end{macro}
% \end{implementation}
%
% \PrintIndex
