% \iffalse meta-comment
%
%% File: l3pdftools.dtx
%
% Copyright (C) 2018-2020 The LaTeX3 Project
%
% It may be distributed and/or modified under the conditions of the
% LaTeX Project Public License (LPPL), either version 1.3c of this
% license or (at your option) any later version.  The latest version
% of this license is in the file
%
%    http://www.latex-project.org/lppl.txt
%
% This file is part of the "(experimental) pdfresources bundle" (The Work in LPPL)
% and all files in that bundle must be distributed together.
%
% -----------------------------------------------------------------------
%
% The development version of the bundle can be found at
%
%    https://github.com/latex3/pdfresources
%
% for those people who are interested.
%
%<*driver>
\RequirePackage{expl3}
\documentclass[full]{l3doc}
\providecommand\potentialclash{\noindent\llap{\dbend\ }}
\begin{document}
  \DocInput{\jobname.dtx}
\end{document}
%</driver>
% \fi
%
% \title{^^A
%   The \pkg{l3pdftools} package\\ collection of pdf related commands ^^A
% }
%
% \author{^^A
%  The \LaTeX3 Project\thanks
%    {^^A
%      E-mail:
%        \href{mailto:latex-team@latex-project.org}
%          {latex-team@latex-project.org}^^A
%    }^^A
% }
%
% \date{Released XXXX-XX-XX}
%
% \maketitle
% \begin{documentation}
%
% \section{\pkg{l3pdftools} documentation}
%
% This module collects a number of commands of the tool type which haven't found
% another home yet ... Probably this module will disappear again at some time.
%
% \begin{function}[EXP,added=2020-07-04]
%   {\pdf_text_convert:nn}
%  \begin{syntax}
%   \cs{pdf_text_convert:nn}  \Arg{format}  \Arg{content}
%  \end{syntax}
%  This converts \Arg{content} following the rules defined by \Arg{format}.
%  Non-ascii input should be utf8 encoded.
%  Currently the following formats exist:
%  \begin{description}
%     \item[name] This will first expand the content with \cs{text_expand:n} and
%      then escape it in the way needed in a PDF Name with
%      \cs{str_convert_pdfname:e}.
%     \item[name-print] This does the same as the \texttt{name} format, but also
%     adds a slash before, so |\pdf_text_convert:nn{name-print}{abc}| will
%     output \texttt{/abc}.
%   \end{description}
% \end{function}
% \begin{function}[added=2020-07-04]
%   {\pdf_text_convert:nnN}
%  \begin{syntax}
%   \cs{pdf_text_convert:nn}  \Arg{format}  \Arg{content} \Arg{tlvar}
%  \end{syntax}
%  This converts \Arg{content} following the rules defined by \Arg{format} and stores
%  the result in \Arg{tlvar}. The assignment is done locally.
%  Non-ascii input should be utf8 encoded.
%  Currently the following formats exist:
%  \begin{description}
%     \item[name] see above
%     \item[name-print] see above
%     \item[string-lit]
%      this converts with \cs{\str_set_convert:Nnnn} into utf16/string.
%     \item[string-lit-print]
%      this converts into utf16/string and add parentheses around.
%     \item[string-hex]
%      this converts into utf16/hex
%     \item[string-hex-print]
%     this converts into utf16/hex and add bracket around.
%   \end{description}
% \end{function}
% \end{documentation}
%
% \begin{implementation}
%
% \section{\pkg{l3pdftools} implementation}
%
%    \begin{macrocode}
%<*package>
\ProvidesExplPackage {l3pdftools} {2020-07-04} {0.1}
  {various PDF tools}
%    \end{macrocode}
%
%    \begin{macrocode}
%<@@=pdf>
%    \end{macrocode}
% \subsection{Conversions}
% \begin{macro}{\pdf_text_convert:nn,\pdf_text_convert:nV}
% \begin{macro}{\pdf_text_convert:nnN}
% \begin{macro}{\@@_text_convert_name:n}
% \begin{macro}{\@@_text_convert_name:nN}
% \begin{macro}{\@@_text_convert_name-print:n}
% \begin{macro}{\@@_text_convert_name-print:n}
%    \begin{macrocode}

\cs_new:Npn \pdf_text_convert:nn #1 #2
  {
    \cs_if_exist_use:cF { @@_text_convert_#1:n }
      {
        \__kernel_msg_error:nnn { pdf } { unknown-exp-convert } {#1}
        \use_none:n
      }
    { #2 }
  }
\cs_generate_variant:Nn \pdf_text_convert:nn {nV}

\cs_new:Npn \pdf_text_convert:nnN #1 #2 #3
  {
    \cs_if_exist_use:cF { @@_text_convert_#1:nN }
      {
        \__kernel_msg_error:nnn { pdf } { unknown-convert } {#1}
        \use_none:nn
      }
    { #2 } #3
  }

\cs_generate_variant:Nn \pdf_text_convert:nnN {nVN}
%% tool command escape name
%% commands in the argument are expanded if possible
%% with \text_expand:n
%% name: only name,
%% name_print: with / before, see also string naming

\cs_generate_variant:Nn \str_convert_pdfname:n { e }

\cs_new:Npn \@@_text_convert_name:n #1
  {
    \str_convert_pdfname:e { \text_expand:n { #1 } }
  }

\cs_new_protected:Npn \@@_text_convert_name:nN #1 #2
  {
    \tl_set:Nx #2 {\@@_text_convert_name:n {#1} }
  }

\cs_new:cpn { @@_text_convert_name-print:n } #1
  {
    / \str_convert_pdfname:e { \text_expand:n { #1 } }
  }

\cs_new_protected:cpn { @@_text_convert_name-print:nN } #1 #2
  {
    \tl_set:Nx #2 {\use:c { @@_text_convert_name-print:n } {#1} }
  }

%    \end{macrocode}
% \end{macro}
% \end{macro}
% \end{macro}
% \end{macro}
% \end{macro}
% \end{macro}
%
% \begin{macro}{ \@@_text_convert_string-lit:nN }
% \begin{macro}{ \@@_text_convert_string-lit-print:nN }
% \begin{macro}{ @@_text_convert_string-hex:nN }
% \begin{macro}{ @@_text_convert_string-hex-print:nN }
%    \begin{macrocode}
%% string conversions and printing
%% we assume here that the text purify step has been done. The input is
%% a list of (utf8) chars.
%% str convert, not expandable.
%  filespec (attachment view)  tests:
%  utf8:  gr\303\274\303\237e.txt                    %doesn't work, umlaut wrong,
%  utf8 with BOM \357\273\277gr\303\274\303\237e.txt %doesn't work, umlaut wrong, bom visible
%  utf16 with BE: (FEFF)  \376\377\000g\000r\000\374\000\337\000e\000.\000t\000x\000t %works
%                xetex converts to <feff0067007200fc00df0065002e007400780074>
%  utf16 with BE / HEX: <FEFF0067007200FC00DF0065002E007400780074> works

% bookmarks: as pdfoutline uses () currently only utf16 with BE is usable.
% check if one can use HEX too when directly writing the object

% "input" is utf8 for pdftex, empty (native) for unicode engine
% commands to output literal strings  (...)
\sys_if_engine_pdftex:TF
  {
    \cs_new_protected:cpn { @@_text_convert_string-lit:nN }  #1 #2
      {
         \str_set_convert:Nnnn #2 { #1 }{ utf8 } {utf16/string}
      }

    \cs_new_protected:cpn { @@_text_convert_string-lit-print:nN } #1 #2
      {
         \str_set_convert:Nnnn #2 { #1 }{ utf8 } {utf16/string}
         \str_put_left:Nn #2  {(}
         \str_put_right:Nn #2 {)}
      }
  }
  {
     \cs_new_protected:cpn { @@_text_convert_string-lit:nN } #1 #2
      {
        \str_set_convert:Nnnn #2 { #1 }{  } {utf16/string}
      }

    \cs_new_protected:cpn { @@_text_convert_string-lit-print:nN } #1 #2
      {
        \str_set_convert:Nnnn #2 { #1 }{  } {utf16/string}
        \str_put_left:Nn #2  {(}
        \str_put_right:Nn #2 {)}
      }
  }
% commands to output hex strings  (...)

\sys_if_engine_pdftex:TF
  {
    \cs_new:cpn { @@_text_convert_string-hex:nN } #1 #2
      {
         \str_set_convert:Nnnn #2 { #1 }{ utf8 } {utf16/hex}
      }
    \cs_new:cpn { @@_text_convert_string-hex-print:nN } #1 #2
      {
         \str_set_convert:Nnnn #2 { #1 }{ utf8 } {utf16/hex}
         \str_put_left:Nn #2  {<}
         \str_put_right:Nn #2 {>}
      }
  }
  {
    \cs_new:cpn { @@_text_convert_string-hex:nN } #1 #2
      {
        \str_set_convert:Nnnn #2 { #1 }{  } {utf16/hex}
      }

    \cs_new:cpn { @@_text_convert_string-hex-print:nN } #1 #2
      {
        \str_set_convert:Nnnn #2 { #1 }{  } {utf16/hex}
         \str_put_left:Nn #2  {<}
         \str_put_right:Nn #2 {>}
      }
  }

%    \end{macrocode}
% \end{macro}
% \end{macro}
% \end{macro}
% \end{macro}
%    \begin{macrocode}
%</package>
%    \end{macrocode}
%
% \end{implementation}
%
% \PrintIndex
