% \iffalse meta-comment
%
%% File: ltdocinit.dtx
%
% Copyright (C) 2018-2025 The LaTeX Project
%
% It may be distributed and/or modified under the conditions of the
% LaTeX Project Public License (LPPL), either version 1.3c of this
% license or (at your option) any later version.  The latest version
% of this license is in the file
%
%    http://www.latex-project.org/lppl.txt
%
% This file is part of the "LaTeX PDF management testphase bundle" (The Work in LPPL)
% and all files in that bundle must be distributed together.
%
% -----------------------------------------------------------------------
%
% The development version of the bundle can be found at
%
%    https://github.com/latex3/pdfresources
%
% for those people who are interested.
%
%<*driver>
\DocumentMetadata{tagging=on,pdfstandard=ua-2}
\documentclass[full]{l3doc}
\usepackage{latex-lab-testphase-l3doc}
\usepackage{array,booktabs}
\hypersetup{pdfauthor=The LaTeX Project,pdftitle=ltdocinit (LaTeX PDF management testphase bundle)}

\begin{document}
  \DocInput{\jobname.dtx}
\end{document}
%</driver>
% \fi
%
% \title{^^A
%   The \pkg{ltdocinit} module
% }
%
% \author{^^A
%  The \LaTeX{} Project\thanks
%    {^^A
%      E-mail:
%        \href{mailto:latex-team@latex-project.org}
%          {latex-team@latex-project.org}^^A
%    }^^A
% }
%
% \date{Version 0.96t, released 2025-06-29}
%
% \maketitle
% \begin{documentation}
%
% \section{\pkg{ltdocinit} documentation}
% This small module previously defined \cs{DocumentMetadata} but the definition
% has been moved to latex-lab. The documentation can be found in
% \texttt{documentmetadata-support-doc.pdf}
%
% \cs{DocumentMetadata} is also used to
% activate the new PDF management code and it loads
% a number of required files for the PDF management code.
% As this forces the loading of the backend files, a backend
% which can't be detected automatically like |dvipdfmx|,
% must be set in the first \cs{DocumentMetadata}.
%
%
% The module also defines commands to store document properties in a global container.
%
% This module will slowly disappear.
%
%
% \subsection{Container for document properties}
%
% The module provides a container where classes, packages and users can store
% properties of the document which are perhaps of interest
% or use for other packages or the author.
%
% The properties are stored with a key |label/property|. The values can be
% retrieved expandably.
%
% \begin{function}{\AddToDocumentProperties}
% \begin{syntax}
%  \cs{AddToDocumentProperties}\oarg{label}\Arg{property}\Arg{value}
% \end{syntax}
% This stores \meta{value} under the key \meta{label}/\meta{property}.
% By default \meta{label} is the current package name |\@currname|. If another
% label is chosen, it should be one which avoids clashes with other packages
% using the container. The label |document| is reserved.
% \end{function}
%
% \begin{function}[updated=2025-06-21]{\GetDocumentProperty,\pdfmanagement_get_documentproperty:n}
% \begin{syntax}
%  \cs{GetDocumentProperty}\Arg{label/property}\\
%  \cs{pdfmanagement_get_documentproperty:n}\Arg{label/property}
% \end{syntax}
% Expands to the \meta{value} corresponding to \meta{label/property}
% in the container. If \meta{label/property} is missing,
% this has an empty expansion.
% The result is returned within \cs{exp_not:n}, which means
% that the \meta{value} does not expand further
% when appearing in an x-type argument expansion.
% The older names \cs{GetDocumentProperties} and \cs{pdfmanagement_get_documentproperties:n}
% are still provided for compability. 
% \end{function}
%
% \begin{function}[updated=2025-06-21,TF]{\pdfmanagement_get_documentproperty:nN}
% \begin{syntax}
%  \cs{pdfmanagement_get_documentproperty:nNTF}\Arg{label/property} \meta{token list variable}\\
%  ~~~~ \Arg{true code} \Arg{false code}
% \end{syntax}
%   If the \meta{label/property} is not present in the document properties
%   container, leaves the \meta{false code} in the input stream.
%   The value of the
%   \meta{token list variable} is not defined in this case and should
%   not be relied upon.  If the \meta{label/property} is present in the
%   container, stores the corresponding \meta{value} in the
%   \meta{token list variable} without removing it from the
%   container, then leaves the \meta{true code} in the input
%   stream.  The \meta{token list variable} is assigned locally.
%   The older name \cs{pdfmanagement_get_documentproperty:nN}
%   is still provided for compability. 
% \end{function}
%
% \begin{function}{\ShowDocumentProperties,\LogDocumentProperties}
% \begin{syntax}
%  \cs{ShowDocumentProperties}\\
%  \cs{LogDocumentProperties}
% \end{syntax}
% This shows/logs the current content of the container.
% \end{function}
%
% \begin{thebibliography}{9}
%
% \bibitem{blueprint} Frank Mittelbach and Chris Rowley:
%   \emph{\LaTeX{} Tagged PDF\,---\,A blueprint for a large project}.
%   \url{https://latex-project.org/publications/indexbyyear/2020/}
%
% \end{thebibliography}
%
% \end{documentation}
%
% \begin{implementation}
%
% \section{\pkg{ltdocinit} implementation}
%    \begin{macrocode}
%<@@=pdfmanagement>
%<*header>
\ProvidesExplPackage{ltdocinit}{2025-06-29}{0.96t}
  {Initialize document metadata}
%</header>
%    \end{macrocode}
%    \begin{macrocode}
%<*package>
%    \end{macrocode}
% \subsection{Variables}
% This variable definitions can be remove once latex-lab has been updated to
% provide them too.
%  \begin{variable}{\g_@@_firstaidoff_clist}
%  A list to store the firstaid code which should be disabled
%    \begin{macrocode}
\clist_if_exist:NF \g_@@_firstaidoff_clist
  { \clist_new:N \g_@@_firstaidoff_clist }
%    \end{macrocode}
% \end{variable}
%
%
% \subsection{Container for document Properties}
% The container for the document properties is a prop
% \begin{variable}{\g_@@_documentproperties_prop}
%    \begin{macrocode}
\prop_new:N \g_@@_documentproperties_prop %
%    \end{macrocode}
% \end{variable}
% \begin{macro}{\AddToDocumentProperties}
%    \begin{macrocode}
\NewDocumentCommand\AddToDocumentProperties{O{\@currname}mm}
  {
    \exp_args:NNe
      \prop_gput:Nnn \g_@@_documentproperties_prop
        {
          \tl_if_blank:eTF {#1}{top-level/}{#1/} #2
        }
        { #3}
  }
%    \end{macrocode}
% \end{macro}
% \begin{macro}[EXP]{\GetDocumentProperty,\GetDocumentProperties}
%    \begin{macrocode}
\cs_new:Npn\GetDocumentProperty #1
  {
    \prop_item:Nn \g_@@_documentproperties_prop {#1}
  }
\cs_new_eq:NN\GetDocumentProperties\GetDocumentProperty
%    \end{macrocode}
% \end{macro}
% For uses in modules (e.g. l3pdfmeta)
% we provide an expl3 version without the xparse overhead:
%
% \begin{macro}[EXP]{\pdfmanagement_get_documentproperty:n}
%    \begin{macrocode}
\cs_new:Npn\pdfmanagement_get_documentproperty:n #1
  {
    \prop_item:Nn \g_@@_documentproperties_prop {#1}
  }
\cs_new_eq:NN 
  \pdfmanagement_get_documentproperties:n 
  \pdfmanagement_get_documentproperty:n  
%    \end{macrocode}
% \end{macro}
% The following allows to retrieve the values with branching.
% \begin{macro}[added=2025-06-21]{\pdfmanagement_get_documentproperty:nNTF}
%    \begin{macrocode}
\prg_new_protected_conditional:Npnn
  \pdfmanagement_get_documentproperty:nN #1#2 { T , F , TF }
    {
     \prop_get:NnNTF \g_@@_documentproperties_prop  {#1} #2
       {
        \prg_return_true:
       }
       {
        \prg_return_false:
       }
    }
\prg_new_eq_conditional:NNn   
  \pdfmanagement_get_documentproperties:nN  
  \pdfmanagement_get_documentproperty:nN
  { T , F , TF }
%    \end{macrocode}
% \end{macro}
% \begin{macro}{\ShowDocumentProperties}
%    \begin{macrocode}
\msg_new:nnn  { pdfmanagement } { show-properties }
  {
    The~following~document~properties~have~been~stored:
    #1
  }
\NewDocumentCommand\ShowDocumentProperties {}
  {
    \msg_show:nne {pdfmanagement}{show-properties}
      {
        \prop_map_function:NN \g_@@_documentproperties_prop \msg_show_item:nn
      }
  }
\NewDocumentCommand\LogDocumentProperties {}
  {
    \msg_log:nne {pdfmanagement}{show-properties}
      {
        \prop_map_function:NN \g_@@_documentproperties_prop \msg_show_item:nn
      }
  }
%    \end{macrocode}
% \end{macro}
%    \begin{macrocode}
%</package>
%    \end{macrocode}
%
% \end{implementation}
%
% \PrintIndex
