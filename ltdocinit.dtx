% \iffalse meta-comment
%
%% File: ltdocinit.dtx
%
% Copyright (C) 2018-2022 The LaTeX Project
%
% It may be distributed and/or modified under the conditions of the
% LaTeX Project Public License (LPPL), either version 1.3c of this
% license or (at your option) any later version.  The latest version
% of this license is in the file
%
%    http://www.latex-project.org/lppl.txt
%
% This file is part of the "LaTeX PDF management testphase bundle" (The Work in LPPL)
% and all files in that bundle must be distributed together.
%
% -----------------------------------------------------------------------
%
% The development version of the bundle can be found at
%
%    https://github.com/latex3/pdfresources
%
% for those people who are interested.
%
%<*driver>
\RequirePackage{pdfmanagement-testphase}
\DocumentMetadata{pdfstandard=A-2b}
\documentclass[full]{l3doc}
\usepackage{array,booktabs,hyperxmp}
\hypersetup{pdfauthor=The LaTeX Project,pdftitle=ltdocinit (LaTeX PDF management testphase bundle)}

\begin{document}
  \DocInput{\jobname.dtx}
\end{document}
%</driver>
% \fi
%
% \title{^^A
%   The \pkg{ltdocinit} module
% }
%
% \author{^^A
%  The \LaTeX{} Project\thanks
%    {^^A
%      E-mail:
%        \href{mailto:latex-team@latex-project.org}
%          {latex-team@latex-project.org}^^A
%    }^^A
% }
%
% \date{Version 0.95q, released 2022-07-08}
%
% \maketitle
% \begin{documentation}
%
% \section{\pkg{ltdocinit} documentation}
% This small module defines \cs{DocumentMetadata} and the older alias
% \cs{DeclareDocumentMetadata} and the related keys.
% It also defines commands to store document properties in a global container.
%
% When the kernel will provide \cs{DocumentMetadata} directly
% this module will slowly disappear.
%
% \subsection{\cs{DocumentMetadata}/\cs{DeclareDocumentMetadata}}
%
% \begin{function}{\DocumentMetadata,\DeclareDocumentMetadata}
% \begin{syntax}
%  \cs{DocumentMetadata}\Arg{key-value list}\\
%  \cs{DeclareDocumentMetadata}\Arg{key-value list} (deprecated)
% \end{syntax}
% \end{function}
%
% Currently there is no dedicated location to declare settings concerning
% a document as a whole. Settings are placed somewhere in the preamble or
% with the class options or even with some package options.
% For some settings this can be too late,
% for example the pdf version can no longer be changed if a
% package has used code which already opened the PDF.
%
% \cs{DocumentMetadata} as a new command
% should unify such settings in one place.
% It should be used before \cs{documentclass}.
% \cs{DocumentMetadata} can be used more than once.
%
% With a \LaTeX{} 2022-06-01 or newer (currently provided as |latex-dev|) the command
% is provided by the kernel and will load the pdfmanagement,
% with older \LaTeX{} the pdfmanagement must be loaded
% first manually with |\RequirePackage{pdfmanagement-testphase}|.
%
% The keys defined for \cs{DocumentMetadata}
% currently allows to set the PDF version, to set the PDF \texttt{/Lang},
% to uncompress a pdf, to set the language and to declare a few PDF standards
% and to load some colorprofiles.
%
% \cs{DocumentMetadata} is also used to
% activate the new PDF management code and it loads
% a number of required files for the PDF management code.
% As this forces the loading of the backend files, a backend
% which can't be detected automatically like |dvipdfmx|,
% must be set in the first \cs{DocumentMetadata}.
%
% Currently the following keys are implemented
%
% \begin{description}
%    \item[\texttt{backend}] Passes the backend name to expl3. This is
%      needed only if the needed backend can't be automatically
%      determined or if the workflow used requires a special backend.
%
%^^A    This will probably be extended to  pass the value also to
%^^A      packages.
%
%    \item[\texttt{pdfversion}] Sets the PDF version explicitly, e.g., \texttt{pdfversion=1.7}
%    \item[\texttt{uncompress}] (no value) Forces an uncompressed pdf
%      --- mainly for debugging purposes.
%
%    \item[\texttt{lang}] Explicitly sets the Lang entry in the Catalog,
%     e.g., \texttt{lang=de-DE}. If not given the default value used is |en-US|.
%
%    \item[\texttt{pdfstandard}] Choice key to set the pdf standard.
%      Currently |A-1b|, |A-2a|, |A-2b|, |A-2u|, |A-3a|, |A-3b|, |A-3u| and |A-4|
%      are accepted as
%      values. The casing is irrelevant, |a-1b| works too.
%      The underlying code to ensure the requirements (as far as they
%      can be ensured) is still incomplete, but a color profile is included and the
%      \texttt{/OutputIntent} is set. The |u| variants for example do not force unicode,
%      but they will pass the information to hyperref and hyperxmp. The |a| variants
%      do \emph{not} enforce (or even test) a tagged pdf yet.
%      More information can be found in the documentation
%      of \pkg{l3pdfmeta}.
%
%    \item[\texttt{colorprofiles}] This allows to load icc-colorprofiles. Details
%       are described in the documentation of \pkg{l3pdfmeta}.
%
%    \item[\texttt{testphase}] This key is used to load testphase code. The values it accepts
%    and their effect will change over time, when testphase packages are added or
%    removed or when the code is moved into the kernel.
%     \begin{description}
%    \item[\texttt{phase-I}]
%       This value loads code implementing the first phase of the project~\cite{blueprint}, i.e., it
%       will load the tagpdf package. It will also activate tagging by issuing
%       |\tagpdfsetup{activate,interwordspace}|. This phase
%       is frozen.
%    \item[\texttt{phase-II}]
%       This is the current development phase.
%       Currently it differs from \texttt{phase-I} only in one point: It will
%       additionally activate tagging of paragraphs with
%       |\tagpdfsetup{paratagging}|. In the upcoming months it will
%       also enable automatic tagging of other basic document elements.
%    \end{description}
%    The |testphase| key can only be used in the first \cs{DocumentMetadata}.
%
%    \item[\texttt{debug}] This key activates some debug options. It takes a list of key-values
%    as value. Currently the following keys are known:
%     \begin{description}
%     \item[\texttt{para}] with the default and only value |show|. It will activate the |paratagging-show|
%       option of \pkg{tagpdf},
%     \item[\texttt{log}]  with the values as described in the documentation \pkg{tagpdf},
%     \item[\texttt{uncompress}] which does the same as |uncompress| as main key
%     \item[\texttt{pdfmanagement}] a boolean which allows to deactivate the pdfmanagement.
%     \item[\texttt{firstaidoff}] This accepts a comma lists of keywords and disables the patches
%       related to them. More information can be found in the documentation of
%       \pkg{pdfmanagement-firstaid}.
%    \end{description}
% \end{description}
%
% \subsection{Container for document properties}
%
% The module provides a container where classes, packages and users can store
% properties of the document which are perhaps of interest
% or use for other packages or the author.
%
% The properties are stored with a key |label/property|. The values can be
% retrieved expandably.
%
% \begin{function}{\AddToDocumentProperties}
% \begin{syntax}
%  \cs{AddToDocumentProperties}\oarg{label}\Arg{property}\Arg{value}
% \end{syntax}
% This stores \meta{value} under the key \meta{label}/\meta{property}.
% By default \meta{label} is the current package name |\@currname|. If another
% label is chosen, it should be one which avoids clashes with other packages
% using the container. The label |document| is reserved.
% \end{function}
%
% \begin{function}{\GetDocumentProperties}
% \begin{syntax}
%  \cs{GetDocumentProperties}\Arg{label/property}
% \end{syntax}
% Expands to the \meta{value} corresponding to \meta{label/property}
% in the container. If \meta{label/property} is missing,
% this has an empty expansion.
% The result is returned within \cs{exp_not:n}, which means
% that the \meta{value} does not expand further
% when appearing in an x-type argument expansion.
% \end{function}
%
% \begin{function}{\ShowDocumentProperties}
% \begin{syntax}
%  \cs{ShowDocumentProperties}
% \end{syntax}
% This show the current content of the container.
% \end{function}
%
% \begin{thebibliography}{9}
%
% \bibitem{blueprint} Frank Mittelbach and Chris Rowley:
%   \emph{\LaTeX{} Tagged PDF\,---\,A blueprint for a large project}.
%   \url{https://latex-project.org/publications/indexbyyear/2020/}
%
% \end{thebibliography}
%
% \end{documentation}
%
% \begin{implementation}
%
% \section{\pkg{ltdocinit} implementation}
%    \begin{macrocode}
%<@@=pdfmanagement>
%<*header>
\ProvidesExplPackage{ltdocinit}{2022-07-08}{0.95q}
  {Initialize document metadata}
%</header>
%    \end{macrocode}
% \cs{DocumentMetadata} will be defined by the kernel in short time.
% So we define it and the keys here only if it is not already defined.
%  \subsection{The keys for \cs{DocumentMetadata}}
%  We define the keys first so that we can test if \cs{DocumentMetadata}
%  exist (testing for the format date would be nice but would fail for the current
%  latex-dev).
%  \begin{variable}{\g_@@_firstaidoff_clist}
%  A list to store the firstaid code which should be disabled
%    \begin{macrocode}
%<*package>
\clist_new:N \g_@@_firstaidoff_clist
%    \end{macrocode}
% \end{variable}
%  \begin{variable}{\g_@@_testphase_tl}
%  a tl to store the testphase loading code so that we can load them at
%  the end of the command.
%    \begin{macrocode}
\tl_new:N \g_@@_testphase_tl
%    \end{macrocode}
% \end{variable}
%    \begin{macrocode}
\cs_if_free:NT \DocumentMetadata
  {
    \keys_define:nn { document / metadata }
      {
        backend .choices:nn =
          { dvipdfmx , dvips , dvisvgm , luatex , pdftex , pdfmode , xdvipdfmx , xetex }
          {
            \sys_load_backend:n {#1}
          },
        backend .groups:n = { init } ,
      }

    \keys_define:nn { document / metadata }
      {
        ,pdfversion .code:n =
          {
            \pdf_version_gset:n { #1 }
            \AddToDocumentProperties[document]{pdfversion}{#1}
          }
        ,uncompress .code:n =
          {
            \pdf_uncompress:
          }
        ,uncompress .value_forbidden:n = true
        ,lang .code:n =
          {
            \pdfmanagement_add:nnn {Catalog} {Lang}{(#1)}
            \AddToDocumentProperties[document]{lang}{#1}
          }
        %,xmpmeta .bool_gset:N = \g_pdfmeta_xmp_bool %see pdfmeta unused and undefined for now!
        % this uses internal command from pdfmeta, it should probably move there ...
        ,pdfstandard .code:n =
          {
            \exp_args:Nnx
            \keys_set:nn {document / metadata} {_pdfstandard=\str_uppercase:n{#1}}
          }
        ,_pdfstandard .choices:nn =
          {A-1B,A-2A,A-2B,A-2U,A-3A,A-3B,A-3U,A-4}
          {
            \prop_if_exist:cT { g__pdfmeta_standard_pdf/#1_prop }
              {
                \prop_gset_eq:Nc \g__pdfmeta_standard_prop { g__pdfmeta_standard_pdf/#1 _prop }
              }
            \AddToDocumentProperties [document]{pdfstandard}{#1}
          }
        ,_pdfstandard / unknown .code:n =
          {
            \msg_warning:nnn{pdf}{unknown-standard}{#1}
          }
        ,testphase .multichoice:
        ,testphase / tagpdf .code:n =
          {
            \tl_gput_right:Nn\g_@@_testphase_tl
              {
                \file_if_exist_input:nF {tagpdf-latex-lab-testphase.ltx}
                  {
                    \RequirePackage{tagpdf}
                    \AddToDocumentProperties [document]{testphase/tagpdf}{loaded}
                    \tagpdfsetup{activate,paratagging,interwordspace}
                    \AddToDocumentProperties [document]{tagging}{active}
                    \AddToDocumentProperties [document]{tagging/para}{active}
                    \AddToDocumentProperties [document]{tagging/interwordspace}{active}
                  }
              }
          }
        ,testphase / unknown .code:n =
          {
            \tl_gput_right:Nn\g_@@_testphase_tl
               {
                 \AddToDocumentProperties [document]{testphase/#1}{loaded}
                 \file_if_exist_input:nF {#1-latex-lab-testphase.ltx}
                  {
                     \msg_warning:nnn{document}{latex-lab-pkg-missing}{#1}
                     \AddToDocumentProperties [document]{testphase/#1}{missing}
                  }
               }
          }
        ,activate .multichoice:
        ,activate / tagging .code:n =
          {
            \PackageWarning{pdfmanagement-testphase}
             {The~activate~key~is~deprecated.\MessageBreak
              Tagging~is~activated~with~'testphase=tagpdf'~directly}{}
          }
        ,debug .code:n =
          {
            \keys_set:nn { document / metadata / debug } {#1}
          }
        ,debug / para .code:n =
          {
            \AddToHook
              {
                package/tagpdf/after
              }
              {
                 \tagpdfsetup{paratagging-show}
              }
          }
        ,debug / log .code:n =
          {
            \AddToHook
              {
               package/tagpdf/after
              }
              {
                 \tagpdfsetup{log=#1}
              }
          }
        ,debug / uncompress .code:n =
          {
            \pdf_uncompress:
          }
        ,debug / pdfmanagement .bool_gset:N = \g_@@_active_bool
        ,debug / firstaidoff .clist_gset:N = \g_@@_firstaidoff_clist
      }
   }
%    \end{macrocode}
%
% \subsection{\cs{DocumentMetadata}/\cs{DeclareDocumentMetadata}}
% At first we have to define the older alias \cs{DeclareDocumentMetadata} if
% \cs{DocumentMetadata} is already defined in a newer LaTeX version. Here
% we load the latex-lab support (if it hasn't been loaded
% before) and redefine then \cs{DeclareDocumentMetadata}
% to \cs{DocumentMetadata}
%
%    \begin{macrocode}
\cs_if_exist:NT \DocumentMetadata
   {
     \cs_new_protected:Npn \DeclareDocumentMetadata
        {
          \cs_if_free:NT \DocumentMetadata
            {
              \file_if_exist_input:nF {documentmetadata-support.ltx}%
               {
                 \@latex@error
                   {No~support~files~for~\noexpand\DeclareDocumentMetadata~found}
                   {
                     Is~the~'LaTeX-lab'~bundle~installed?
                     \MessageBreak
                     Without~it,~the~declaration~is~ignored.
                   }
                \let\DeclareDocumentMetadata\@gobble
               }
            }
          \RenewDocumentCommand \DeclareDocumentMetadata {}
           { \DocumentMetadata }
          \DocumentMetadata
        }
    }
%    \end{macrocode}
% Now we provide a definition for \cs{DocumentMetadata} if it doesn't exist yet.
%    \begin{macrocode}
\cs_if_free:NT \DocumentMetadata
  {
%    \end{macrocode}
% \cs{DocumentMetadata} should for now not be used after
% \cs{documentclass} so we error in this case. It can be used more than once
% but follow-up calls should not do the initialization code.
%    \begin{macrocode}
    \prop_gput:Nnn \g_msg_module_type_prop { document } { LaTeX }
    \prop_gput:Nnn \g_msg_module_name_prop { document } { DocumentMetadata }
    \msg_new:nnn
      { document } { setup-after-documentclass }
      {
        \token_to_str:N \DocumentMetadata \c_space_tl
        should~be~used~only~before~\token_to_str:N\documentclass
      }
    \msg_new:nnn
      { document } { latex-lab-pkg-missing }
      {
        LaTeX-lab~package~'#1'~not~found.
      }
%    \end{macrocode}
% \begin{macro}{\DocumentMetadata}
%    \begin{macrocode}
    \NewDocumentCommand\DocumentMetadata { m }
      {
        \cs_if_eq:NNTF \documentclass \@twoclasseserror
          { \msg_error:nn { document }{ setup-after-documentclass } }
          {
%    \end{macrocode}
% The wanted backend must be detected first, we read the init key and then
% force the loading by loading expl3.
% The backend can contains management commands, so the boolean should be set to true first.
%    \begin{macrocode}
            \bool_gset_true:N \g_@@_active_bool
            \keys_set_groups:nnn { document / metadata} {init}{ #1 }
        %if no backend has been loaded force it now:
            \str_if_exist:NF \c_sys_backend_str
              {
                \sys_load_backend:n {}
              }
%    \end{macrocode}
% Now we load the extra backend code
%    \begin{macrocode}
        \ExplSyntaxOn\makeatletter
              \file_input:n {l3backend-testphase-\c_sys_backend_str.def}
        \ExplSyntaxOff\makeatother
%    \end{macrocode}
% set the default language, process the rest of the keys,
% and setup the generic driver
%    \begin{macrocode}
            \keys_set_filter:nnn  { document / metadata } { init } { lang=en-US, #1 }
            \bool_if:NT \g_@@_active_bool
              {
                \PassOptionsToPackage{customdriver=hgeneric-testphase}{hyperref}
              }
%    \end{macrocode}
% \cs{pdfmanagement_add:nnn} has collected values in this hook.
%    \begin{macrocode}
           \hook_use_once:n {pdfmanagement/add}
%    \end{macrocode}
% Redefine \cs{DocumentMetadata} so that it only process the keys.
% We need to update the hyperref option if the active status changes.
%    \begin{macrocode}
           \RenewDocumentCommand\DocumentMetadata { m }
              {
                \keys_set_filter:nnn  { document / metadata } { init } { ##1 }
                \bool_if:NTF \g_@@_active_bool
                  {
                    \str_remove_all:cn
                      {opt@hyperref.sty}{customdriver=hgeneric-testphase}
                    \PassOptionsToPackage
                      {customdriver=hgeneric-testphase}{hyperref}
                 }
                 {
                   \str_remove_all:cn
                     {opt@hyperref.sty}{customdriver=hgeneric-testphase}
                 }
              }
%    \end{macrocode}
% Load more modules, the testphase code and the firstaid code.
%    \begin{macrocode}
            \g_@@_testphase_tl
            \RequirePackage{pdfmanagement-firstaid}
          }
      }
%    \end{macrocode}
% At last we need to provide the older alias here too
%    \begin{macrocode}
    \NewDocumentCommand\DeclareDocumentMetadata{}{\DocumentMetadata}
  }
%    \end{macrocode}
% \end{macro}
%
% \subsection{Container for document Properties}
% The container for the document properties is a prop
% \begin{variable}{\g_@@_documentproperties_prop}
%    \begin{macrocode}
\prop_new:N \g_@@_documentproperties_prop %
%    \end{macrocode}
% \end{variable}
% \begin{macro}{\AddToDocumentProperties}
%    \begin{macrocode}
\NewDocumentCommand\AddToDocumentProperties{O{\@currname}mm}
  {
    \exp_args:NNx
      \prop_gput:Nnn \g_@@_documentproperties_prop
        {
          \tl_if_blank:eTF {#1}{top-level/}{#1/} #2
        }
        { #3}
  }
%    \end{macrocode}
% \end{macro}
% \begin{macro}{\GetDocumentProperties}
%    \begin{macrocode}
\NewExpandableDocumentCommand\GetDocumentProperties{m}
  {
    \prop_item:Nn \g_@@_documentproperties_prop {#1}
  }
%    \end{macrocode}
% \end{macro}
% \begin{macro}{\ShowDocumentProperties}
%    \begin{macrocode}
\msg_new:nnn  { pdfmanagement } { show-properties }
  {
    The~following~document~properties~have~been~stored:
    #1
  }
\NewDocumentCommand\ShowDocumentProperties {}
  {
    \msg_show:nnx {pdfmanagement}{show-properties}
      {
        \prop_map_function:NN \g_@@_documentproperties_prop \msg_show_item:nn
      }
  }
%    \end{macrocode}
% \end{macro}
%    \begin{macrocode}
%</package>
%    \end{macrocode}
%
% \end{implementation}
%
% \PrintIndex
