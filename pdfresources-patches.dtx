% \iffalse meta-comment
%
%% File: pdfresources-patches.dtx
%
% Copyright (C) 2018-2020 The LaTeX3 Project
%
% It may be distributed and/or modified under the conditions of the
% LaTeX Project Public License (LPPL), either version 1.3c of this
% license or (at your option) any later version.  The latest version
% of this license is in the file
%
%    http://www.latex-project.org/lppl.txt
%
% This file is part of the "(experimental) pdfresources bundle" (The Work in LPPL)
% and all files in that bundle must be distributed together.
%
% -----------------------------------------------------------------------
%
% The development version of the bundle can be found at
%
%    https://github.com/latex3/pdfresources
%
% for those people who are interested.
%
%<*driver>
\documentclass[full]{l3doc}
\begin{document}
  \DocInput{\jobname.dtx}
\end{document}
%</driver>
% \fi
% \title{^^A
%   The \pkg{pdfresources-patches} package  -- temporarly patches  ^^A
% }
%
% \author{^^A
%  The \LaTeX3 Project\thanks
%    {^^A
%      E-mail:
%        \href{mailto:latex-team@latex-project.org}
%          {latex-team@latex-project.org}^^A
%    }^^A
% }
%
% \date{Released XXXX-XX-XX}
%
% \maketitle
% \begin{documentation}
%
% \section{\pkg{pdfresources-patches} documentation}
% This code is temporary! It tries to patch commands of other packages which
% fills pdfresources too, to remove clashes and test if everything works as expected.
% This code should disappear when packages adapt to the central interfaces.
%
% \subsection{\pkg{pgf}}
%
% In \pkg{pgf}, resource management is set up in the file |pgfutil-common.tex|.
% This then provides three functions for adding to the resources, all of which
% are objects:
% \begin{itemize}
%   \item \cs{pgfutil@addpdfresource@extgs}: Extended graphics state
%   \item \cs{pgfutil@addpdfresource@colorspaces}: Color spaces
%   \item \cs{pgfutil@addpdfresource@patterns}: Patterns
% \end{itemize}
%
% These resource dictionaries are used by adding entries in a cumulative sense;
% the macro layer deals with ensuring that each entry is only given once. Note
% that the objects themselves must be given only once for each page.
%
% To support these functions, there are a series of set-up macros which install
% these resources. That has to take place for every page: the exact route
% therefore depends on the driver.
%
% For the pdfresources project we need to avoid that pgf interferes in ExtGState,
% ColorSpace and Pattern (Shadings are added to the xform resources and so probably
% unproblematic for now).
%    \begin{macrocode}
%<@@=pdf>
%<*package>
\ProvidesExplPackage {pdfresources-patches} {2020-07-13} {0.7}
  {experimental pdf-resource management / patches}
% patches for xetex/dvips doesn't make much sense for colorspace and
% transparent. But pgf should be tested sometimes.
\RequirePackage{etoolbox}
\@ifpackageloaded{pgfrcs}
 {
  \msg_warning:nn{pdf}{pgf-too-late}
  \file_input_stop:
 }
 {
  \msg_info:nn{pdf}{patches}
 }
\RequirePackage{pgfrcs}
%
%\def\pgfutil@addpdfresource@extgs#1{\pgf@sys@addpdfresource@extgs@plain{#1}}
%\def\pgfutil@addpdfresource@colorspaces#1{\pgf@sys@addpdfresource@colorspaces@plain{#1}}
%\def\pgfutil@addpdfresource@patterns#1{\pgf@sys@addpdfresource@patterns@plain{#1}}
%\def\pgfutil@setuppdfresources{\pgf@sys@setuppdfresources@plain}
% \pgf@sys@pdf@possible@resources %used in xform
%Trying to patch pgf ..
\cs_new_protected:Npn \@@_pgf_sys_setuppdfresources_plain:
  {
   %objects are already created ...
    \def\pgf@sys@pdf@possible@resources
      {
        /ColorSpace~\pdf_object_ref:n {Page/Resources/ColorSpace}
        /Pattern   ~\pdf_object_ref:n {Page/Resources/Pattern}
        /ExtGState ~\pdf_object_ref:n {Page/Resources/ExtGState}
      }
    \let\pgf@sys@pdf@check@resources=\relax%
    %not sure if needed, but perhaps the lists are used somewhere else ...
    \let\pgf@sys@pgf@resource@list@extgs=\pgfutil@empty%
    \let\pgf@sys@pgf@resource@list@patterns=\pgfutil@empty%
    \let\pgf@sys@pgf@resource@list@colorspaces=\pgfutil@empty%
    % the commands to add page resources
    \def\pgf@sys@addpdfresource@extgs@plain##1
      {
        \exp_after:wN %for transparent which passes a command
          \@@_patch_pgfextgs:w ##1\q_stop
      }
    \def\pgf@sys@addpdfresource@patterns@plain##1
      {
        \@@_patch_pgfpatterns:w ##1\q_stop
      }
   \def\pgf@sys@addpdfresource@colorspaces@plain##1
      {
        \@@_patch_pgfcolorspaces:w ##1\q_stop
      }
  }
\cs_set_eq:NN \@@_pgfori_pgfutil@setuppdfresources\pgfutil@setuppdfresources
\def\pgfutil@setuppdfresources
  {
    \pdfmanagement_if_active:TF
      {
        \@@_pgf_sys_setuppdfresources_plain:
      }
      {
        \@@_pgfori_pgfutil@setuppdfresources
      }
  }

% Why on earth do I explicitly call the command? Doesn't pgf do it?
%\hook_gput_code:nnn {begindocument/end} {pdf} { \pgfutil@setuppdfresources }

%\AtEndPreamble{\pgfutil@setuppdfresources}
% helper commands as pgf doesn't pass resources as two arguments
% code to add to the resources existing stuff in the format "/name value":
\cs_new:Npn \@@_split_dict_entry_aux:NNw #1 #2 /#3~#4\q_stop
  {
    \tl_set:Nn #1 {#3}
    \tl_set:Nn #2 {#4}
  }
\tl_new:N \l_@@_name_tmp_tl
\tl_new:N \l_@@_value_tmp_tl

\cs_new:Npn \@@_patch_pgfextgs:w  #1/#2<<#3>>#4\q_stop
  {
    \exp_args:Nne \@@_backend_PageResources_gput:nnn {ExtGState}{\tl_trim_spaces:n{#2}}{<<#3>>}
  }
\cs_new:Npn \@@_patch_pgfpatterns:w  #1/#2\space#3\q_stop
  {
    \exp_args:Nnxx \@@_backend_PageResources_gput:nnn {Pattern}{\tl_trim_spaces:n{#2}}{#3}
  }
\cs_new:Npn \@@_patch_pgfcolorspaces:w  #1/#2[#3]#4\q_stop
  {
    \exp_args:Nne
    \@@_backend_PageResources_gput:nnn {ColorSpace}{\tl_trim_spaces:n{#2}}{[#3]}
  }
%</package>
%    \end{macrocode}
% \subsection{\pkg{transparent}}
% transparent, we assume that pdfresources is loaded first.
% the code does nothing is with new-transparent ...
%    \begin{macrocode}
%<*package>
%label??
\hook_gput_code:nnn {begindocument/end} {pdf}
  {
    \def\TRP@addresource
      {
        \@@_backend_PageResources_gput:nnn{ExtGState}{TRP1}{<</ca~1/CA~1>>}
      }
    \def\transparent@use#1
      {
        \tl_if_exist:cF{TRP#1}
          {
            \@@_backend_PageResources_gput:nnn
              {ExtGState}
              {TRP#1}
              {<</ca~#1/CA~#1>>}
            \tl_const:cn{TRP#1}{/TRP#1~gs}
          }
       }
  }
%</package>
%    \end{macrocode}
% \subsection{\pkg{colorspace}}
% This is rather difficult as no real places to inject patches
% at first a try to avoid that its ExtGState is missing:
% it can not be avoided to recreate the objects (and so to get duplicates)
% as colorspace uses temporary macros whose contents is lost.
%    \begin{macrocode}
%<*package>
% this must be earlier, to avoid problems with luatex which has two pageresources
% lua/tex
\hook_gput_code:nnn {begindocument} {pdf}
  {
    \tl_if_exist:NT \spc@op
      {
        \def\spc@Pageresources#1{}
      }

  }

\hook_gput_code:nnn {begindocument/end} {pdf}
  {
    \tl_if_exist:NT \spc@op
      {
        \@@_backend_object_new:nn  {__spc_extgstate_op_false}{dict}
        \@@_backend_object_write:nn
          {__spc_extgstate_op_false}
          {/Type /ExtGState~/op~false~/OP~false}
        \pdfmanagement_add:nnn
          {Page/Resources/ExtGState}
          {SPCko}
          {\@@_backend_object_ref:n {__spc_extgstate_op_false}}
        \@@_backend_object_new:nn  {__spc_extgstate_op_true0}{dict}
        \@@_backend_object_write:nn
          {__spc_extgstate_op_true0}
          {/Type /ExtGState~/op~true~/OP~true~/OPM~0}%
        \pdfmanagement_add:nnn
          {Page/Resources/ExtGState}
          {SPCmz}
          {\@@_backend_object_ref:n {__spc_extgstate_op_true0}}
        \@@_backend_object_new:nn  {__spc_extgstate_op_true1}{dict}
        \@@_backend_object_write:nn
          {__spc_extgstate_op_true1}
          {/Type /ExtGState~/op~true~/OP~true~/OPM~1}%
         \pdfmanagement_add:nnn
          {Page/Resources/ExtGState}
          {SPCop}
          {\@@_backend_object_ref:n {__spc_extgstate_op_true1}}
      }
  }
%</package>
%    \end{macrocode}%
% \PrintIndex
