% \iffalse meta-comment
%
%% File: l3pdfmanagement.dtx
%
% Copyright (C) 2018-2020 The LaTeX3 Project
%
% It may be distributed and/or modified under the conditions of the
% LaTeX Project Public License (LPPL), either version 1.3c of this
% license or (at your option) any later version.  The latest version
% of this license is in the file
%
%    http://www.latex-project.org/lppl.txt
%
% This file is part of the "(experimental) pdfmanagement bundle" (The Work in LPPL)
% and all files in that bundle must be distributed together.
%
% -----------------------------------------------------------------------
%
% The development version of the bundle can be found at
%
%    https://github.com/latex3/pdfresources
%
% for those people who are interested.
%
%<*driver>
\RequirePackage{expl3}
\documentclass[full]{l3doc}
\usepackage{tabularx}
\providecommand\potentialclash{\noindent\llap{\dbend\ }}

% Fixing footnotes in  functions and variables: this should be in l3doc!
\newcommand\fixfootnote[2]{\footnotemark
  \AddToHookNext{env/#1/after}{\footnotetext{#2}}}
\AddToHook{env/function/begin}{\def\footnote{\fixfootnote{function}}}
\AddToHook{env/variable/begin}{\def\footnote{\fixfootnote{variable}}}

\begin{document}
  \DocInput{\jobname.dtx}
\end{document}
%</driver>
% \fi
%
% \title{^^A
%   The \pkg{l3pdfmanagement} package\\ Managing central PDF resources ^^A
% }
%
% \author{^^A
%  The \LaTeX3 Project\thanks
%    {^^A
%      E-mail:
%        \href{mailto:latex-team@latex-project.org}
%          {latex-team@latex-project.org}^^A
%    }^^A
% }
%
% \date{Released XXXX-XX-XX}
%
% \maketitle
% \begin{documentation}
%
% \section{\pkg{l3pdfmanagement} documentation}
% When creating a pdf a number of objects, dictionaries and entries to
% central \enquote{core} dictionaries must be created.
%
% The commands in this module offer interfaces to this core PDF dictionaries
% They unify a number of primitives like the pdftex
% registers and commands \cs{pdfcatalog}, \cs{pdfpageattr},
% \cs{pdfpagesattr}, \cs{pdfinfo}, \cs{pdfpageresources}
% and similar commands of the other backends in a backend independant way.
%
% The supported backends are pdflatex, lualatex, (x)dvipdfmx (latex, xelatex)
% and dvips with ps2pdf (not completely yet). dvips with distiller could work too
% but is untested.
%
% That the interfaces are backend independent doesn't mean that the results and even
% the compilation behavior is identical. The backends are too different to allow
% this. Some backends expand arguments e.g. in a \cs{special} while other don't.
% Some backends can insert a resource at the first compilation, while another uses
% the aux-file and a label and so needs at least two. Some backends create and
% manage resources automatically which must be managed manually by other backends.
%
% The dictionaries and resources handled by this module are inserted only
% once in a PDF or only once per page. Examples are the Catalog dictionary,
% the Info dictionary, the page resources. For these dictionaries and resources
% management by the \LaTeX{} kernel is necessary to avoid
% that packages overwrite settings from
% other packages which would lead to clashes and incompatibilities.
% It is therefore necessary that all packages which want to add content to this
% dictionaries and resources use the interface provided by this module.
%
% As these dictionaries and resources are so central for the PDF format values to these
% dictionaries are always added globally. Through the interface values
% can be added (and in many cases also removed) by users and packages,
% but the actually writing of the
% dictionary entries and resources to the PDF is handled by
% the kernel code.
%
% The interface uses as main name to address the resources \emph{Paths}
% which follow the names and structure described in the PDF reference. This
% should make it easy to identify the names needed to insert a specific
% PDF resources with the new interfaces.
% All \emph{Paths} have names starting with an uppercase letter.
%
% The following tabular summarize the \emph{Paths} and which pdftex primitive they
% replace:
%
% \begin{tabular}{ll}
%  Info                                  & \cs{pdfinfo}           \\
%  Catalog \&  various subdictionaries   & \cs{pdfcatalog}        \\
%  Pages                                 & \cs{pdfpagesattr}      \\
%  Page, ThisPage                        & \cs{pdfpageattr}       \\
%  Page/Resources/ExtGState              & \cs{pdfpageresources}  \\
%  Page/Resources/Shading                & \cs{pdfpageresources}  \\
%  Page/Resources/Pattern                & \cs{pdfpageresources}  \\
%  Page/Resources/ColorSpace             & \cs{pdfpageresources}  \\%
%  \end{tabular}
%
%  There is no \texttt{Page/Resources/Properties} dictionary in the list,
%  because this dictionary is not filled directly, but
%  managed through side effects when setting BDC-marks.
%
%  \subsection{User Commands}
%  To avoid problems with older documents the resource management of this
%  module is not activated unconditionally. The values are pushed out to the
%  dictionaries only if a boolean has been set to true. The state can be tested
%  with a conditional.
%  \begin{function}[EXP,pTF,added=2020-07-04]
%   {\pdfmanagement_if_active:}
%   This conditional tests if the resource management code is active.
%
%  \end{function}
% \begin{function}[added = 2020-04-06]
%   {\pdfmanagement_add:nnn}
%   \begin{syntax}
%     \cs{pdfmanagement_add:nnn} \Arg{resource path} \Arg{name} \Arg{value}
%   \end{syntax}
% This function puts \Arg{name} \Arg{value} in the PDF resource described by
% the symbolic name \Arg{resource path}. Technically it stores it globally in an internal
% property lists and writes it later into the right PDF dictionary\footnote{Currently all
% resources are PDF dictionaries, so resource and dictionary mean the same.}
% Which values for \Arg{resource path} exist is described in the following.
% \Arg{name} should be a PDF Name without the starting slash. Like with all
% keys used in PDF dictionaries (see the l3pdfdict module) the name is escaped
% with \cs{str_convert_pdfname:n} when stored.
% \Arg{value} should be a valid PDF value for this Name in the
% target dictionary.
%
%
% The code works with all major engines but not necessarily
% in the same way. Most importantly
% \begin{itemize}
% \item The expansion behaviour of the backends can differ. Some backends expand a
% value always fully when writing to the PDF, with other backends command names
% could end as strings in the PDF. So one should neither rely on \Arg{name}
% \Arg{value} to be expanded nor not expanded by the backend commands.
% \item The number of compilations needed can differ between the engines and
% backends. Some engines have to use labels and the aux-file to setup
% the dictionaries and so need at least two compilations to put everything
% in place.
% \end{itemize}
% \end{function}
%
%
% \begin{function}[added = 2020-04-08]
%   {\pdfmanagement_show:n }
%   \begin{syntax}
%     \cs{pdfmanagement_show:n}  \Arg{resource path}
%   \end{syntax}
%   This shows the content of the dictionary targetted by
%   \Arg{resource path} in the log and on the terminal if possible.
%
%   It is not reliable for page resources as these are filled at shipout.
%
%   It also doesn't show necessarly all the content. For example most backends add automatically
%   entries to the Info dictionary.
%\end{function}
% \begin{function}[added = 2020-04-07]
%   {
%     \pdfmanagement_remove:nn,
%   }
%   \begin{syntax}
%     \cs{pdfmanagement_remove:nn} \Arg{resource path} \Arg{name}
%   \end{syntax}
%   Removes  |/|\meta{name} and its associated \meta{value} from the
%   dictionary described with \Arg{resource path}
%   The removal is global.
%   If \meta{name} is not found no change occurs,
%   \emph{i.e}~there is no need to test for the existence of a name before
%   trying to remove it.
%   Values from the special Catalog entries where the values are collected in arrays
%   can't be removed (but should ever a use case appear it could be added).
% \end{function}
% \subsection{Description of the resource pathes}
% \subsubsection{Info: The Info dictionary}
% \begin{NOTE}{UF}
% path: Info
% The info dictionary is filled by e.g. \cs{pdfinfo}. Multiple appearances of
% \cs{pdfinfo} are concatenated, so one could end with multiple /Title or /Author entries.
% It is then viewer dependent which one is showed, so it is better to avoid this.
% We therefore setup a property which is filled and written to the info
% directory in one go. According to hyperref a few odd drivers (hvtex, dvipsone, dviwind)
% don't support arbitrary keys, but this should be handle elsewhere.
% As entries with empty content
% should be omitted we add a test. The string command should perhaps escape the argument,
% but for now we are assuming that the argument is pdf safe.
% hyperref writes to the info dictionary at the shipout of the first page --
% probably to catch the  case that \cs{title} is issued after the begin of the document.
% We are outputting at the last page -- this needs a second compilation but
% this is needed anyway.
% \end{NOTE}
% \potentialclash If the primitive commands of the engines are used too there will
% be double entries in the pdf (at least with the backend pdftex and luatex).
% How pdf viewer handles this is unpredictable.
%
%  \begin{function}
%   {
%     pdfmanagement: Info
%   }
%   \begin{syntax}
%     \cs{pdfmanagement_add:nnn} \{Info\} \Arg{name} \Arg{value}
%   \end{syntax}
%   Adds |/|\meta{name} and the \meta{value} to the Info dictionary.
%   \meta{name} should be a PDF name without the leading slash,
%   Like with all
%   keys used in PDF dictionaries (see the l3pdfdict module) the name is escaped
%   with \cs{str_convert_pdfname:n} when stored.
%   \meta{value} should be a valid pdf value. Any escaping or (re)encoding must be
%   done explicitly.
%   If a \meta{name} is used twice, only the last \meta{value}
%   set will be used. The Info dictionary is written at the end of the compilation,
%   so values can be set at any time.
%   The Info dictionary expects utf16be in the strings, so a conversion like this is
%   normally sensible:
%   \begin{verbatim}
%    \str_set_convert:Nnnn \l_tmpa_str { Grüße }{ default } {utf16/string}
%    \pdfmanagement_add:nnx {Info} {Title}{(\l_tmpa_str)}
%   \end{verbatim}
% \end{function}
%
%  The entries in Info dictionary are rather special as the engines/backends adds some
%  core entries, and changing or removing these entries is not always possible.
%
%  The special entries are
%  \begin{description}
%  \item[Producer] Added by all engines and backends. Removing the entry is only possible
%  with luatex with |\pdfvariable suppressoptionalinfo 128|. Changing is possible
%  with \cs{pdfmanagement_add:nnn} with the exception of dvips/pstopdf where the entry is
%  always something like |GPL Ghostscript 9.53.3|.
%  \item[Creator] Added by all engines and backends. Removal only possible in luatex by adding
%  16 to the bitset. Changing is possible with the management command.
%  \item[CreationDate] Added by all engines and backends. With the exception of |dvips/ps2pdf|
%  |SOURCE_DATE_EPOCH| is honored. With pdftex it is possible to suppress it with
%  |\pdfinfoomitdate = 1|, and in luatex by adding 32 to the bitset.
%  Changing is possible with the management command and will overwrite an epoch setting.
%  \item[ModDate] Added by all engines and backends with the exception of xdvipdfmx.
%  With the exception of |dvips/ps2pdf| |SOURCE_DATE_EPOCH| is honored.
%   Suppressing it is possible in pdftex with
%  |\pdfinfoomitdate = 1|, and in luatex by adding 64 to the bitset. Changing is possible with
%  the management command.
%  \item[Trapped] Added by pdftex and luatex. Removal only possible in
%  luatex by adding 256 to the bitset. Changing (and adding in the other backends) is
%  possible with the management command.
%  \item[PTEX.Fullbanner] Added by pdftex and luatex. Removal possible in pdftex with
%  |\pdfsuppressptexinfo-1|, in luatex by adding 512 to the bitset.
%  Changing is not possible.
%  \item[Title] Added by dvips/ps2pdf and set to |filename.dvi|. Removal is probably
%  not possible, but it can be overwritten with the management command.%
%  \end{description}
% \subsubsection{Pages: The \enquote{Pages} dictionary}
% \potentialclash As the content of this dictionary is written at the end it will
% in pdftex and luatex overwrite values added with the primitive commands (e.g.
% \cs{pdfpagesattr}.
% Package authors should use the management commands instead.

% By using this path with the pdfmanagement interface,
% values can be added to the /Pages object.
% This replaces for example \cs{pdfpagesattr}.
%
% \begin{function}{pdfmanagement: Pages}
%   \begin{syntax}
%     \cs{pdfmanagement_add:nnn} \{Pages\} \Arg{name} \Arg{value}
%   \end{syntax}
%
%   Adds |/|\meta{name} \meta{value} to the |/Pages| dictionary.
%   It is always stored globally. The content is written to the pdf
%   at the end of the compilation, so values can be added, changed or
%   removed until then.
%   \meta{name} should be a valid pdf name without the leading slash,
%   \meta{value} should be a valid pdf value. Any escaping or (re)encoding must
%   be done explicitly. Some backends expand the value but this should not be
%   relied on. If a \meta{name} is used twice, only the last \meta{value}
%   set will be used.
%
% \end{function}
%
% \subsubsection{\enquote{Page} and \enquote{ThisPage}}
% \begin{NOTE}{UF}
% Open is the question if one need a command to set attribute on a page by page number.
% Open is the setter for /AF (and perhaps /OutputIntents).
% See also https://tex.stackexchange.com/questions/479812/extension-of-rotating-package-to-set-pdf-rotation
% (should work now)
% \end{NOTE}
% \begin{function}[added = 2020-04-12]
%   {pdfmanagement: Page}
%   \begin{syntax}
%     \cs{pdfmanagement_add:nnn} \{Page\} \Arg{name} \Arg{value}
%   \end{syntax}
%   Values added with the path \texttt{Page} are added to the page dictionary
%   of the current page and the following pages. The current page means the page
%   on which the command is \emph{executed}. \meta{name} should be a valid pdf name
%   without the leading slash. Typical names used here are e.g.
%   \texttt{Rotate} and \texttt{CropBox}.
%   \meta{value} should be a valid pdf value.
%   Any escaping or (re)encoding must be done explicitly. Some backends expand the
%   value but this should not be relied on.
%   To avoid problems with the asynchronous page breaking
%   the command should be used after \cs{newpage} or in the header.
%   It should not be used in a float, as it will then quite probably be executed
%   on the wrong page.
%   The value is assigned directly and is always stored globally.
%   If a \meta{name} is used twice, only the last \meta{value}
%   set will be used. Names set with \cs{pdfmanagement_add:nnn}|{ThisPage}| will overwrite
%   names set with \cs{pdfmanagement_add:nnn}|{Page}| if there is a clash.
%   Values can be removed again with \cs{pdfmanagement_remove:nn}.
%   This replaces \cs{pdfpageattr}.
% \end{function}
% \begin{function}[added = 2020-04-12]
%   { pdfmanagement: ThisPage}
%   \begin{syntax}
%     \cs{pdfmanagement_add:nnn} \{ThisPage\} \Arg{name} \Arg{value}
%   \end{syntax}
%   Adds |/|\meta{name} \meta{value} at \emph{shipout} to the page dictionary of the
%   current page. Current page means here the \emph{shipout} page.
%   It is always stored globally.
%   If \Arg{name} has already a value set in the \texttt{Page}
%   dictionary it will be overwritten for this page.
%   \meta{name} should be a valid pdf name without the leading slash,
%   \meta{value} should be a valid pdf value. Any escaping or (re)encoding must be
%   done explicitly. If a \meta{name} is used twice, only the last \meta{value}
%   set will be used.
%   With the engine pdflatex (at least) a second compilation is needed.
%   Values added to \texttt{ThisPage} can not be removed. It is not possible to
%   show the content of this dictionary with \cs{pdfmanagement_show:n}.
% \end{function}
% \subsubsection{\enquote{Page/Resources}: ExtGState, ColorSpace, Shading, Pattern}
% \begin{NOTE}{UF}
% Only for pdf/luatex and xdvipdfmx backend- and pdf-code is needed to add values
% to these resources.
% With dvips the resources are added through high-level code (e.g. transparency), so the
% backend/pdf commands are no-ops.
% For every resources there is only one object. References to these objects are added to
% all pages starting from the page where the first time something has been added to the
% resource and to all XObjects. For luatex and pdftex it must be done together
% with the /Properties, see above.
% I don't see a need to set e.g. /ColorSpace page wise: preflight handles this
% fine, see experiment colorspace-resources.
% As pgf does the same, there is a need to patch it for now. Ditto for package colorspace.
% \end{NOTE}
% \begin{function}[updated = 2020-04-10]
%   {
%    pdfmanagement: Page/Resources/ExtGState,
%    pdfmanagement: Page/Resources/ColorSpace,
%    pdfmanagement: Page/Resources/Shading,
%    pdfmanagement: Page/Resources/Pattern,
%   }
%   \begin{syntax}
%     \cs{pdfmanagement_add:nnn} \{Page/Resources/\meta{resource}\} \Arg{name} \Arg{value}
%   \end{syntax}
%   Adds |/|\meta{name} \meta{value} to the page resource \meta{resource}.
%   \meta{resource} can be |ExtGState|, |ColorSpace|, |Pattern| oder |Shading|.
%   The values are always stored globally. The content is written to the pdf
%   at the end of the compilation, so values can be added until then.
%   \meta{name} should be a valid pdf name without the leading slash,
%   \meta{value} should be a valid pdf value for the resource.
%   Any escaping or (re)encoding must be done explicitly. If a \meta{name} is
%   used twice, only the last \meta{value} set will be used.
%
%   With the dvips backend the command does nothing: these resources are managed by
%   ghostscript or the distiller if e.\,g. transparency is used.
%
%   The resources are added to all pages starting with the first where something has
%   been added to a resources. That means that for example
%   all ExtGState resources are combined in one
%   dictionary object and every page with a ExtGState resource refer to this object%
%   \footnote{This is similar to how pgf handles this resources}.
%
%   \potentialclash The primitive commands (e.g. \cs{pdfpageresources})
%   to set the resources should not be used
%   together with this code as the calls will overwrite each other and values
%   will be lost. This means that currently there are clashes with the packages tikz,
%   transparent and colorspace.
% \end{function}
% \subsubsection{\enquote{Catalog} \& subdirectories}
% \begin{NOTE}{UF}
% Perhaps some tools to create the AF-file specification dictionaries is useful.
% Open for now:
% /Extensions               (dict, pdf 2.0)
% /Dests ? difference to subdict in Names?
% /DSS                      (dict, pdf 2.0)
% /Acroform/DR/ExtGState etc probably unneeded.
% \end{NOTE}
% The catalog is a central dictionary in a PDF with a number of subdictionaries.
% Entries to the top level of the catalog can be added with\\
% |\pdfmanagement_add:nnn {Catalog}|\Arg{Name}\Arg{Value}.
% Entries to subdictionaries by using in the first
% argument one of the pathes described later.
% The entries in the catalog have varying requirements regarding the
% PDF management. Some entries (like \texttt{/Lang}) are simple values
% where new values should
% overwrite existing values, other like for example \texttt{/OutputIntents}
% can contain a number of values and can be filled from more than one source.
% In some cases the values that needs to be added are not at the top-level
% but in some subsubdictionary or are actually part of an array.
% To handle the pdf management uses a variety of internal, special handlers.
%
% \potentialclash In some cases entries are added implicitly.
% For example entries to the name
% tree of the \texttt{/EmbeddedFiles} key in the \texttt{/Names} directory are
% added with the commands of the \texttt{l3pdffile} module. This clashes with
% e.g. the embedfile package which should not be used!
%
%
%   \paragraph{Entries at the top level of the catalog}
%   The Names in the following tabular are entries that are added to the
%   top level of the catalog.
%
%   If \meta{Name} gets assigned a value more than once the last one wins.
%   There is no check that the values have the correct type and format.
%   It is up to the user to ensure that the value does what is intended.
%
%   The required PDF version is only mentioned if it is larger than 1.5.
%
%   Example: |\pdfmanagement_add:nnn {Catalog}{PageMode}{/UseNone}|
%
%   \medskip
%   \noindent
%   \begin{tabularx}{\linewidth}{ll>{\raggedright\arraybackslash}X}
%    \bfseries Name          &  \bfseries Value  & \bfseries Remark \\\midrule
%    Collection              &  objref or dict     & the content should be
%     build by external packages (see eg embedfile)                   \\
%    DPartRoot               &  objref or dict     & PDF 2.0           \\
%    Lang                    &  string             & e.g. \texttt{(de-DE)} \\
%    Legal                   &  objref or dict                             \\
%    Metadata                &  objref or stream                       \\
%    NeedsRendering          &  boolean            & PDF 1.7\\
%    OpenAction              &  array (dest) or dict (action) \\
%    PageLabels              &  objref or dict     &  number tree \\
%    PageLayout              &  name               & one of /SinglePage, /OneColumn,
%                                        /TwoColumnLeft, /TwoColumnRight,
%                                        /TwoPageLeft,
%                                        /TwoPageRight \\
%    PageMode                &  name               & one of /UseNone, /UseOutlines, /UseThumbs,
%                                    /UseOC, /UseAttachments (PDF 1.6)\\
%    Perms                   &  objref or dict     & permissions\\
%    PieceInfo               &  objref or dict \\
%    SpiderInfo              &  objref or dict \\
%    StructTreeRoot          &  objref or dict \\
%    Threads                 &  objref to an array\\
%    URI                     &  objref or dict \\
%    Version                 &  name           & eg. \texttt{/1.7} \\
%    \meta{unknown}          &                 & an unknown \meta{name} will be
%                                                inserted without a warning.\\
%   \end{tabularx}
%   \par\medskip
%
%   \paragraph{Simple entries in subdictionaries of the catalog}
%   The following resource pathes have been predeclared and allow to
%   add values to the respective subdictionaries of the catalog. The
%   names of the dictionaries follow the naming and location of the dictionaries
%   in the PDF reference.
%   If \meta{Name} gets assigned two values the last one wins.
%
%   Example: |\pdfmanagement_add:nnn {Catalog/MarkInfo}{Marked}{true}|
%
%   \medskip
%   \noindent
%   \begin{tabularx}{\linewidth}{lll>{\raggedright\arraybackslash}X}
%    \bfseries Path/dictionary    & \bfseries Names & \bfseries Value  & \bfseries Remark
%    \\\midrule
%    Catalog/AA              &WC, WS, DS, WP,DP& all dict    \\
%    Catalog/AcroForm        &  NeedAppearances&  boolean & In pdf 2.0
%                                                           NeedAppearances
%                                                           is deprecated,
%                                                           it is then required
%                                                           that every widget has
%                                                           an appearance streams.\\
%                            &  SigFlags       & Integer\\
%                            &  DA             & String \\
%                            &  Q              & Integer\\
%                            &  XFA            & stream or array & pdf 1.5\\
%  Catalog/AcroForm/DR       & \meta{name}     &            & probably unneeded \\
%  Catalog/AcroForm/DR/Font  & \meta{name}     & dict       & \\
%  Catalog/MarkInfo          & Marked          & boolean      \\
%                            & UserProperties  & boolean      \\
%                            & Suspects        & boolean      \\
%  Catalog/ViewerPreferences & HideToolbar     & boolean      \\
%                            & Direction       & /R2L or /L2R \\
%                            & \ldots          &              & many more, see the reference \\
% \end{tabularx}
%
%
% \paragraph{Catalog entries with multiple values in arrays}
% The following entries are special: Their values are arrays and
% it must be possible to append to such arrays. This means that a new
% call to set this value doesn't replace the value but appends it.
% The value is an object reference. It is sensible to declare the object
% first. E.g.
%  \begin{verbatim}
%  \pdf_object_new:nn   {pkg@intent}{dict}
%  \pdf_object_write:nn {pkg@intent}{...}
%  \pdfmanagement_add:nnx  {Catalog} {OutputIntents}{\pdf_object_ref:n {pkg@intent}}
%  \end{verbatim}
%
%  or
%  \begin{verbatim}
%  \pdf_object_unnamed_write:nn   {dict} { ... }
%  \pdfmanagement_add:nnx  {Catalog} {OutputIntents}{\pdf_object_ref_last:}
%  \end{verbatim}
%
%
%   \medskip
%   \noindent
%   \begin{tabularx}{\linewidth}{lll>{\raggedright\arraybackslash}X}
%    \bfseries Path/dictionary &\bfseries Name        & \bfseries Value   & \bfseries Remark \\\midrule
%    Catalog/AcroForm     & Fields               & object reference\\
%    Catalog/AcroForm     & CO                   & object reference\\
%    Catalog              & AF                   & object reference & PDF 2.0, associated files\\
%    Catalog/OCProperties & OCGs                 & object reference  &if there are OCProperties, OCGs and D are required.\\
%    Catalog/OCProperties & Configs              & object reference \\
%    Catalog/OCProperties & D                    & object reference & This is actually a single value as
%                                         there can be only one default.
%                                         If the value is set twice, the
%                                         second wins, and the first is
%                                         added to OCProperties/Configs.\\
%    Catalog              & OutputIntents        & object reference\\
%    Catalog              & Requirements         & object reference & PDF 1.7 \\
%    Catalog/Names        & EmbeddedFiles        & object reference & This should reference a filespec dictionary. It will
%                                                  attach the file to the file panel.
%   \end{tabularx}
%
% \end{documentation}
%
% \begin{implementation}
%  \section{\pkg{l3pdfmanagement} implementation}
%    \begin{macrocode}
%<@@=pdfmanagement>
%<*package>
%
\ProvidesExplPackage {l3pdfmanagement} {2020-04-08} {0.5}
  {Main PDF Management}
%    \end{macrocode}
% \subsection{Messages}
%    \begin{macrocode}
\msg_new:nnn  { pdfmanagement } { unknown-dict }
              { The~PDF~management~resource~'#1'~is~unknown. }

\msg_new:nnn  { pdfmanagement } { empty-value }
              { The~value~for~#1~is~empty~and~will~be~ignored }

\msg_new:nnn  { pdfmanagement } { no-removal }
              { It~is~not~possible~to~remove~values~from~'#1'.}

\msg_new:nnn  { pdfmanagement } { no-show }
              { It~is~not~possible~to~show~the~content~of~'#1'.}

\msg_new:nnn  { pdfmanagement } { show-dict }
  {
    The~PDF~resource~'#1'~
    \tl_if_empty:nTF {#2}
      { is~empty \\>~ . }
      { contains~the~pairs~(without~outer~braces): #2 . }
  }
\msg_new:nnn  { pdfmanagement } { dict-already-defined  }
  {
    The~path~'#1'~is~already~defined.
  }
\msg_new:nnn  { pdfmanagement } { inactive  }
  {
    The~PDF~resources~management~is~not~active\\
    command~'#1'~ignored.
  }
%    \end{macrocode}
% \begin{variable}{\g_@@_active_bool}
% This boolean will control the activation of the management code.
% It is used in the hooks, and in some backend files.
% % \cs{DeclareDocumentMetadata} should set it to true
%    \begin{macrocode}
\bool_new:N \g_@@_active_bool
%    \end{macrocode}
% \end{variable}
% A user predicate to test if the management code is active
%    \begin{macrocode}
\prg_new_conditional:Npnn \@@_if_active:  { p , T , F , TF }
  {
    \bool_if:NTF \g_@@_active_bool
      { \prg_return_true:  }
      { \prg_return_false: }
  }
\prg_set_eq_conditional:NNn
  \pdfmanagement_if_active: \@@_if_active: { p , T , F , TF }

%    \end{macrocode}
% We use a hook, to collect value added before the backend is ready.
%    \begin{macrocode}
\hook_new:n {pdfmanagement/add}
\cs_new_protected:Npn \pdfmanagement_add:nnn #1 #2 #3
  {
    \@@_if_active:TF
      {
        \pdfdict_if_exist:nTF { g__pdf_Core/#1 }
          {
            \hook_gput_code:nnn
              {pdfmanagement/add}
              {pdfmanagement}
              {
                \@@_handler_gput:nnn { #1 }{ #2 }{ #3 }
              }
          }
          {
            \msg_error:nnn{pdfmanagement}{unknown-dict}{#1}
          }
      }
      {
        \msg_warning:nnx {pdfmanagement}{inactive}{\tl_to_str:n {\pdfmanagement_add:nnn}}
      }
  }

\cs_generate_variant:Nn \pdfmanagement_add:nnn {nnx}
%    \end{macrocode}
% \subsection{Hooks -- shipout and end of run code}
% Code is executed in three places: At shipout of every page,
% at shipout of the last page, at the end of the document
% (after the last clearpage). Due to backend differences the code in the
% three places (and the exact timing) can be different: pdflatex/lualatex
% can execute code after the last \cs{clearpage} which the dvi-based
% drivers have to add on a shipout page.
%
% \begin{variable}
%   {
%    pdf/management/end_run,
%    pdf/management/lastpage_shipout,
%    pdf/management/thispage_shipout
%   }
% This hooks contain the code run in the three places.
%    \begin{macrocode}
\hook_new:n {pdf/management/end_run}
\hook_new:n {pdf/management/lastpage_shipout}
\hook_new:n {pdf/management/thispage_shipout}
%    \end{macrocode}
% \end{variable}
%    \begin{macrocode}
\hook_gput_code:nnn {pdf/management/thispage_shipout} {pdf}
  {
     \bool_if:NT \g_@@_active_bool
       {
         \exp_args:NV \__pdf_backend_ThisPage_gpush:n      { \g_shipout_readonly_int }
         \exp_args:NV \__pdf_backend_PageResources_gpush:n { \g_shipout_readonly_int }
       }
  }

\hook_gput_code:nnn {pdf/management/lastpage_shipout} {pdf}
  {
    \bool_if:NT \g_@@_active_bool
       {
         \__pdf_backend_PageResources_obj_gpush:          %ExtGState etc
       }
  }

\hook_gput_code:nnn {pdf/management/end_run} {pdf}
  {
    \bool_if:NT \g_@@_active_bool
       {
         \@@_Pages_gpush:            %pagesattr
         \@@_Info_gpush:             %pdfinfo
         \@@_Catalog_gpush:
       }
  }
%    \end{macrocode}
% \subsection{Naming convention}

%  Currently the following names are used: ^^A!!!!! check, compare with g_@@_gnames_seq
%  All have internally additionally a \texttt{Core} before the slash, to
%  hide the real name a bit.
%  \begin{verbatim}
%  /Info                      %    (\pdfinfo)
%  /Catalog                   %    (\pdfcatalog)
%  /Catalog/AA                %
%  /Catalog/AcroForm
%  /Catalog/OCProperties
%  /Catalog/OutputIntents
%  /Catalog/AcroForm/DR
%  /Catalog/AcroForm/DR/Font
%  /Catalog/MarkInfo
%  /Catalog/ViewerPreferences
%  /Pages                     %    (\pagesattr)
%  /Page                      %    (\pageattr)
%  /ThisPage                  %    (\pageattr)
%  /backend_PageN/Resources/Properties % this is only internal.
%  /Page/Resources/ExtGState
%  /Page/Resources/ColorSpace
%  /Page/Resources/Pattern
%  /Page/Resources/Shading
%  /Page/Resources/Properties
%  /Xform/Resources/Properties
%  \end{verbatim}

%  \begin{macro}{
%                \@@_handler_gput:nnn,
%                \@@_get:nnN,
%                \@@_gremove:nn,
%                \@@_show:n
%                }
%  \cs{@@_handler_gput:nnn} is the main command to fill the dictionaries.
%  In simple cases it directly fill the property list, but if a handler exists
%  this is called. It is important to use it only in places where this make sense.
%
%    \begin{macrocode}

%global
\cs_new_protected:Npn \@@_handler_gput:nnn #1 #2 #3  %#1 dict, #2 name, #3 value
  {
    \tl_if_empty:nTF { #3 }
      {
        \msg_none:nnn { pdfmanagement }{ empty-value }{ /#1/#2 }
      }
      {
        \pdfdict_if_exist:nTF { g__pdf_Core/#1 }
          {
            \cs_if_exist:cTF
              { @@_handler/#1/?_gput:nn } %general, name independant handler
              { \use:c {@@_handler/#1/?_gput:nn} {#2} {#3} }
              {
                \cs_if_exist:cTF
                  { @@_handler/#1/#2_gput:n }
                  { \use:c {@@_handler/#1/#2_gput:n} {#3} } %special handler
                  {
                    \exp_args:Nnx
                    \prop_gput:cnn
                      { \__kernel_pdfdict_name:n { g__pdf_Core/#1 } }
                      { \str_convert_pdfname:n { #2 } }
                      { #3 }
                  }
              }
          }
          {
            \msg_error:nnn { pdfmanagement } { unknown-dict } { #1 }
          }
      }
  }


\cs_generate_variant:Nn \@@_handler_gput:nnn {nxx}

\cs_new_protected:Npn \@@_get:nnN  #1 #2 #3 %path,key,macro
  {
    \exp_args:Nnx
    \prop_get:cnN
      { \__kernel_pdfdict_name:n { g__pdf_Core/#1 } }
      { \str_convert_pdfname:n {#2} } #3
  }


\cs_new_protected:Npn \@@_handler_gremove:nn #1 #2 %path,key
  {
    \pdfdict_if_exist:nTF { g__pdf_Core/#1 }
          {
            \cs_if_exist:cTF
              { @@_handler/#1/?_gremove:n } %general, name independant handler
              { \use:c {@@_handler/#1/?_gremove:n} {#2} }
              {
                \cs_if_exist:cTF
                  { @@_handler/#1/#2_gremove: }
                  { \use:c {@@_handler/#1/#2_gremove:} } %special handler
                  {
                    \exp_args:Nnx
                    \prop_gremove:cn
                      { \__kernel_pdfdict_name:n { g__pdf_Core/#1 } }{ \str_convert_pdfname:n {#2} }
                  }
              }
          }
          {
            \msg_error:nnn { pdfmanagement } { unknown-dict } { #1 }
          }
  }

\cs_new_protected:Npn \@@_gremove:nn #1 #2 %path,key
  {
    \pdfdict_if_exist:nTF { g__pdf_Core/#1 }
          {
            \exp_args:Nnx
            \prop_gremove:cn
              { \__kernel_pdfdict_name:n { g__pdf_Core/#1 } }{ \str_convert_pdfname:n{#2} }
          }
          {
            \msg_error:nnn { pdfmanagement } { unknown-dict } { #1 }
          }
  }


\cs_new_protected:Npn \@@_show:Nn #1#2
  {
    \cs_if_exist:cTF
      { @@_handler/#2/?_show: } %general, name independant handler
      { \use:c {@@_handler/#2/?_show:} }
      {
        \prop_if_exist:cTF { \__kernel_pdfdict_name:n { g__pdf_Core/#2 } }
          {
             #1
               { pdfmanagement } { show-dict }
               { \tl_to_str:n {#2} }
               { \prop_map_function:cN {\__kernel_pdfdict_name:n { g__pdf_Core/#2 }} \msg_show_item:nn }
               { } { }
          }
          {
            #1 { pdfmanagement } { unknown-dict } {#2}{}{}{}
          }
       }
  }

\cs_new_protected:Npn \@@_show:n #1  %path
  {
    \prop_show:c { \__kernel_pdfdict_name:n { g__pdf_Core/#1 } }
  }
%    \end{macrocode}
% \end{macro}
%
%
%    \begin{macrocode}
\cs_new_protected:Npn \pdfmanagement_show:n #1
  {
    \@@_show:Nn \msg_show:nnxxxx {#1}
  }
%    \end{macrocode}
%    \begin{macrocode}
\cs_new_protected:Npn \pdfmanagement_remove:nn #1 #2
  {
    \pdfdict_if_exist:nTF { g__pdf_Core/#1 }
      {
        \@@_handler_gremove:nn { #1 }{ #2 }
      }
      {
        \msg_error:nnn{pdfmanagement}{unknown-dict}{#1}
      }
  }
%    \end{macrocode}
%    \begin{macrocode}
\cs_new_protected:Npn \pdfmanagement_get:nnN #1 #2 #3
  {
    \pdfdict_if_exist:nTF { g__pdf_Core/#1 }
      {
        \@@_get:nnN { #1 }{ #2 } #3
      }
      {
        \msg_error:nnn{pdfmanagement}{unknown-dict}{#1}
      }
  }
%    \end{macrocode}
% \subsection{The Info dictionary}
% Initialization of the dictionary:
%    \begin{macrocode}
\pdfdict_new:n { g__pdf_Core/Info}
%    \end{macrocode}
%
% \begin{macro}{\@@_Info_gpush:}
% \cs{@@_Info_gpush:} is the command that outputs the info dictionary (currently
% in the end-of-run hooks).
%    \begin{macrocode}
% push to the register command / issue the special
\cs_new_protected:Npn \@@_Info_gpush:
  {
    \prop_map_function:cN  { \__kernel_pdfdict_name:n { g__pdf_Core/Info} } \__pdf_backend_info_gput:nn
    \prop_gclear:c { \__kernel_pdfdict_name:n { g__pdf_Core/Info} }
  }
%    \end{macrocode}
% \end{macro}
% \subsection{The Pages dictionary code}
% \begin{NOTE}{UF}
% The register is normally used only a few times in a document, so it would be
% okay to update the register/add the special at every change,
% but with dvips/dvipdfmx this would disable removing entries.
% So we issue the push code only at the end of the document.
% \end{NOTE}
% At first the initialisation
%     \begin{macrocode}
\pdfdict_new:n { g__pdf_Core/Pages}
%    \end{macrocode}
%
% \begin{macro}{\@@_Pages_gpush:}
% This is the command that outputs the Pages dictionary. It is used
% at the end of the document in \cs{g__pdf_backend_end_run_tl}
%    \begin{macrocode}
% push to the register command / issue the special
\cs_new_protected:Npn \@@_Pages_gpush:
  {
    \exp_args:Nx \__pdf_backend_Pages_primitive:n
      {
        \pdfdict_use:n { g__pdf_Core/Pages}
      }
  }

%    \end{macrocode}
% \end{macro}
% \subsection{The Page and ThisPage dictionary}
% At first the initialisation.
%    \begin{macrocode}
\pdfdict_new:n { g__pdf_Core/Page }
\pdfdict_new:n { g__pdf_Core/ThisPage }

%handler for pdfmanagement
\cs_new_protected:cpn { @@_handler/Page/?_gput:nn } #1 #2
  {
    \__pdf_backend_Page_gput:nn { #1 }{ #2 }
  }
% remove:
\cs_new_protected:cpn { @@_handler/Page/?_gremove:n } #1
  {
    \__pdf_backend_Page_gremove:n { #1 }
  }

% handler for pdfmanagement
\cs_new_protected:cpn { @@_handler/ThisPage/?_gput:nn } #1 #2
  {
    \prop_gput:cnn  { \__kernel_pdfdict_name:n { g__pdf_Core/ThisPage } }{ #1 } { #2 }
    \bool_if:NT \g_@@_active_bool
      {
        \__pdf_backend_ThisPage_gput:nn { #1 }{ #2 }
      }
  }

\cs_new_protected:cpn { @@_handler/ThisPage/?_gremove:n } #1
  {
    \msg_warning:nnn { pdfmanagement } { no-removal }{ThisPage}
  }

\cs_new_protected:cpn { @@_handler/ThisPage/?_show: }
  {
    \msg_warning:nnn { pdfmanagement } { no-show }{ThisPage}
  }

%    \end{macrocode}
% \subsubsection{\enquote{Page/Resources}: ExtGState, ColorSpace, Shading, Pattern}
%    \begin{macrocode}
\clist_const:Nn \c_@@_PageResources_clist
  {
    ExtGState,
    ColorSpace,
    Pattern,
    Shading,
  }

\clist_map_inline:Nn \c_@@_PageResources_clist
  {
    \pdfdict_new:n { g__pdf_Core/Page/Resources/#1}
  }
%
% setter:  #1 is the name of the resource
\cs_new_protected:cpn { @@_handler/Page/Resources/ExtGState/?_gput:nn } #1 #2
  {
    \__pdf_backend_PageResources_gput:nnn {ExtGState} { #1 }{ #2 }
  }

\cs_new_protected:cpn { @@_handler/Page/Resources/ColorSpace/?_gput:nn } #1 #2
  {
    \__pdf_backend_PageResources_gput:nnn {ColorSpace} { #1 }{ #2 }
  }

\cs_new_protected:cpn { @@_handler/Page/Resources/Shading/?_gput:nn } #1 #2
  {
    \__pdf_backend_PageResources_gput:nnn {Shading} { #1 }{ #2 }
  }

\cs_new_protected:cpn { @@_handler/Page/Resources/Pattern/?_gput:nn } #1 #2
  {
    \__pdf_backend_PageResources_gput:nnn {Pattern} { #1 }{ #2 }
  }
%    \end{macrocode}
% \subsubsection{\enquote{Catalog}}
% The catalog has mixed entries: toplevel, subdictionaries, and entries
% which must build arrays.
% \begin{variable}[added=2019-08-24]
%     {
%      \c_@@_Catalog_toplevel_clist,
%      \c_@@_Catalog_sub_clist,
%      \c_@@_Catalog_seq_clist,
%     }
%  This variables hold the list of the various types of entries. With it
%  the various \verb=_gput= commands are generated.
%  \end{variable}
%  \begin{macro}{ \@@_catalog_XX_gput:n }
% Various commands to handle subentries and special cases.
%    \begin{macrocode}
\pdfdict_new:n { g__pdf_Core/Catalog}

\clist_const:Nn \c_@@_Catalog_toplevel_clist
  {
    Collection,
    DPartRoot,
    Lang,
    Legal,
    Metadata,
    NeedsRendering,
    OCProperties/D,
    OpenAction,
    PageLabels,
    PageLayout,
    PageMode,
    Perms,
    PieceInfo,
    SpiderInfo,
    StructTreeRoot,
    Threads,
    URI,
    Version
  }

\clist_const:Nn \c_@@_Catalog_sub_clist
  {
    AA,
    AcroForm,
    AcroForm/DR,
    AcroForm/DR/Font,
    MarkInfo,
    ViewerPreferences,
    OCProperties
  }

\clist_map_inline:Nn \c_@@_Catalog_sub_clist
  {
    \pdfdict_new:n { g__pdf_Core/Catalog/#1}
  }


\clist_const:Nn \c_@@_Catalog_seq_clist
  {
    AF,
    OCProperties/OCGs,
    OCProperties/Configs,
    OutputIntents,
    Requirements,
    AcroForm/Fields,
    AcroForm/CO
  }



\clist_map_inline:Nn \c_@@_Catalog_seq_clist
 {
   \seq_new:c { g_@@_/Catalog/#1_seq } % new name later
   \cs_new_protected:cpn { @@_handler/Catalog/#1_gput:n } ##1
     {
       \seq_gput_right:cn { g_@@_/Catalog/#1_seq } {  ##1  }
     }
 }

\cs_new_protected:cpn { @@_handler/Catalog/OCProperties/D_gput:n } #1
  {
    \seq_gput_left:cn
      { g_@@_/Catalog/OCProperties/Configs_seq }
      {  #1  }
  }
%    \end{macrocode}
% \end{macro}
% % \paragraph {Building the catalog: Push order}
% \begin{macro}{\@@_Catalog_gpush:}
%    \begin{macrocode}
\cs_new_protected:Npn \@@_Catalog_gpush:
  {
    \use:c { @@_/Catalog/AA_gpush: }
    \use:c { @@_/Catalog/AcroForm_gpush: }
    \use:c { @@_/Catalog/AF_gpush: }
    \use:c { @@_/Catalog/MarkInfo_gpush: }
    \pdfmeta_standard_verify:nT {Catalog_no_OCProperties}
      {
        \use:c { @@_/Catalog/OCProperties_gpush: }
      }
    \use:c { @@_/Catalog/OutputIntents_gpush: }
    \use:c { @@_/Catalog/Requirements_gpush: }
    \use:c { @@_/Catalog/ViewerPreferences_gpush: }
    % output the single values:
    \prop_map_function:cN { \__kernel_pdfdict_name:n { g__pdf_Core/Catalog} }  \__pdf_backend_catalog_gput:nn
    % output names tree:
    \use:c { @@_/Catalog/Names/EmbeddedFiles_gpush: }
  }
%    \end{macrocode}
% \end{macro}
% \paragraph{Building catalog entries: AA}
% \begin{macro}{\@@_/Catalog/AA_gpush:}
%    \begin{macrocode}
\cs_new_protected:cpn { @@_/Catalog/AA_gpush: }
  {
    \prop_if_empty:cF
     { \__kernel_pdfdict_name:n { g__pdf_Core/Catalog/AA } }
     {
       \__pdf_backend_object_new:nn  { g_@@_/Catalog/AA_obj } { dict }
       \__pdf_backend_object_write:nx
            { g_@@_/Catalog/AA_obj }
            { \pdfdict_use:n { g__pdf_Core/Catalog/AA } }
       \exp_args:Nnx
         \__pdf_backend_catalog_gput:nn
           {AA}
           {
             \__pdf_backend_object_ref:n { g_@@_/Catalog/AA_obj }
           }
     }
  }
%    \end{macrocode}
% \end{macro}
%
% \paragraph{Building catalog entries: AcroForm}
% This is the most complicated case.
% The entries is build from
% /Catalog/AcroForm/Fields  (array),
% /Catalog/AcroForm/CO      (array),
% /Catalog/AcroForm/DR/Font (dict),
% /Catalog/AcroForm/DR      (dict),
% /Catalog/AcroForm
%
% \begin{macro}{\@@_/Catalog/AcroForm_gpush:}
%    \begin{macrocode}
\cs_new_protected:cpn { @@_/Catalog/AcroForm_gpush: }
  {
    \seq_if_empty:cF { g_@@_/Catalog/AcroForm/Fields_seq }
      {
        \__pdf_backend_object_new:nn  { g_@@_/Catalog/AcroForm/Fields_obj } { array }
        \__pdf_backend_object_write:nx
            { g_@@_/Catalog/AcroForm/Fields_obj }
            { \seq_use:cn { g_@@_/Catalog/AcroForm/Fields_seq } {~} }
        \exp_args:Nnnx
          \prop_gput:cnn %we have to use \prop here to avoid the handler ...
            { \__kernel_pdfdict_name:n { g__pdf_Core/Catalog/AcroForm } }
            { Fields }
            { \__pdf_backend_object_ref:n { g_@@_/Catalog/AcroForm/Fields_obj } }
      }
    \seq_if_empty:cF { g_@@_/Catalog/AcroForm/CO_seq }
      {
        \__pdf_backend_object_new:nn  { g_@@_/Catalog/AcroForm/CO_obj } { array }
        \exp_args:Nnx
          \__pdf_backend_object_write:nn
            { g_@@_/Catalog/AcroForm/CO_obj }
            { \seq_use:cn { g_@@_/Catalog/AcroForm/CO_seq } {~} }
        \exp_args:Nnnx
          \prop_gput:cnn %we have to use \prop here to avoid the handler ...
            { \__kernel_pdfdict_name:n { g__pdf_Core/Catalog/AcroForm } }
            { CO }
            { \__pdf_backend_object_ref:n { g_@@_/Catalog/AcroForm/CO_obj } }
      }
     \prop_if_empty:cF { \__kernel_pdfdict_name:n { g__pdf_Core/Catalog/AcroForm/DR/Font}}
       {
         \__pdf_backend_object_new:nn { g_@@_/Catalog/AcroForm/DR/Font_obj } {dict}
         \exp_args:Nnx
           \__pdf_backend_object_write:nn
             { g_@@_/Catalog/AcroForm/DR/Font_obj }
             { \pdfdict_use:n { g__pdf_Core/Catalog/AcroForm/DR/Font } }
         \exp_args:Nnnx
           \prop_gput:cnn %we have to use \prop here to avoid the handler ...
             { \__kernel_pdfdict_name:n { g__pdf_Core/Catalog/AcroForm/DR } }
             { Font }
             { \__pdf_backend_object_ref:n { g_@@_/Catalog/AcroForm/DR/Font_obj } }
       }
     \prop_if_empty:cF { \__kernel_pdfdict_name:n { g__pdf_Core/Catalog/AcroForm/DR}}
       {
         \__pdf_backend_object_new:nn { g_@@_/Catalog/AcroForm/DR_obj } {dict}
         \exp_args:Nnx
           \__pdf_backend_object_write:nn
             { g_@@_/Catalog/AcroForm/DR_obj }
             { \pdfdict_use:n { g__pdf_Core/Catalog/AcroForm/DR } }
         \exp_args:Nnnx
           \prop_gput:cnn %we have to use \prop here to avoid the handler ...
             { \__kernel_pdfdict_name:n { g__pdf_Core/Catalog/AcroForm } }
             { DR }
             { \__pdf_backend_object_ref:n { g_@@_/Catalog/AcroForm/DR_obj } }
       }
     \prop_if_empty:cF { \__kernel_pdfdict_name:n { g__pdf_Core/Catalog/AcroForm} }
       {
         \__pdf_backend_object_new:nn { g_@@_/Catalog/AcroForm_obj } {dict}
         \exp_args:Nnx
           \__pdf_backend_object_write:nn
             { g_@@_/Catalog/AcroForm_obj }
             { \pdfdict_use:n { g__pdf_Core/Catalog/AcroForm } }
         \exp_args:Nnnx
           \@@_handler_gput:nnn
             { Catalog }
             { AcroForm }
             { \__pdf_backend_object_ref:n { g_@@_/Catalog/AcroForm_obj } }
       }
  }

%    \end{macrocode}
% \end{macro}
%
% \paragraph{Building catalog entries: AF}
% AF is an array.
% \begin{macro}{\@@_/Catalog/AF_gpush:}
%    \begin{macrocode}
\cs_new_protected:cpn { @@_/Catalog/AF_gpush: }
  {
    \seq_if_empty:cF
     { g_@@_/Catalog/AF_seq }
     {
       \__pdf_backend_object_new:nn  { g_@@_/Catalog/AF_obj } { array }
       \exp_args:Nnx
         \__pdf_backend_object_write:nn
            { g_@@_/Catalog/AF_obj }
            { \seq_use:cn { g_@@_/Catalog/AF_seq } {~} }
       \exp_args:Nnx
         \__pdf_backend_catalog_gput:nn
           {AF}
           {
             \__pdf_backend_object_ref:n {g_@@_/Catalog/AF_obj}
           }
     }
  }
%    \end{macrocode}
% \end{macro}
% %
% \paragraph{Building catalog entries: MarkInfo}
% \begin{macro}{\@@_/Catalog/MarkInfo_gpush:}
%    \begin{macrocode}
\cs_new_protected:cpn { @@_/Catalog/MarkInfo_gpush: }
  {
    \prop_if_empty:cF
     { \__kernel_pdfdict_name:n { g__pdf_Core/Catalog/MarkInfo } }
     {
       \__pdf_backend_object_new:nn  { g_@@_/Catalog/MarkInfo_obj } { dict }
       \exp_args:Nnx
         \__pdf_backend_object_write:nn
            { g_@@_/Catalog/MarkInfo_obj }
            { \pdfdict_use:n { g__pdf_Core/Catalog/MarkInfo } }
       \exp_args:Nnx
         \__pdf_backend_catalog_gput:nn
           {MarkInfo}
           {
             \__pdf_backend_object_ref:n {g_@@_/Catalog/MarkInfo_obj}
           }
     }
  }
%    \end{macrocode}
% \end{macro}
%
%\paragraph{Building catalog entries: OCProperties}
%  This is a dictionary with three entries:
%  \begin{description}
%  \item[/OCGs] (required) An array of indirect references,
%               access needed for more than one package.
%  \item[/D] (required) a dict (given as an object name) to the default
%            configuration
%  \item[/Configs] (optional) an array of indirect references to more
%        configurations.
%  \end{description}
%  The /D entry is also a config, it is the first of the seq.
%  The overall structure is nested: a dict with arrays.
% \begin{macro}{\@@_/Catalog/OCProperties_gpush:}
%    \begin{macrocode}
% Catalog/OCProperties: OCGs + D is required
\cs_new_protected:cpn { @@_/Catalog/OCProperties_gpush: }
  {
   \int_compare:nNnT
      {
        ( \seq_count:c { g_@@_/Catalog/OCProperties/OCGs_seq }  )*
        ( \seq_count:c { g_@@_/Catalog/OCProperties/Configs_seq } )
      }
      >
      { 0 }
      {
        \__pdf_backend_object_new:nn  { g_@@_/Catalog/OCProperties_obj } { dict }
        \seq_gpop_left:cN { g_@@_/Catalog/OCProperties/Configs_seq} \l_tmpa_tl
        \exp_args:Nnx
          \__pdf_backend_object_write:nn {g_@@_/Catalog/OCProperties_obj}
            {
              /OCGs~[ \seq_use:cn { g_@@_/Catalog/OCProperties/OCGs_seq } {~} ]
              /D~\l_tmpa_tl~
              \seq_if_empty:cF { g_@@_/Catalog/OCProperties/Configs_seq }
                {
                  /Configs~
                  [ \seq_use:cn { g_@@_/Catalog/OCProperties/Configs_seq} {~} ]
                }
            }
        \exp_args:Nnx
          \__pdf_backend_catalog_gput:nn
            { OCProperties }
            { \__pdf_backend_object_ref:n {g_@@_/Catalog/OCProperties_obj} }
      }
  }
%    \end{macrocode}
% \end{macro}
%
% \paragraph{Building catalog entries: OutputIntents}
% OutputIntents is an array.
% \begin{macro}{\@@_/Catalog/OutputIntents_gpush:}
%    \begin{macrocode}
\cs_new_protected:cpn { @@_/Catalog/OutputIntents_gpush: }
  {
    \seq_if_empty:cF
     { g_@@_/Catalog/OutputIntents_seq }
     {
       \__pdf_backend_object_new:nn  { g_@@_/Catalog/OutputIntents_obj } { array }
       \exp_args:Nnx
         \__pdf_backend_object_write:nn
            { g_@@_/Catalog/OutputIntents_obj }
            { \seq_use:cn { g_@@_/Catalog/OutputIntents_seq } {~} }
       \exp_args:Nnx
         \__pdf_backend_catalog_gput:nn
           {OutputIntents}
           {
             \__pdf_backend_object_ref:n {g_@@_/Catalog/OutputIntents_obj}
           }
     }
  }
%    \end{macrocode}
% \end{macro}
%
% \paragraph{Building catalog entries: Requirements}
% Requirements  is an array.
% \begin{macro}{\@@_/Catalog/Requirements_gpush:}
%    \begin{macrocode}
\cs_new_protected:cpn { @@_/Catalog/Requirements_gpush: }
  {
    \seq_if_empty:cF
     { g_@@_/Catalog/Requirements_seq }
     {
       \__pdf_backend_object_new:nn  { g_@@_/Catalog/Requirements_obj } { array }
       \exp_args:Nnx
         \__pdf_backend_object_write:nn
            { g_@@_/Catalog/Requirements_obj }
            { \seq_use:cn { g_@@_/Catalog/Requirements_seq } {~} }
       \exp_args:Nnx
         \__pdf_backend_catalog_gput:nn
           {Requirements}
           {
             \__pdf_backend_object_ref:n { g_@@_/Catalog/Requirements_obj }
           }
     }
  }
%    \end{macrocode}
% \end{macro}
%
%
% \paragraph{Building catalog entries: ViewerPreferences}
% \begin{macro}{\@@_/Catalog/ViewerPreferences_gpush:}
%    \begin{macrocode}
\cs_new_protected:cpn { @@_/Catalog/ViewerPreferences_gpush: }
  {
    \prop_if_empty:cF
     { \__kernel_pdfdict_name:n { g__pdf_Core/Catalog/ViewerPreferences } }
     {
       \__pdf_backend_object_new:nn  { g_@@_/Catalog/ViewerPreferences_obj } { dict }
       \exp_args:Nnx
         \__pdf_backend_object_write:nn
            { g_@@_/Catalog/ViewerPreferences_obj }
            { \pdfdict_use:n { g__pdf_Core/Catalog/ViewerPreferences } }
       \exp_args:Nnx
         \__pdf_backend_catalog_gput:nn
           {ViewerPreferences}
           {
             \__pdf_backend_object_ref:n {g_@@_/Catalog/ViewerPreferences_obj}
           }
     }
  }
%    \end{macrocode}
% \end{macro}
% \paragraph{Building catalog entries: Names/EmbeddedFiles}
% \begin{NOTE}{UF}
% TODO access function for the name in the name tree
% \end{NOTE}
% \begin{macro}{ Handler}
% EmbeddedFiles is an array and needs a special handler to add values.
%    \begin{macrocode}
\pdfdict_new:n { g__pdf_Core/Catalog/Names }

\cs_new_protected:cpn { @@_handler/Catalog/Names/EmbeddedFiles_gput:n } #1
  {
    \__pdf_backend_NamesEmbeddedFiles_add:n { #1 }
  }
%    \end{macrocode}
% \end{macro}
% The entry should only be added if there are actually embedded files.
% This can be tested by checking the names_seq
% \begin{macro}{\@@_/Catalog/Names/EmbeddedFiles_gpush:}
%    \begin{macrocode}
%
\cs_new_protected:cpn { @@_/Catalog/Names/EmbeddedFiles_gpush: }
  {
    \seq_if_empty:NF \g__pdf_backend_EmbeddedFiles_seq
      {
        \exp_args:Nx \__pdf_backend_NamesEmbeddedFiles_gpush:n
          {
            \seq_use:Nn \g__pdf_backend_EmbeddedFiles_seq {~}
          }
      }
  }
%    \end{macrocode}
% \end{macro}
% \begin{macro}{@@_handler/Catalog/?_show:}
% A handler to show the catalog.
%    \begin{macrocode}
\cs_new_protected:cpn {@@_handler/Catalog/?_show:}
  {
    \iow_term:x
      {
        \iow_newline:
        The~Catalog~contains~in~the~top~level~the~single~value~entries
        \prop_map_function:cN {\__kernel_pdfdict_name:n { g__pdf_Core/Catalog }} \msg_show_item:nn
      }
    \clist_map_inline:Nn \c_@@_Catalog_seq_clist
     {
       \seq_if_empty:cF {  g_@@_/Catalog/##1_seq }
         {
           \iow_term:x
             {
               The~'##1'~array~contains~the~entries
               \seq_map_function:cN { g_@@_/Catalog/##1_seq } \msg_show_item:n
             }
         }
      }
    \clist_map_inline:Nn \c_@@_Catalog_sub_clist
      {
        \prop_if_empty:cF { \__kernel_pdfdict_name:n { g__pdf_Core/Catalog/##1 } }
          {
            \iow_term:x
              {
                The~Catalog~subdirectory~'##1'~contains~the~single~value~entries
                \prop_map_function:cN {\__kernel_pdfdict_name:n { g__pdf_Core/Catalog/##1 }} \msg_show_item:nn
              }
          }
      }
    \tl_show:x {\tl_to_str:n{\pdfmanagement_show:n{Catalog}}}
  }
%    \end{macrocode}
% \end{macro}
% \subsection{ xform / Properties }
%    \begin{macrocode}
\pdfdict_new:n { g__pdf_Core/Xform/Resources/Properties}
%    \end{macrocode}
%    \begin{macrocode}
%</package>
%    \end{macrocode}
%
% \end{implementation}
%
% \PrintIndex
