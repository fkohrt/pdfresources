% !Mode:: "TeX:DE:UTF-8:Main"
\documentclass{article}
\usepackage{l3pdf,ifxetex}
\usepackage%[patches] %doesn't work currently
{pdfresources}
\usepackage{tikz}
\ExplSyntaxOn
\pdf_uncompress:
\ExplSyntaxOff
%\usepackage[generic]
%{attachfile2}

\ExplSyntaxOn
\box_new:N  \l__pdf_backend_tmpa_box
\bool_new:N \l__pdf_backend_xform_bool
\bool_if:nT {\sys_if_engine_pdftex_p: && \sys_if_output_pdf_p: }
{
 \cs_new:Npn \__pdf_backend_xform_new:nnnn #1 #2 #3 #4
  % #1 name
  % #2 attributes
  % #3 resources
  % #4 content, not necessarly a box!
  {
   \hbox_set:Nn \l__pdf_backend_tmpa_box
    {
     \bool_set_true:N \l__pdf_backend_xform_bool
     #4
    }
   \tl_const:cx
     { c__pdf_backend_xform_wd_ \tl_to_str:n {#1} _tl }
     { \tex_the:D \box_wd:N \l__pdf_backend_tmpa_box }
   \tl_const:cx
     { c__pdf_backend_xform_ht_ \tl_to_str:n {#1} _tl }
     { \tex_the:D \box_ht:N \l__pdf_backend_tmpa_box }
   \tl_const:cx
     { c__pdf_backend_xform_dp_ \tl_to_str:n {#1} _tl }
     { \tex_the:D \box_dp:N \l__pdf_backend_tmpa_box }
   %% do we need to test if #2 and #3 are empty??
   \tex_immediate:D \tex_pdfxform:D
    ~  attr      ~ { #2 }
   %% add local properties.... !!!!!!
   %% which resources should be default? Is an argument actually needed?
    ~  resources ~ { #3 }
    \l__pdf_backend_tmpa_box
   \int_const:cn
      { c__pdf_backend_xform_ \tl_to_str:n {#1} _int }
      { \tex_pdflastxform:D }
  }
    %only for the test here
  \cs_new:Npn \xxx_pdfresources: {\tex_the:D \tex_pdfpageresources:D}
}
\sys_if_engine_luatex:T
{
 \cs_new:Npn \__pdf_backend_xform_new:nnnn #1 #2 #3 #4
  % #1 name
  % #2 attributes
  % #3 resources
  % #4 content, not necessarly a box!
  {
   \hbox_set:Nn \l__pdf_backend_tmpa_box
    {
     \bool_set_true:N \l__pdf_backend_xform_bool
     #4
    }
   \tl_const:cx
     { c__pdf_backend_xform_wd_ \tl_to_str:n {#1} _tl }
     { \tex_the:D \box_wd:N \l__pdf_backend_tmpa_box }
   \tl_const:cx
     { c__pdf_backend_xform_ht_ \tl_to_str:n {#1} _tl }
     { \tex_the:D \box_ht:N \l__pdf_backend_tmpa_box }
   \tl_const:cx
     { c__pdf_backend_xform_dp_ \tl_to_str:n {#1} _tl }
     { \tex_the:D \box_dp:N \l__pdf_backend_tmpa_box }
   %% do we need to test if #2 and #3 are empty??
   \tex_immediate:D \tex_pdfxform:D
    ~  attr      ~ { #2 }
   %% add local properties.... !!!!!!
   %% which resources should be default? Is an argument actually needed?
    ~  resources ~ { #3 }
    \l__pdf_backend_tmpa_box
   \int_const:cn
      { c__pdf_backend_xform_ \tl_to_str:n {#1} _int }
      { \tex_pdflastxform:D }
  }
  %only for the test here
  \cs_new:Npn \xxx_pdfresources: {\tex_the:D \tex_pdfvariable:D pageresources}
}
% already in pdfresources
%  \cs_new:Npn \__pdf_backend_xform_ref:n #1 %#1 name
%  \cs_new:Npn \__pdf_backend_xform_use:n #1 %#1 name
%   \cs_new:Npn  \pdf_xform_ref:n #1

 \cs_new:Npn  \pdf_xform_new:nnnn #1 #2 #3 #4
  {
   \__pdf_backend_xform_new:nnnn { #1 } { #2 } { #3 } { #4 }
  }

 \cs_new:Npn \pdf_xform_use:n #1
  {
   \__pdf_backend_xform_use:n { #1 }
  }
 % hm ...
 \cs_new:Npn \pdf_xform_wd:n #1
  {
     \tl_use:c { c__pdf_backend_xform_wd_ \tl_to_str:n {#1} _tl }
  }
 \cs_new:Npn \pdf_xform_ht:n #1
  {
     \tl_use:c { c__pdf_backend_xform_ht_ \tl_to_str:n {#1} _tl }
  }
 \cs_new:Npn \pdf_xform_dp:n #1
  {
     \tl_use:c { c__pdf_backend_xform_dp_ \tl_to_str:n {#1} _tl }
  }

\sys_if_engine_xetex:T
{
% it needs a bit testing if it really works to set the box to 0 before the special ...
% does it disturb viewing the xobject?
% what happens with the resources (bdc)? (should work as they are specials too)
\cs_new:Npn \__pdf_backend_xform_new:nnnn #1 #2 #3 #4
  % #1 name
  % #2 attributes
  % #3 resources
  % #4 content, not necessarly a box!
 {
  \int_gincr:N \g__pdf_backend_object_int
  \int_const:cn
      { c__pdf_backend_xform_ \tl_to_str:n {#1} _int }
      { \g__pdf_backend_object_int }
  \hbox_set:Nn \l__pdf_backend_tmpa_box
    {
     \bool_set_true:N \l__pdf_backend_xform_bool
     #4
    }
   \tl_const:cx
     { c__pdf_backend_xform_wd_ \tl_to_str:n {#1} _tl }
     { \tex_the:D \box_wd:N \l__pdf_backend_tmpa_box }
   \tl_const:cx
     { c__pdf_backend_xform_ht_ \tl_to_str:n {#1} _tl }
     { \tex_the:D \box_ht:N \l__pdf_backend_tmpa_box }
   \tl_const:cx
     { c__pdf_backend_xform_dp_ \tl_to_str:n {#1} _tl }
     { \tex_the:D \box_dp:N \l__pdf_backend_tmpa_box }
   \box_set_dp:Nn  \l__pdf_backend_tmpa_box { \c_zero_dim }
   \box_set_ht:Nn  \l__pdf_backend_tmpa_box { \c_zero_dim }
   \box_set_wd:Nn  \l__pdf_backend_tmpa_box { \c_zero_dim }
   \exp_args:Nx
    \__pdf_backend:n
    { 
       bxobj  ~ \__pdf_backend_xform_ref:n  { #1 }
       \c_space_tl width  ~ \pdf_xform_wd:n { #1 }
       \c_space_tl height ~ \pdf_xform_ht:n { #1 }
       \c_space_tl depth  ~ \pdf_xform_dp:n { #1 } 
    }
   \box_use_drop:N \l__pdf_backend_tmpa_box
  \exp_args:Nx\__pdf_backend:n {put ~ @resources ~<<#3>> }
  \exp_args:Nx\__pdf_backend:n{exobj ~<<#2>>}
 }
 
 \cs_new:Npn \__pdf_backend_xform_ref:n #1
  { @pdf.xform \int_use:c { c__pdf_backend_xform_ \tl_to_str:n {#1} _int } }
  
 \cs_new:Npn \__pdf_backend_xform_use:n #1
 {
  \hbox_set:Nn \l__pdf_backend_tmpa_box
   {
    \exp_args:Nx 
    \__pdf_backend:n
     {
      uxobj~ \__pdf_backend_xform_ref:n { #1 }
     }
   }
   \box_set_wd:Nn  \l__pdf_backend_tmpa_box { \pdf_xform_wd:n { #1 } }
   \box_set_ht:Nn  \l__pdf_backend_tmpa_box { \pdf_xform_ht:n { #1 } }
   \box_set_dp:Nn  \l__pdf_backend_tmpa_box { \pdf_xform_dp:n { #1 } }
   \box_use_drop:N \l__pdf_backend_tmpa_box 
 }
\cs_new:Npn \xxx_pdfresources: {/ExtGState~@pgfextgs} 
}
\ExplSyntaxOff
\begin{document}
\ExplSyntaxOn
%correct g to c, is already in master ...
\sys_if_engine_xetex:T
{
 \cs_set:Npn \__pdf_backend_object_ref:n #1
   { @pdf.obj \int_use:c { c__pdf_backend_object_ \tl_to_str:n {#1} _int } }
}

%object for bdc test
\pdf_object_new:nn   {objA}{dict}
\pdf_object_write:nn {objA}{/Type/Artifact}

A\pdf_xform_new:nnnn{FormA}{/ABC /CDE}{}{ZZZZZZZZZZZZZZ}B\par

C\pdf_xform_use:n{FormA}D \par


E\pdf_xform_new:nnnn {FormB}{}{\xxx_pdfresources: /ABC /XYZ}
 {
 \csname pdf_bdc:nn\endcsname {Span}{objA}
  Some Text    
 \csname pdf_emc:\endcsname 
 \quad \tikz\fill[opacity=0.5,red](0,0)rectangle(1,1);
 \tikz\fill[opacity=1,red](0,0)rectangle(1,1);
 }F
 
\par

G\pdf_xform_use:n{FormB}H 
\ExplSyntaxOff


\end{document}
%   \pbs_pdfxform:nnnnn
%     #1: add pgf/tikz resources (transparency, shading)? (1|0) %dvipdfmx/xetex
%     #2: used as PDF annotation appearance? (1|0)              %dvips/pdftex
%     #3: additional resources                                  %all BUT dvips
%     #4: additional dictionary entries
%     #5: savebox number

% xform: immediate or not?  -> immediate for now
% argument: a box / content -> \pdf will use content and put in a hbox
% with dvipdfmx, xform should be in a 0-box.
% xform attr:  
% pdflatex: attr keyword
% xform resources
% pdflatex: resources keyword
% pdfbase current page resource + special stuff:
% \tl_set:Nx\l_tmpa_tl{\the\pdfpageresources~#3}\tl_trim_spaces:N\l_tmpa_tl

Some Text \tikz\fill[opacity=0.5,blue](0,0)rectangle(1,1);

\newbox\testxform
\savebox\testxform{Some Text \tikz\fill[opacity=0.5,red](0,0)rectangle(1,1);}

\ExplSyntaxOn
\bool_if:nT {\sys_if_engine_pdftex_p: || \sys_if_engine_luatex_p: }
{
 \wd\testxform=0.5\wd\testxform
 %Box before:\usebox\testxform \par %\showthe\pdfpageresources
 %saving doesn't occupy space, but as it is a whatsits ...
 % transparency needs local resources ...
 % abc\immediate\pdfxform resources {\the\pdfpageresources}
%     \testxform abc \par

 \pdf_xform_new:nnnn {test}{/ABC/CDE}{\xxx_pdfresources:}{Some~Text~ \tikz\fill[opacity=0.5,red](0,0)rectangle(1,1);}

  xxx\pdf_xform_use:n {test}\par

 Box after:\usebox\testxform \par

 %we need to resave the box to get the width ...
 \savebox\testxform{Some~Text \tikz\fill[opacity=0.5,red](0,0)rectangle(1,1);}
  \fbox{
 xx
  \__pdf_backend_annotation:nnnn
    {\pdf_xform_wd:n{test}}
    {\pdf_xform_ht:n{test}}
    {\pdf_xform_dp:n{test}}
    {
     /Subtype /Stamp
     /AP << /N~\pdf_xform_ref:n{test} >>
    }
   }
}
\par


\ExplSyntaxOff

%\ifxetex %occupies space!
%xx\special{pdf:bxobj @MyStampA
%width 280pt height 0pt depth 40pt}
%%\tikz\draw[opacity=0.5](0,0)--(1,1);
%\tikz\fill[opacity=0.5,red](0,0)rectangle(1,1);
%
%My Own Stamp
% \special{pdf:put @resources~<</ExtGState @pgfextgs>>}
%\special{pdf:exobj}yy
%\fi

%\newpage

\ifxetex
\newsavebox\testxformA
\savebox\testxform{\csname pdf_bdc:nn\endcsname {Span}{objA}xxxyyy Some Text    \csname pdf_emc:\endcsname \tikz\fill[opacity=0.5,red](0,0)rectangle(1,1);}
\savebox\testxformA{\csname pdf_bdc:nn\endcsname {Span}{objA}xxxyyy Some Text    \csname pdf_emc:\endcsname \tikz\fill[opacity=0.5,red](0,0)rectangle(1,1);}
% when saving the xform object, is should be in a zero box.
% all dimensions must be there, or the content is typeset ...
% and don't forget the \the!!
% transparency resources etc must be add to the box
abc%
\bigskip
%xobjects must be defined before use ...
\ExplSyntaxOn
\box_set_dp:Nn  \testxform { \c_zero_dim }
\box_set_wd:Nn  \testxform { \c_zero_dim }
\box_set_ht:Nn  \testxform { \c_zero_dim }


A\pdf_xform_new:nnnn{MyStampB}{/StructParent~1}{/ExtGState~@pgfextgs}
{
\csname pdf_bdc:nn\endcsname {Span}{objA}xxxyyy Some Text    \csname pdf_emc:\endcsname \tikz\fill[opacity=0.5,blue](0,0)rectangle(1,1);
}B

\par
AA\pdf_xform_use:n{MyStampB}B


\ExplSyntaxOff
\begin{picture}(0,0)
  \put(0,0)
   {
    \special
     {pdf:bxobj @MyStamp
       width  \the\wd\testxformA\space
       height \the\ht\testxformA\space
       depth  \the\dp\testxformA
      }%
  %  \csname pdf_bdc:nn\endcsname {Span}{objA}xxx
    \usebox\testxform
  %  \csname pdf_emc:\endcsname
   \special{pdf:put @resources <</ExtGState @pgfextgs>>} %
   \special{pdf:exobj}}
 \end{picture}yyyy

Use xform: \fboxsep0pt \fbox{\makebox[\the\wd\testxformA][l]{\special{pdf:uxobj @MyStamp}}}

\newpage
%Use as appearance:
\fbox{
 xx
  \special
  {pdf:ann
   width  \the\dimexpr2\wd\testxform\relax\space
   height \the\dimexpr3\ht\testxform\relax\space
   depth  \the\dp\testxform
   <<
   /Type /Annot
   /Subtype /Stamp
   /AP << /N @MyStamp >>
   >>
   }
  \phantom{\usebox\testxform}xx}
abc



Use xform after: \special{pdf:uxobj @MyStamp}

\fbox{
 xx
  \special
  {pdf:ann
   width  \the\dimexpr2\wd\testxform\relax\space
   height \the\dimexpr3\ht\testxform\relax\space
   depth  \the\dp\testxform
   <<
   /Type /Annot
   /Subtype /Stamp
   /AP << /N @MyStamp >>
   >>
   }
  \phantom{\usebox\testxform}xx}
abc

%\fi
%\end{document}


\fi
\end{document} 