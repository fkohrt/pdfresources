% !Mode:: "TeX:DE:UTF-8:Main"
\RequirePackage{pdfmanagement-testphase}
\DeclareDocumentMetadata{uncompress}
\RequirePackage{expl3}
\ExplSyntaxOn
\pdf_uncompress:
\ExplSyntaxOff
\documentclass{article}
\usepackage[T1]{fontenc}
\usepackage{hyperref}
\newcommand\cs[1]{\texttt{\textbackslash #1}}

\makeatletter
{\catcode`\%=12 \gdef\hyper@space{%20}}
 \let\hyperspace\hyper@space
% \let\ \hyper@space
\makeatother
\begin{document}


%\href[encode]{www.öäü%§(.de}{blblbl}
%
%%\hypersetup{encode}
%
%%\href{www.öäü§(.de}{seöäüß}
%
%\href[encode=false]{www.abc%25(xxx.de}{seöäüß}

%\href[encode]{www.%.de}{abc}   %-> www.%25.de
%\href{www.%.de}{abc}           %-> www.%.de
%\href[encode]{www.grüße.de}{abc} %-> www.gr%C3%BC%C3%9Fe.de
%% \href{www.grüße.de}{abc} % error
%\href[encode]{grüße}{abc} %-> (gr\303\274\303\237e.pdf)
%
%\href{filexxx}{seöäüß}
%\end{document}
\section{Some background about URL's and file references with hyperref}

hyperref has three commands related to URL and file references: \cs{url},
\cs{nolinkurl} and \cs{href}. The first two take one argument,
while the last has two, the url and some free text.


\cs{url} and \cs{href} create link annotations. \cs{url} creates always an URI
type, \cs{href} creates URI, GoToR and Launch depending on the structure of the argument.

\cs{href} has to create a (in the PDF) valid url or file name from its first argument.
\cs{url} has to create a (in the PDF) valid url from its only argument and has also to print
this argument as url. \cs{nolinkurl} only prints the url.

For the printing \cs{url} and \cs{nolinkurl} rely on the url package and its \cs{Url} command.

(Expandable) commands are expanded and special chars can also be input by commands but 
beside this no conversion is done: for all input hyperref basically assumes that 
the input is already a valid percent encoded url or a valid file name. hyperref also
doesn't extend or add protocols.

As nowadays everyone is used to copy and paste links with all sorts of unicode into a browser and
they work the hyperref input is clearly rather restricted. 


With \cs{href} it is possible to extend the input methods and to allow unicode input which is then
percent encoded by the code. 

But extending the \emph{print} options for \cs{url} and \cs{nolinkurl}
is hard to impossible in pdf\LaTeX{} due to the way the url package works. 
Some chars can be added with the help of \cs{UrlSpecial} (at the cost of warnings)
but it doesn't work for every input and documenting and explaining all the edge cases is no joy.

\subsection{Links to files}

When a file is linked with \cs{href} than normally it is added as URI link. The exceptions are PDF's, 
for them PDF has the special type GoToR which allows also to link to a destination or a special page.

The rules here are 
\begin{itemize}
\item GoToR links to PDF's with ascii names in the same folder or relative pathes normally works.
\item URI links to PDF's with ascii names in the same folder or relative pathes normally works.
\item Everything else works only sometimes in some PDF viewer.
\end{itemize}

\href{pdfa-test.pdf}{test-ascii}

\ExplSyntaxOn
\tl_set:Nn \l__hyp_text_enc_file_print_tl {utf8/URI}
\ExplSyntaxOff
\href[urlencode]{grüßpdf.pdf}{test utf8/URI}

\ExplSyntaxOn
\tl_set:Nn \l__hyp_text_enc_file_print_tl {utf8/string}
\ExplSyntaxOff
\href[urlencode]{grüßpdf.pdf}{test utf8/string}


\ExplSyntaxOn
\tl_set:Nn \l__hyp_text_enc_file_print_tl {utf16/string}
\ExplSyntaxOff
\href[urlencode]{grüßpdf.pdf}{test utf16/string}


\ExplSyntaxOn
\tl_set:Nn \l__hyp_text_enc_file_print_tl {utf16/hex}
\ExplSyntaxOff
\href[urlencode]{grüßpdf.pdf}{test utf16/hex}


\hrefurl[urlencode]{grüßpdf.pdf}{test as URI}

\href[urlencode]{grüße.txt}{test} %URI, WORKS

\href{https://www.abc(blub).de}{xxx}
\end{document}


%\begingroup
%\makeatletter
%\def\url@#1{\hyper@linkurl{§#1}{xxxx#1}}
%\url{www.blub.de}
%\endgroup
%\ExplSyntaxOn\makeatletter
%\begingroup
%%\let\protect\@unexpandable@protect
%\def\blub{abc}
%\__hyp_text_pdfstring:noN
%           { §x\blub xx}
%           { \l__hyp_text_enc_uri_print_tl }
%           \l__hyp_uri_tmpa_tl
% \tl_show:N \l__hyp_uri_tmpa_tl
%\ExplSyntaxOff
%\endgroup
%\end{document}

\section{string}

\ExplSyntaxOn
\str_set:Nx\l_tmpa_str{file://www.abcstring.de\c_percent_str \c_hash_str 123}
\expandafter\href\expandafter{\l_tmpa_str}{blub}
\ExplSyntaxOff
\section{splits}

\begin{verbatim}
href creates different link types.

If there is a colon :
  -> if prefix = file: GoToR
  -> if prefix = run:  Launch
  -> if prefix empty:  GoToR
  -> rest: URI

If there is no colon:
  if there is a dot:
    -> extension: pdf  GoToR
    -> else URI
  if there is no dot: GoToR

url creates always URI

See
  -> \@hyper@readexternallink
  -> \@hyper@linkfile
\end{verbatim}

\url{run://runuri}{xxx}

\href{:emptyprefixGoToR}{xxx}

\href{file:fileprefixGoToR}{xxx}

\href{run:runprefixLaunch}{xxx}

\href{blub:blubprefixURI}{xxx}

\href{nocolonnodotGoToR}{xxx}

\href{colonURI.dot}{xxx}

\href{colonGoToR.pdf}{xxx}

%\end{document}


\section{parentheses}
These are correctly escaped
% URI(https://www.xyz.de\(blub\))
\href{https://www.xyz.de(blub)}{xxx}
\url{https://www.xyz.de(blub)}

\href{https://www.xyz.de(blub}{xxx}
\url{https://www.xyz.de(blub}

%\end{document}


\section{linebreaks}

% gives https://www.xyz\040zzz.de
\href{https://www.xyz
 zzzlinebreak.de}{xxx}
\url{https://www.xyz
zzzlinebreak.de}

% ok
\href{https://www.xyz%
zzzlinebreak.de}{xxx}
\url{https://www.xyz%
zzzlinebreak.de}

%\end{document}

\section{tilde}
%\makeatletter \def\hyper@tilde{X}\makeatother
\href{https://www.xyz.de~blub}{xxx}
\url{https://www.xyz.de~blub}

\href{https://www.xyz.de\~blub}{xxx}
\url{https://www.xyz.de\~blub}

\makeatletter
\href{https://www.xyz.de\textasciitilde blub}{xxx}
\url{https://www.xyz.de\textasciitilde blub}
\makeatother
%\end{document}


\section{space}
Print output is wrong, uri in pdf too (but this works with pdfmanagement).
\href{https://www.xyz.de blub}{xxx}
\url{https://www.xyz.de blub}

\href{https://www.xyz.de\space blub}{xxx}
\url{https://www.xyz.de\space blub}

\parbox{5cm}{\url{https://www.xyz.de\space blub}}

%\end{document}
\section{percentchar}
The percent is not escaped, but in pdfmanagement it is!!
\href{https://www.xyz.de%blub}{xxx}
\href{https://www.xyz.de\%blub}{xxx}
\url{https://www.xyz.de%blub}
\url{https://www.xyz.de\%blub}

\makeatletter
\href{https://www.xyz.de\@percentchar blub}{xxx}
\url{https://www.xyz.de\@percentchar blub}
\parbox{3cm}{\url{https://www.xyz.de\@percentchar blub}}


\makeatother
\section{non ascii}
Is expanded to commands
%\href{https://www.xyz.deöxxx}{xxx}
%\href{filewithnonasciiöäüxxx}{xxx}
%\url{https://www.xyz.deöxxx}

%loops
%\href{https://www.xyz.de§xxx}{xxx}
%\url{https://www.xyz.de§xxx}

Works more or less but is nonsense and errors with
pdfmanagement.
\fontencoding{T1}\selectfont
\href{filewithnonasciiöäüxxx}{xxx}
%\href{https://www.xyz.deöxxx}{xxx}
%\url{https://www.xyz.deöxxx}

% expanded to \\T1\\ss\040
%\href{https://www.xyz.deßxxx}{xxx}
%\url{https://www.xyz.deßxxx}


%\section{simple command}
\def\blub{abc}
\href{https://www.xyz.de\blub xxx}{xxx}
\url{https://www.xyz.de\blub xxx}


%\section{backslash}
In pdf it is doubled! \verb+\textbackslash+ loops.
\href{https://www.xyz.de\\blub}{xxx}
\url{https://www.xyz.de\\blub}
%\href{https://www.xyz.de\textbackslash blub}{xxx}
%\url{https://www.xyz.de\textbackslash blub}

\makeatletter
\href{https://www.xyz.de\@backslashchar blub}{xxx}
\url{https://www.xyz.de\@backslashchar blub}

\makeatother


\section{Tilde}

\href{https://www.xyztilde.de~blub}{xxx}
\href{https://www.xyzcmdtilde.de\~blub}{xxx}
\url{https://www.xyztilde.de~blub}
\url{https://www.xyzcmdtilde.de\~blub}
\href{https://www.xyztexttilde.de\textasciitilde blub}{xxx}
\url{https://www.xyztexttilde.de\textasciitilde blub}

\makeatletter
\href{https://www.xyzhypertilde.de\hyper@tilde blub}{xxx}
\url{https://www.xyzhypertilde.de\hyper@tilde blub}

\makeatother



\section{Math shift}
only direct input works
\href{https://www.xyzmathshift.de$blub}{xxx}
\url{https://www.xyzmathshift.de$lub}



\section{Ambersand}

\href{https://www.xyz.de&blub}{xxx}
\href{https://www.xyz.de\&blub}{xxx}
\url{https://www.xyz.de&blub}
\url{https://www.xyz.de\&blub}



\section{underscore}

\href{https://www.xyz.de_blub}{xxx}
\href{https://www.xyz.de\_blub}{xxx}
\url{https://www.xyz.de_blub}
\url{https://www.xyz.de\_blub}

\href{https://www.xyz.de\textunderscore blub}{xxx}
\url{https://www.xyz.de\textunderscore blub}





\section{hash}

\href{https://www.xyz.de#blub}{xxx}
\href{https://www.xyz.de\#blub}{xxx}
\href{filehash#blub}{xxx}
\url{https://www.xyz.de#blub}
\url{https://www.xyz.de\#blub}

\makeatletter
\href{https://www.xyz.de\hyper@hash blub}{xxx}
\url{https://www.xyz.de\hyper@hash blub}
\makeatother



\end{document} 